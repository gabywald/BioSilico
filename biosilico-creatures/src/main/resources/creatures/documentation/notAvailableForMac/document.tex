\documentclass[11pt,twoside,a4paper]{article}
% http://www-h.eng.cam.ac.uk/help/tpl/textprocessing/latex_maths+pix/node6.html symboles de math
% http://fr.wikibooks.org/wiki/Programmation_LaTeX Programmation latex (wikibook)
%=========================== En-Tete =================================
%--- Insertion de paquetages (optionnel) ---
\usepackage[french]{babel}   % pour dire que le texte est en fran{\'e}ais
\usepackage{a4}	             % pour la taille   
\usepackage[T1]{fontenc}     % pour les font postscript
\usepackage{epsfig}          % pour gerer les images
%\usepackage{psfig}
\usepackage{amsmath, amsthm} % tres bon mode mathematique
\usepackage{amsfonts,amssymb}% permet la definition des ensembles
\usepackage{float}           % pour le placement des figure
\usepackage{verbatim}

\usepackage{longtable} % pour les tableaux de plusieurs pages

\usepackage[table]{xcolor} % couleur de fond des cellules de tableaux

\usepackage{lastpage}

\usepackage{multicol} % pour {\'e}crire dans certaines zones en colonnes : \begin{multicols}{nb colonnes}...\end{multicols} 

% \usepackage[top=1.5cm, bottom=1.5cm, left=1.5cm, right=1.5cm]{geometry}
% gauche, haut, droite, bas, entete, ente2txt, pied, txt2pied
\usepackage{vmargin}
\setmarginsrb{1.00cm}{1.00cm}{1.00cm}{1.00cm}{15pt}{3pt}{50pt}{20pt}

\usepackage{lscape} % changement orientation page
%\usepackage{frbib} % enlever pour obtenir references en anglais
% --- style de page (pour les en-tete) ---
\pagestyle{empty}

\def\txtTITLE{TITRE} %%%%% !! TITRE !! %%%%%
\def\imgCORNER{\includegraphics[width=0.25cm]{../../../imgGraphics/logos/glider/logo-glider.png}}

\def\imgGLIDERLEFTT{\includegraphics[width=1.95cm]{../../../imgGraphics/logos/glider/logo-glider-left.png}}
\def\imgGLIDERRIGHT{\includegraphics[width=1.95cm]{../../../imgGraphics/logos/glider/logo-glider-right.png}}

\def\imgGLIDERLEFTTsmall{\includegraphics[width=0.25cm]{../../../imgGraphics/logos/glider/logo-glider-left.png}}
\def\imgGLIDERRIGHTsmall{\includegraphics[width=0.25cm]{../../../imgGraphics/logos/glider/logo-glider-right.png}}

%--- Definitions de nouvelles couleurs ---
\definecolor{verylightgrey}{rgb}{0.8,0.8,0.8}
\definecolor{verylightgray}{gray}{0.80}
\definecolor{lightgrey}{rgb}{0.6,0.6,0.6}
\definecolor{lightgray}{gray}{0.6}

% % % en-tete et pieds de page configurables : fancyhdr.sty

% http://www.trustonme.net/didactels/250.html

% http://ww3.ac-poitiers.fr/math/tex/pratique/entete/entete.htm
% http://www.ctan.org/tex-archive/macros/latex/contrib/fancyhdr/fancyhdr.pdf
%% \usepackage{fancyhdr}
%% \pagestyle{fancy}
%% \renewcommand{\chaptermark}[1]{\markboth{#1}{#1}}
%% \renewcommand{\sectionmark}[1]{\markright{\thesection\ #1}}
%% \fancyhf{}
%% \fancyhead[LE,RO]{\bfseries\thepage}
%% \fancyhead[LO]{\bfseries\rightmark}
%% \fancyhead[RE]{\bfseries\leftmark}
%% \fancyfoot[LE]{\thepage /\pageref{LastPage} \hfill
%% 	\scriptsize{\txtTITLE} % TITLE
%% \hfill \imgGLIDERLEFTTsmall }
%% \fancyfoot[RO]{\imgGLIDERRIGHTsmall \hfill
%% 	\scriptsize{\txtTITLE} % TITLE
%% \hfill \thepage /\pageref{LastPage}}
%% \renewcommand{\headrulewidth}{0.5pt}
%% \renewcommand{\footrulewidth}{0.5pt}
%% \addtolength{\headheight}{0.5pt}
%% % \fancypagestyle{plain}{
%% 	% \fancyhead{}
%% 	% \renewcommand{\headrulewidth}{0pt}
%% % }

\usepackage{lettrine}
\usepackage{fancybox}

\title{\txtTITLE}
\date{ --- }

%============================= Corps =================================
\begin{document}

%% \fancypagestyle{plain}

\setlength\parindent{0pt} % \noindent for all document

-- https://groups.google.com/group/alt.games.creatures/browse\_thread/thread/f44ef75ec142c485/c1b13167542f6801 -- 

\begin{multicols*}{2}
	 \footnotesize
 	
not available for MAC?
	
	  	Messages 1 -- 25 sur 112 -- Tout r{\'e}duire  --  Traduire tous les contenus en Fran\c{c}ais 	  Plus r{\'e}cents >
	
		
\textbf{KRR --- \emph{\texttt{12 feb 1998, 10:00}}}~\\

I thought that i read somewhere that the new creatures wont be available for MACs ..    is this true?    just curious..

kim
	
	
		
\textbf{David ``I Don't Like SPAM'' Wood --- \emph{\texttt{12 feb 1998, 10:00}}}~\\

KRR wrote:
\emph{> I thought that i read somewhere that the new creatures wont be available for MACs ..    is this true?    just curious..}~\\

You heard it from me, in all probability, as a response to Toby Simpson's comments on the program and announcement of the press release. All my statements were inferences.

They could be wrong, I could be off the mark, and they're working up a Mac version of Creatures 2 even as we speak. They could have gotten a full development kit, technotes from Apple, and everything they need to make the program shine...

...and I'm genetically engineering a flying pig.

\emph{> kim}~\\

--David

 
		
	
		
\textbf{Scott Schilz --- \emph{\texttt{12 feb 1998, 10:00}}}~\\

Nope, you heard right. In an interview at \texttt{http://www.next-generation.com/} Toby Simpson definitely states "there will be no Mac version of Creatures 2."~\\
First they screw us Mac users over for upgrades and add-ons for the first game, and now they're not even gonna give give us a buggy, PC-based port of the second game! Don't get me wrong, I'd still buy Creatures 2 in a heartbeat even if it was twice as buggy as the first one. I guess the people at Cyberlife looked at all the mail they got from Mac users and said, "these people complain too much about bug fixes and upgrades. We don't wanna deal with this again!" Now, of course they're never gonna release any kind of Mac enhancements for the first Creatures because they're too busy working on Creatures 2!!!~\\
Just a little pissed,~\\
Scott

-- ~\\
Austin: "Hey, there you are!" ~\\
Man: "Well, Hi! Do I know you?" ~\\
Austin: "No, but that's where you are, you're there!" ~\\
-- Austin Powers ~\\

Scott Schilz sasch...@earthlink.net s...@goplay.com

 
		
	
		
\textbf{Karma --- \emph{\texttt{12 feb 1998, 10:00}}}~\\

David "I Don't Like SPAM" Wood wrote: ~\\
> According to Toby, in his interview at \texttt{http://www.next-generation.com/} ~\\

---
Should we expect a hybrid (Mac and PC) release for the sequel?
Simpson: There are no plans for a Macintosh version of Creatures 2.
---

The word straight from the Grendel's mouth... ~\\
--  ~\\
Karma of NORN ~\\
(aka Andrea Hearn) ~\\

Creatures on the Web ~\\
\texttt{http://www.budget.net/\textasciitilde tonyjett} ~\\

Co-Founder: ~\\
Society for the Prevention of Cruelty to Grendels (SPCG) ~\\
\texttt{http://www.budget.net/\textasciitilde tonyjett}/spcg.html ~\\

----BEGIN\_C-ADD\_CODE\_BLOCK---- ~\\
Version Number: 1.2.1 ~\\
AO/P d(-) s:+ a- x- t-:t++ c+>++ A S- ~\\
rC/c-/B/N+/E1,2+,3?,4++,5+,6++,7,8?,9?/O++/o+:,NornHist ~\\
W+ P--->+ Gd-:,Quake/s-:,Civ2/p*:/o+:,Aquazone ~\\
gQ+++/W1+,2---,3+,4+,5-,6+ ~\\
-----END\_C-ADD\_CODE\_BLOCK----- ~\\

 
		
	
		
\textbf{RudeDog --- \emph{\texttt{13 feb 1998, 10:00}}}~\\

{snip}

\emph{> According to Toby, in his interview at \emph{http://www.next-generation.com/}}~\\

\emph{>---}~\\
\emph{> Should we expect a hybrid (Mac and PC) release for the sequel?}~\\
\emph{> Simpson: There are no plans for a Macintosh version of Creatures 2.}~\\
\emph{>---}~\\

\emph{> The word straight from the Grendel's mouth...}~\\
\emph{>--}~\\
\emph{> Karma of NORN}~\\
\emph{>(aka Andrea Hearn)}~\\

Even though I don't use a Mac, I think this sucks big time!! ~\\

RudeDog ~\\
--- ~\\
:growling a CyberLife: ~\\

 
		
	
		
\textbf{rajamaki --- \emph{\texttt{13 feb 1998, 10:00}}}~\\

r...@mindspring.com (KRR) wrote: ~\\

\emph{> I thought that i read somewhere that the new creatures wont be available for MACs ..    is this true?    just curious..}~\\
\emph{> kim}~\\

Yep, its true <grrrr...> Toby said so in an interview at \texttt{www.next-generation.com}.  To be honest I don't really blame them, they couldn't even make the first one properly, (its still Creatures 0.92 or something), so imagine how they'd mess up 2. I just really, really, really, wish they'd finish what they started and fix all the bugs in the Mac version before they go ahead and start planning a whole new release. But I guess that's just too much to ask <sigh> ~\\

/Sandy ~\\
------------ ~\\
MacNorns -- \texttt{http://www.crosswinds.net/brussels/\textasciitilde rajamaki} ~\\

-- -- -- -- -- == Posted via Deja News, The Leader in Internet Discussion == -- -- -- -- --  ~\\
\texttt{http://www.dejanews.com/}   Now offering spam-free web-based newsreading ~\\

 
		
	
		
\textbf{Toby Simpson --- \emph{\texttt{13 feb 1998, 10:00}}}~\\

Everyone is correct. There will not be a Macintosh version of Creatures 2. For that to be the case, the press, publishers, retailers and developer would need to believe it was a viable product. Sadly, and with great regret, I have to confirm that this is not the case. ~\\

We do not have an internal Macintosh development department. Finding quality Mac programmers prepared to work in-house is like finding Unicorn horns -- you simply don't find them lying around. The costs involved in producing Mac C2 would be in excess of the costs for producing the Windows version. With this in mind, it obviously makes little commercial sense. The majority of our user base would much rather that we invested extra time and money in ensuring that our primary platform version was the very best it could possibly be. ~\\

We've constantly been in a no-win situation with regards to the Macintosh. We developed, tested and released a Mac version of Creatures 1 in good faith. This product passed the same quality assurance tests that the Windows build did. In the gap between this pass and the USA release (about 8 months), system 8 appeared, and this inserted so many spanners in the cross-development tools we used that I'm surprised it worked at all. None of this could have either been predicted or avoided short of coding the Mac version separately from scratch. ~\\

Given the nature of the Creatures code, trying to recode around the cross development platforms would be more work than we could justify from a cost/resource point of view. We are not a charity, and we need to make money otherwise there will be no more Creatures products for *anyone*. I'm anxious to ensure that ALL of the software we release is of the same high quality as the Windows build of Creatures. Given this, and market demand, we have decided that a Macintosh version of Creatures 2 is not possible. Despite claims from one message in this thread that "I'd rather have a buggy Creatures 2 than no Creatures 2 at all" (sure, sure), I am *NOT* prepared to develop and release software that is not up to standard, full stop. Macintosh Creatures 1 was entirely reliable on all test platforms we used. As my mother once said: "Just when you are about to make ends meet, someone moves the ends." Wise words. ~\\

The majority of feedback we have had from Mac users is both constructive and helpful, and has led to documentation of many work-arounds for System 8. We've invested considerable time and money into trying to address the issues that users have found (particularly with System 8). However, this subject is now closed. There will not be a Macintosh version of Creatures 2. ~\\

We *will* continue to help customers with problems on both Mac and Windows versions of Creatures 1, regardless of C2 release. ~\\

Toby ~\\
--  ~\\
Toby Simpson ~\\
Executive Producer/Manager -- Creatures Products ~\\
CyberLife Technology Ltd. ~\\
\texttt{www.creatures.co.uk} ~\\

 
		
	
		
\textbf{David ``I Don't Like SPAM'' Wood --- \emph{\texttt{13 feb 1998, 10:00}}}~\\

Scott Schilz wrote: ~\\
\emph{> Nope, you heard right. In an interview at \texttt{http://www.next-generation.com/} Toby Simpson definitely states "there will be no Mac version of Creatures 2."}~\\

Yep, I found that too at last. It sounds like a really nice program. Not enough inducement for me to get a PC, though.~\\

And point of information: I have a printout of that interview in front of me now, and it has Toby saying "There are \_no\_plans\_ for a Macintosh version of Creatures 2."~\\

The distinction is very important to note, and I'll explain it further below.~\\

\emph{> First they screw us Mac users over for upgrades and add-ons for the first game, and now they're not even gonna give give us a buggy, PC-based port of the second game!}~\\

Well, "screw us" isn't the right word; by my definition, "screwing us" would mean that they provided upgrades and add-ons at exorbitant prices. This is, of course, not true; they didn't provide any upgrades and add-ons in the first place. So really, they didn't as much "screw us" as "stiff us."~\\

\emph{> Don't get me wrong, I'd still buy Creatures 2 in a heartbeat even if it was twice as buggy as the first one.}~\\

If it was twice as buggy, would it even run?~\\

Judging by the reviews that Sandy posted on MacNorns, people were warned away from Creatures 1 because it was as buggy as *it was*.~\\

I sincerely hope Toby, Mark, and Cyberlife took those reviews into consideration when they decided not even to attempt any sort of Mac version.~\\

If the issue was poor Macintosh sales, there are several reasons why the reporting from organizations like the Software Publisher's Association are skewed badly toward the PC side ...and those reviews describing the program as too unstable certainly didn't help sales any.~\\

It's possible that sales would be much better if the reviews were better, and reviews would be better if they, say, resolved standing issues in software quality. And fix it so the Mac version doesn't look like a Microsoft program. >:p~\\

\emph{> I guess the people at Cyberlife looked at all the mail they got from Mac users and said, "these people complain too much about bug fixes and upgrades. We don't wanna deal with this again!"}~\\

They may very well have wanted to provide those fixes and upgrades. The problem is simpler than that: they couldn't find a decent Mac programmer with both hands and a flashlight.~\\

Here's where my point about "no plans" comes into play: if tomorrow, someone were to come up to the offices of Cyberlife, knock on the door, introduce himself as the greatest freelance Mac programmer in the world, offer a resume and sample software suite on a CD-ROM, and prove his qualifications on the spot, they might very well pull him in so fast that the sidewalk outside the front door would be yanked in by the suction.~\\

Or not.~\\

The point is, you shouldn't make judgements on Cyberlife's motivations based on a single sentence in an interview. They may want to make good, and simply not be able to.~\\

On the other hand, this does sort of put the lie to what Toby said earlier about bring the Mac version in line with the Windows version...~\\

\emph{> Now, of course they're never gonna release any kind of Mac enhancements for the first Creatures because they're too busy working on Creatures 2!!!}~\\

That's true enough. And what good will those Mac enhancements do us now anyway? Creatures 1 norns and Creatures 2 norns are incompatible. Cyberlife has said that they'll be creating a program to turn Creatures 1 norns into Creatures 2 norns. I'll bet you "five pounds to a cracked piss pot" (one of my favorite lines from Colin Dexter) that they do NOT have a way of turning Creatures 2 norns into Creatures 1 norns...~\\

\emph{> Just a little pissed,}~\\

So am I, so am I. But rattling off hasty accusations to shame them into appropriate action won't work; they will just as likely take a defensive stance, point to your attitude, and say "You're hysterical; we don't have to listen to you."~\\

No, you have to use carefully reasoned arguments and established facts (with biting sarcasm and critical commentary in moderation for flavor) to shame them into appropriate action; it can be more entertaining, just as cathartic, and much harder to defend against since they can't respond with hysterics either.~\\

And with this announcement and its enormity ("...a word referring to great evil as well as great size..."), I plan to do a lot of it when possible.~\\

--David~\\
And studying CodeWarrior. Did I forget to mention that?~\\
--David~\\

 
		
	
		
\textbf{Leanne M. Fornaca --- \emph{\texttt{13 feb 1998, 10:00}}}~\\

Toby Simpson (t...@cyberlife.co.uk) wrote:~\\
\emph{: Everyone is correct. There will not be a Macintosh version of Creatures 2.}~\\
\emph{Toby,}~\\

As a faithful Mac customer of yours, I'm very sorry to hear that.~\\
May we at least hope for a Rhapsody release?~\\

                ==Leanne~\\

****************************************************************************~\\
Leanne Opaskar                                   kja...@spike.wellesley.edu~\\
ale...@sd.znet.com                               \texttt{http://www.znet.com/\textasciitilde alexia}~\\
"Traveling through cyberspace ain't like dusting crops, boy!"~\\
****************************************************************************~\\

 
		
	
		
\textbf{Ping3506 --- \emph{\texttt{13 feb 1998, 10:00}}}~\\

In article <A41EFAA0620AD111B2B700805F0D182F1BE547@cyberserver>, "Toby Simpson" <t...@cyberlife.co.uk> writes:~\\
\emph{> We do not have an internal Macintosh development department. Finding quality Mac programmers prepared to work in-house is like finding Unicorn horns -- you simply don't find them lying around. The costs involved in producing Mac C2 would be in excess of the costs for producing the Windows version. With this in mind, it obviously makes little commercial sense. The majority of our user base would much rather that we invested extra time and money in ensuring that our primary platform version was the very best it could possibly be.}~\\

Exactly! I can see your next problem Toby....~\\

"What? you mean Creatures 2 won't be out on Commadore 64?"~\\

"I don't belive it.... no Vic 20 version!!"~\\

"This is an outrage... I want Creatures 2 on my 1meg 20k hard drive, steam powered Amiga 600!"~\\

Ping -- get yes more sig space!!~\\
ICQ:  6283750~\\
*******Pingz Nornz*************~\\
\texttt{http://www.crosswinds.net/birmingham/\textasciitilde norndude1/Pingz-Nornz1.htm}~\\
Home of the Multimedia Pack~\\
**********************************~\\

 
		
	
		
\textbf{David ``I Don't Like SPAM'' Wood --- \emph{\texttt{13 feb 1998, 10:00}}}~\\

David "I Don't Like SPAM" Wood wrote:~\\
\emph{> Scott Schilz wrote:}~\\
\emph{> > First they screw us Mac users over for upgrades and add-ons for the first game, and now they're not even gonna give give us a buggy, PC-based port of the second game!}~\\
\emph{> This is, of course, not true; they didn't provide any upgrades and add-ons in the first place. So really, they didn't as much "screw us" as "stiff us."}~\\

Once again, my tongue gets me in trouble. Just goes to show that even I can make mistakes.~\\

Purple Mountain Norns were made available to the Mac. So technically, they did provide an upgrade. Of sorts. A little one, anyway.~\\

Gotta be factual about these things...~\\

--David~\\
Aspiring Carnivore.~\\

 
		
	
		
\textbf{Martha Brummett --- \emph{\texttt{13 feb 1998, 10:00}}}~\\

RudeDog <sgam...@ycsi.net> wrote~\\
\emph{> > Simpson: There are no plans for a Macintosh version of Creatures 2.}~\\
\emph{> Even though I don't use a Mac, I think this sucks big time!!}~\\

My first 8088 computer came with Debug and Basic; when I looked at a MacIntosh, I could not get into the root directory, nor look at a program, etc.~\\

"You're gonna reap just what you sow." -- Lou Reed~\\

Martha Brummett~\\

 
		
	
		
\textbf{David ``I Don't Like SPAM'' Wood --- \emph{\texttt{13 feb 1998, 10:00}}}~\\

Toby Simpson wrote:~\\
\emph{> Everyone is correct. There will not be a Macintosh version of Creatures 2.}~\\

"Fasten your seat belts, kiddies, it's going to be a BUMPY ride..."~\\

(That's the problem with newsgroups: you post something, and other people are going to respond. C'mon, admit it: you like it. >:)~\\

\emph{> For that to be the case, the press, publishers, retailers and developer would need to believe it was a viable product. Sadly, and with great regret, I have to confirm that this is not the case.}~\\

Press, publishers, retailers, and developers? Points which I will haveto address separately.~\\

=====PRESS~\\
When you say "press," I hope you don't mean schitzoid publishing companies like Ziff-Davis, which publish one or two Macintosh-supportive magazines, and then funnel the rest of their copiously loaded pockets into magazines which make most of their money by *bashing* the Macintosh. I've read "news" from them -- opinion/editorial pieces released to CNN as if they were fact -- and their journalism is so yellow it's BROWN. Mac users read John Dvorak's column not because they're hanging on his every word, but to see what outrageous twaddle the flatulating butthead will spew next.~\\

If Apple rolled over and played dead every time some pundit cast an ill fortune for it, they'd be buying disposable caskets on a weekly basis. Yet, it's still used in homes, schools, and businesses. I haven't done the research yet, but I'm sure I can find *press agencies* that Apple has outlived...~\\

=====PUBLISHERS AND RETAILERS~\\
Apple has a number of probles that aren't really their fault. The press is one of them. And the publishers and retailers form a vicious circle defining the other. The use of hybrid disks has fueled a certain amount of anti-Mac prejudice in these fields.~\\

The cycle goes like this:~\\

Retailer sticks 1 copy of hybrid game in Mac section, 9, or 14, or 19 copies of hybrid game in section. Retailer has no way of differentiating copies in the Mac section from those in the Windows section.~\\

Now let's say that the one copy in the Mac section sells, and then someone else comes in looking for the same program -- it's been known to happen. It's not there. So he goes to the PC section and picks one up from there.~\\

Now, the retailer looks at these numbers, and figures the one from the Mac section was a Mac sale, and the ones from the PC section were PC sales. They figure not as many people want the Mac version, tell the publisher to keep the PC versions coming but cut back on the Mac version 'cos they're not selling as well, and when the hybrid disks come in, even fewer (if possible) are put in the Mac section.~\\

Think it couldn't happen? Oh, but it already DID! This is exactly the sort of thing which ultimately convinced the Software Publisher's Association that its method of tracking sales by platform was ultimately full of beans. They've gone from proudly announcing the dwindling sales of Macintosh software to not even hazarding guesses.~\\

But the damage is done; the Mac unjustly earned its reputation as a poor seller, and will continue to do so since I now avoid retail stores like the plague and go for the mail-order catalogs and on-line sites which will correctly record my Mac purchases will be recorded as actual Mac purchases. The retail stores continue to report poor Mac sales because I'm not going in there...~\\

=====DEVELOPER [SIC]~\\
I hope you didn't just ask one developer. The most convenient single developer would be the one you have in-house, and he doesn't do Macintosh.~\\

"Okay, it's been established that you don't know terribly much about development on a Macintosh. In fact, it'd be easier on you if you were to work on Windows and Windows alone. So putting all that out of your mind for a moment, do you think any sort of native Macintosh development could be cost-effective?"~\\

What sort of answer would *you* expect?~\\

There are developers who would argue that Mac development is far from dead. Try telling the Windows developers that, though...~\\

\emph{> We do not have an internal Macintosh development department.}~\\

You'll get no argument from me here. And I'll go you one better: for a small company to create and staff one just for one product would be kind of silly.~\\

\emph{> Finding quality Mac programmers prepared to work in-house is like finding Unicorn horns -- you simply don't find them lying around.}~\\

There are many decent Macintosh developers out there, and I wouldn't be surprised if *none* of them knew to look on your home page at your Help Wanted board because, gosh-darn it, NONE OF THEM ARE PSYCHIC!~\\

But as the saying goes, "A wise man knows everything, a shrewd one, every*body*." I'll start working on the problem over the weekend.~\\

Unless, of course, you change your Help Wanted board to read "Mac programmers can take a flying leap at a rolling donut."~\\

\emph{> The costs involved in producing Mac C2 would be in excess of the costs for producing the Windows version. With this in mind, it obviously makes little commercial sense.}~\\

Bungie!~\\

Marathon, Marathon 2, Marathon Infinity, and most recently, Myth: The Fallen Lords. Once upon a time, Bungie was a very little development and publishing house, much like Cyberlife still seems to be. >:) And this little upstart software company created this little first-person shooter game called Marathon. It was a hit, and spawned several sequels. And their latest achievement, M:TFL, is a multiplatform hit, with a free multiuser online gaming service. They're doing a wonderful job of supporting both platforms, even though they only started out writing software on one platform.~\\

It's no fairy tale, either. It's all true. With this twist: Marathon was originally written for the Mac *only*. Bungie managed to get its toehold in the industry on the platform which you're now trying to orphan.~\\

Then consider companies like Ripcord (Postal), Interplay/Macplay (Descent, Starfleet Academy, Carmageddon), WizardWorks/MacSoft (Duke Nukem, Quake), which are porting popular games over to the Macintosh. Bungie isn't the only one who is publishing Mac ports because either 1) they want to, or 2) it makes more commercial sense than you think it does. Given that they're in business just like you are, I'd cross out (1)~\\

\emph{> The majority of our user base would much rather that we invested extra time and money in ensuring that our primary platform version was the very best it could possibly be.}~\\

I'm reminded of the first message I saw from you on this newsgroup and my reply to it, wherein I reminded you that the majority of your user base is using Windows. We both had a big laugh about it then. Now it doesn't seem all that funny.~\\

99\% is a majority. So is 51\%. Likewise, 1\% is a minority. So is 49\%.~\\

I've corresponded with you before, and you don't seem the sort to alienate customers or leave them terminally unhappy. Whatever you're doing, I acknowledge you're doing for the financial health of the company. And yes, a minority of your user base is running this program on Macintosh.~\\

So pick a fraction between 1\% and 49\% inclusive. What do you think is an acceptable percentage of user base to leave behind?~\\

\emph{> We've constantly been in a no-win situation with regards to the Macintosh. We developed, tested and released a Mac version of Creatures 1 in good faith.}~\\

But a development suite very poorly suited to the task.~\\

\emph{> This product passed the same quality assurance tests that the Windows build did. In the gap between this pass and the USA release (about 8 months), system 8 appeared,}~\\

With ample warnings from Apple. They have a rather large backchannel for developers to use to get technotes and information on new technologies.~\\

At least, for those developers that they know about. You may not have known about it, much the same way the plethora of Mac programmers out there may not have known about your help-wanted notice.~\\

\emph{> and this inserted so many spanners in the cross-development tools we used that I'm surprised it worked at all.}~\\

That makes two of us. I'm glad it worked as well as it did, otherwise I would have regretted buying it.~\\

\emph{> None of this could have either been predicted or avoided short of coding the Mac version separately from scratch.}~\\

Under the circumstances, that would have been the best way to go. It still is, too.~\\

\emph{> Given the nature of the Creatures code, trying to recode around the cross development platforms would be more work than we could justify from a cost/resource point of view.}~\\

If I recall my books on OOP correctly, one of the biggest benefits to writing in object-oriented languages like C++ is that high-level routines (like handling Creatures' genetics) and low-level routines (like opening windows) can be written independent of each other, and furthermore, low-level routines can be encapsulated so that high-level routines can be made to work with them, no matter what operating system the low-level routines are dealing with.~\\

In fact, this is the whole philosophy behind development environments like Microsoft Visual C++ and MetroWerks CodeWarrior, which use a set of "base classes" : derive the high-level routines from and/or base them off of the named low-level routines in the base classes. Then you can change the low-level routines without changing the high-level routines, and voila! You've got a program ready to run on another platform.~\\

The good thing about base class development like this is that it's automatic: change the base classes, and change the platform.~\\

The bad thing about base class development like this is that it's automatic: if a platform changes (like a major OS upgrade) and the base classes for that platform *don't* change, Train Wreck City. This is why I recommended CodeWarrior in the first place: ...~\\

	
		
\textbf{slink --- \emph{\texttt{13 feb 1998, 10:00}}}~\\

On 13 Feb 1998 18:21:46 GMT, "Martha Brummett" <mo...@diac.com> wrote:~\\

\emph{> RudeDog <sgam...@ycsi.net> wrote}~\\
\emph{> > Simpson: There are no plans for a Macintosh version of Creatures 2.}~\\
\emph{> Even though I don't use a Mac, I think this sucks big time!!}~\\

\emph{> My first 8088 computer came with Debug and Basic; when I looked at a MacIntosh, I could not get into the root directory, nor look at a program, etc.}~\\
\emph{> "You're gonna reap just what you sow." -- Lou Reed}~\\
\emph{> Martha Brummett}~\\

The first Mac I saw was a sealed box with a black-and-white screen, a button-challenged mouse and no way to even turn it on without inserting a diskette with someone else's preprogrammed agenda.  It violated Apple's own history of open architecture and power to the user.  It was a royal disappointment.~\\

Sandra -> \texttt{http://www.netins.net/showcase/slink/}~\\
GEEK CODE Version 3.12: GS> AT !d(++)@ s:+ a+ C+++(\$) !U(C/H\$) P(+)@~\\
L E? W++ N++ o? K? w(++)@ O !M V(+) PS+() PE(++)@ Y+ PGP? t++@ 5? X?~\\
R+ tv-- b++(++++)@ DI++++ D G e++++ h+(++)(\$) r+++ x+++~\\

 
		
	
		
\textbf{Scott Schilz --- \emph{\texttt{14 feb 1998, 10:00}}}~\\

David "I Don't Like SPAM" Wood wrote:
...

Ok, sorry if some of my original post sounded a little too harsh. I wrote it right after reading the article, was quite frustrated and dejected about the whole topic of "Cyberlife, Creatures, and Mac support", and didn't bother to stop and think things through (somewhat) rationally.~\\
So maybe "screw us" is the wrong term, I just meant that I've been pretty frustrated overall by Cyberlife's support for Mac as opposed to Wintel.~\\
Lessee, what did Cyberlife sell as add-ons for the PC version? Life Kit \& Genetics Kit (yes, I would've bought these if they made a Mac version). What did they give away? Object Injector, COBs, Christmas Packs, etc. And I'm not even going to count all the third-party add-ons...~\\
And what about stuff for the Mac version? Weeeellll, we've got Purple Mountain Norns and uh, some extra norns I could've converted to Mac format myself. Not even an upgrade to version 1.0. Although I must thank the fine folks such as Sandy and CJK who've given us Mac users extras like COB worlds and support programs.~\\

\emph{> > Don't get me wrong, I'd still buy Creatures 2 in a heartbeat even if it was twice as buggy as the first one.}~\\

\emph{> If it was twice as buggy, would it even run?}~\\

In terms of bugginess, I have to admit that I haven't had a whole lot of problems with this (as opposed to those running on System 8). So basically, I was just saying, "pretty please, make a Mac version no matter what, I'll buy it even if it's got some problems (short of having it crash on every save :-)).~\\

Yes, this is true. I know they've had lots of problems finding a Mac programmer. I just meant that maybe they spent so much time trying to find solutions to people's Mac-oriented problems for the first game that they felt it wasn't worth a C2 port without a dedicated Mac support person. It is a different system, after all.~\\

I wasn't referring at all to Creatures 2 here, I just meant that now that they're busy with C2, they're probably not going to do any more extras for Creatures 1 (PC version included). And since they've just said that there won't be a Mac C2, this means that those of us who love the first game and will continue playing it for years to come will be stuck with the exact same version that we have now, bugs and all.~\\

\emph{> Cyberlife has said that they'll be creating a program to turn Creatures 1 norns into Creatures 2 norns. I'll bet you "five pounds to a cracked piss pot" (one of my favorite lines from Colin Dexter) that they do NOT have a way of turning Creatures 2 norns into Creatures 1 norns...}~\\

\emph{> > Just a little pissed,}~\\

\emph{> So am I, so am I. But rattling off hasty accusations to shame them into appropriate action won't work; they will just as likely take a defensive stance, point to your attitude, and say "You're hysterical; we don't have to listen to you."}~\\

Ok, sorry again if it sounded that inflammatory. I was, after all, upset at the time.~\\

\emph{> No, you have to use carefully reasoned arguments and established facts (with biting sarcasm and critical commentary in moderation for flavor) to shame them into appropriate action; it can be more entertaining, just as cathartic, and much harder to defend against since they can't respond with hysterics either.}~\\
\emph{> And with this announcement and its enormity ("...a word referring to great evil as well as great size..."), I plan to do a lot of it when possible.}~\\
\emph{> --David}~\\
\emph{> And studying CodeWarrior. Did I forget to mention that?}~\\
\emph{> --David}~\\

-- ~\\
Austin: "Hey, there you are!"~\\
Man: "Well, Hi! Do I know you?"~\\
Austin: "No, but that's where you are, you're there!"~\\
-- Austin Powers~\\

Scott Schilz    sasch...@earthlink.net         s...@goplay.com~\\

 
		
	
		
\textbf{Ping3506 --- \emph{\texttt{14 feb 1998, 10:00}}}~\\

In article <34e6afaa.18771...@news.netins.net>, sl...@netins.net (slink) writes:~\\

\emph{> The first Mac I saw was a sealed box with a black-and-white screen, a button-challenged mouse and no way to even turn it on without inserting a diskette with someone else's preprogrammed agenda.  It violated Apple's own history of open architecture and power to theuser.  It was a royal disappointment.}~\\

LOL!!! they had the same ones at Epcot!! one of those 1 button mouse jobs!! I used the one on the far right in Innoventions East if anyone's interested.~\\

Ping -- get yes more sig space!!~\\
ICQ:  6283750~\\
*******Pingz Nornz*************~\\
\emph{http://www.crosswinds.net/birmingham/\textasciitilde norndude1/Pingz-Nornz1.htm}~\\
Home of the Multimedia Pack~\\
**********************************~\\

 
		
	
		
\textbf{Lis 'Potato' Morris --- \emph{\texttt{14 feb 1998, 10:00}}}~\\

Could I suggest then, that you publish a norn importer/exporter for mac? This will mean that the mac people will not be left totally behind when everyone else switches over to creatures 2....~\\




 
		
	
		
\textbf{Toby Simpson --- \emph{\texttt{14 feb 1998, 10:00}}}~\\

David "I Don't Like SPAM" Wood wrote in message <34E4A461....@sickofSPAM.erols.com>...~\\

Toby Simpson wrote:~\\
\emph{> > Everyone is correct. There will not be a Macintosh version of Creatures 2.}~\\
\emph{> "Fasten your seat belts, kiddies, it's going to be a BUMPY ride..." (That's the problem with newsgroups: you post something, and other people are going to respond. C'mon, admit it: you like it. >:)}~\\

My, my, we *are* having fun :-)~\\

\emph{> >  We do not have an internal Macintosh development department.}~\\
\emph{> You'll get no argument from me here. And I'll go you one better: for a small company to create and staff one just for one product would be kind of silly.}~\\

Indeed.~\\

\emph{> There are many decent Macintosh developers out there, and I wouldn't be surprised if *none* of them knew to look on your home page at your Help Wanted board because, gosh-darn it, NONE OF THEM ARE PSYCHIC!}~\\

We don't just wait around expecting people to knock on our door, you know. If we want staff, we do actually look for them.~\\

As a mass consumer machine, the Macintosh doesn't have the market share to justify development for Creatures 2. Personally, I think this is a shame. We went to considerable effort (and cost) to try and address the issues in the existing Creatures 1 Macintosh build.~\\

\emph{>99\% is a majority. So is 51\%. Likewise, 1\% is a minority. So is 49\%.}~\\

It is not like the Macintosh consumer base is 49\%, both you and I know that. When market share is down to less than 5\%, then it is difficult to justify the work unless the resources and demand exist. If retailers are not going to stock the product, and we're not going to sell more than a few thousand, then the only possible way of doing the port is if we had the spare capacity and appropriate developers.~\\

\emph{> I've corresponded with you before, and you don't seem the sort to alienate customers or leave them terminally unhappy. Whatever you're doing, I acknowledge you're doing for the financial health of the company. And yes, a minority of your user base is running this program on Macintosh.}~\\

Indeed.~\\

\emph{> >  None of this could have either been predicted or avoided short of coding the Mac version separately from scratch.}~\\
\emph{> Under the circumstances, that would have been the best way to go. It still is, too.}~\\

And on reflection, had we realised that the development tools we were using would have caused so many problems, perhaps we would have done it differently. In the absense of a time-machine, there is little I can do about that. The decision was made, the work was done, and the product released.~\\

\emph{> If I recall my books on OOP correctly, one of the biggest benefits to writing in object-oriented languages like C++ is that high-level routines (like handling Creatures' genetics) and low-level routines (Like opening windows) can be written independent of each other, and furthermore, low-level routines can be encapsulated so that high-level routines can be made to work with them, no matter what operating system the low-level routines are dealing with.}~\\

May I suggest you buy a book on MFC and have a good bedtime read.~\\

Much further down the line (post Creatures 2), we'll be dramatically reducing our reliance on MFC -- for obvious reasons. Not only does it reopen the door to the potential of using other platforms, but it separates the core systems architecture from a single platform. MFC is particularly suited to writing Windows applications, and sticks its fingers into just about every single area of your program at a reasonably fundemental level. For the generation of real-time 3D, for example, the additional baggage that MFC comes with is undesirable. MFC is designed for large scale Windows Apps, such as Creatures and Creatures 2, and Creatures was designed to work with MFC.~\\

\emph{> off of the named low-level routines in the base classes. Then you can change the low-level routines without changing the high-level routines, and voila! You've got a program ready to run on another platform.}~\\

Honest, we do know this stuff, you really don't have to explain to us how to develop software. We're only in business because we can.~\\

\emph{> Microsoft's Foundation Classes were geared for System 7.5, and as far as I know, they *still* haven't been updated. This is what bit Cyberlife on the ass, and I'm sorry to hear that the welts still haven't gone down yet.}~\\

It is certainly a big issue.~\\

\emph{> Given what I've seen in other program conversions, it IS *possible*, but as I read your posting, the message I'm getting is that it is "not worth attempting."}~\\

Everything is possible. I certainly never said it was *impossible*. But a dose of realism is definitely required. Its not like we made this decision "just like that". Contrary to popular belief, some thought does actually go into these things. Technically speaking, Madonna could hold a concert on the moon. There'd be little point, though, no-one would turn up, it would be prohibitavely expensive, and you wouldn't hear anything because there is no air. So, on reflection, she might plump for a tour of large US cities--~\\
*  The Creatures architecture is heavily MFC.~\\
*  MFC on the Macintosh is not up to the same standard of reliability as the Windows version.~\\
*  Macintosh programmers are hard to find.~\\
*  Less than 5\% of our market is Macintosh.~\\
*  Only a tiny minority of retailers would carry and support the product.~\\

\emph{> But I still think closing and locking the door on the as-yet-unconceived Mac version is a dire mistake.}~\\
\emph{> Especially since a lot of Mac users aren't going to stand for it, present company included.}~\\

Well, that's fine. I'm not going to stand here and humour everyone by saying "yes, we're thinking about it." That would not be fair. Even if the market tripled tomorrow, and we decided "yes, lets go for it.", the Windows build would have a clear 6 month head-start. We'd never be able to release it prior to Christmas, and that is only assuming the developers could be found at a cost that makes it realistic to perform the work. With MFC sprinkled all over the application, such a port requires unique skills. In future, this may not be the case. Right now, it is.~\\

\emph{> >  Macintosh Creatures 1 was entirely reliable on all test platforms we used.}~\\
\emph{> None of them were running System 8, and even if they were, the Microsoft Cross-Development suite wouldn't have helped you anyway.}~\\

No, they were not System 8. We did not have System 8 in August 1996.~\\

\emph{> >  The majority of feedback we have had from Mac users is both constructive and helpful, and has led to documentation of many work-arounds for System 8.}~\\
\emph{> Work-around: temporary bug-fix implemented by the user.}~\\

Work-around: Allowing a user to use a product that they would not otherwise be able to by working with them to find a solution.~\\

\emph{> "trying to address" = "unsuccessfully addressing"}~\\

Correct.~\\

\emph{> I still think this is hasty. And if you want to close an issue, the ABSOLUTE LAST thing you want to do is announce on a NEWSGROUP that it's closed. (I look at this as my invitation to become a buttinski. >:)}~\\

Well, I have to tell you all at some point. Hasty doesn't come into this -- we've done our work, we've looked at the situation, we've worked out what the requirements would be, we know what we would sell of the product, and the conclusion is that we cannot develop Creatures 2 for the Macintosh. I refuse to sit here and tell you that there is hope when I know that the probability we'll do a Mac C2 is negligable.~\\

\emph{> I am sorry your attempt at developing a Mac version of your software turned out so disastrously. However, I am sure there is a solution to this particular seemingly insoluble problem. And even if I can't provide that solution myself, I will make every attempt to find someone who CAN.}~\\

Its not even as though I don't like Macs. We use them in the office for video work, web design and audio processing. We use them because they are the best tools for the job, with the best application support, and consequently is the most productive platform. But as a consumer machine, given the complexity of the problem and market share, we can't create C2 for the Mac.~\\

\emph{> All this knocking about isn't intended as a flame. Same as always, I post to either get information, clarify points, refute points, or change minds. I'm just very caustic when I do it.}~\\

That's OK :-) It is the best way to ensure your points are clear, well thought out and cannot be misinterpreted.~\\

\emph{> The points I'm trying to make here are:}~\\
\emph{> 1. There are decent Mac developers out there.}~\\

Yes, obviously, or there would be no Mac software of any kind. However, we have a unique additional complexity, and finding Mac developers to work *in house* is difficult. Even if they could be found, the budget balancing to ensure that income exceeded development costs would currently (given C2) be impossible.~\\

\emph{> 2. Cyberlife doesn't necessarily have to employ them full-time in their own in-house shop to get full use from them; there is nothing wrong with outsourcing large jobs to *competent* people if you're leery of spending the time or money to do it yourself.}~\\

Clarification required. I would only perform the work with in-house developers. Creatures is a "thing" and to get the best results with such work it is necessary to work *with* the rest of the team. I've had mixed experiences from outsourcing such important work, especially where the confidentiality of your technology is to be considered.~\\

\emph{> 3. The experience you had developing Creatures Mac "1" (0.92, really) has been far from an average case.}~\\

Well, the 0.92 is only because we failed to bump the version number for the release CD-ROM, not because we didn't finish it.~\\

\emph{> 4. Developing quality software for the Mac can be profitable.}~\\

Yes, it can, but it greatly depends on the software in question.~\\

Toby~\\

 
		
	
		
\textbf{Ping3506 --- \emph{\texttt{14 feb 1998, 10:00}}}~\\

In article <6c47td\$p0...@chlorine.compulink.co.uk>, "Toby Simpson" <t...@lobster.cix.co.uk> writes:~\\
\emph{> May I suggest you buy a book on MFC and have a good bedtime read.}~\\

LOL!!!!!!~\\

"give me Creatures on a Vic 20 or I'll stage a protest"~\\

<next morning outside Cyberlife's offices, a pimply geek in an anorack in walking around with a sign, that reads: "Me get no Creatures 2">~\\

Ping sleep no Creatures 2!! Ping get no Creatures 2, 8 months!~\\
ICQ:  6283750~\\
*******Pingz Nornz*************~\\
\texttt{http://www.crosswinds.net/birmingham/\textasciitilde norndude1/Index.HTM}~\\
Home of Ping's Things~\\
**********************************~\\

 
		
	
		
\textbf{Lummox JR --- \emph{\texttt{14 feb 1998, 10:00}}}~\\

Lis 'Potato' Morris wrote:~\\
\emph{> Could I suggest then, that you publish a norn importer/exporter for mac? This will mean that the mac people will not be left totally behind when everyone else switches over to creatures 2....}~\\

I must admit, Lis, I don't understand what you're talking about. Creatures 2 Norns won't be backwards-compatible, so how could anyone translate a Creatures 2 Norn for the PC into a Creatures 1 Norn for the Mac?~\\

I also don't think *everyone* will switch to Creatures 2. I wouldn't be surprised if a lot of users and even webmasters stick with the original game. In fact, many (like me, if possible) may opt to install both. At this point I believe my site, the Norn Underground, will be poised to provide support for both games by the time Creatures 2 is released.~\\

The Norn Underground~\\
\texttt{http://www.dreamscape.com/lummoxjr/creatures}~\\
Lummox JR~\\

 
		
	
		
\textbf{Wildo --- \emph{\texttt{14 feb 1998, 10:00}}}~\\

Lummox JR wrote in message <34E5F78F.5...@aol.com>...~\\
<snip>~\\
\emph{> I also don't think *everyone* will switch to Creatures 2. I wouldn't be surprised if a lot of users and even webmasters stick with the original game. In fact, many (like me, if possible) may opt to install both. At this point I believe my site, the Norn Underground, will be poised to provide support for both games by the time Creatures 2 is released.}~\\

\emph{> The Norn Underground}
\emph{> \texttt{http://www.dreamscape.com/lummoxjr/creatures}}~\\
\emph{> Lummox JR}~\\

Same here I think ill keep both on my system.~\\

 
		
	
		
\textbf{Lis 'Potato' Morris --- \emph{\texttt{15 feb 1998, 10:00}}}~\\

Lummox JR wrote in message <34E5F78F.5...@aol.com>...~\\
\emph{> Lis 'Potato' Morris wrote:}~\\
\emph{> >  Could I suggest then, that you publish a norn importer/exporter for mac? This will mean that the mac people will not be left totally behind when everyone else switches over to creatures 2....}~\\

\emph{> I must admit, Lis, I don't understand what you're talking about. Creatures 2 Norns won't be backwards-compatible, so how could anyone translate a Creatures 2 Norn for the PC into a Creatures 1 Norn for the Mac?}~\\

<rant>~\\
It's the lack of backwards compatability that really irritates me... obviously converting a norn to version 1 may result in lose of phenotype, but it should be possible....~\\
The lack of backwards compatbility in many programs makes them far less easy to use in an environment (like most offices) where different pc's have different versions. What if someone has , say a lotus 123 v.4 file, and wants to use it on the only pc avaible with the particular software they need to export it to? Said pc only takes lotus 123 v3 files. Tough. It wastes huge amounts of time, and effort....and that is a real example from when I worked at zeneca as an underpaid skivvy (read biomathematics research associate). The only thing I can think of that is fully backwards compatible is html... if you use a low version html browser, it just says, 'I don;t understand that bit- I'll ignore it!'... that's the way they should work....</rant>~\\

\emph{> I also don't think *everyone* will switch to Creatures 2. I wouldn't be surprised if a lot of users and even webmasters stick with the original game. In fact, many (like me, if possible) may opt to install both. At this point I believe my site, the Norn Underground, will be poised to provide support for both games by the time Creatures 2 is released.}~\\

\emph{> The Norn Underground}~\\
\emph{> \texttt{http://www.dreamscape.com/lummoxjr/creatures}}~\\
\emph{> Lummox JR}~\\

That is great!! And it is what I wanna see <applauds Lummox>~\\

 
		
	
		
\textbf{Ping3506 --- \emph{\texttt{15 feb 1998, 10:00}}}~\\

In article <34E5F78F.5...@aol.com>, Lummox JR <Lummo...@aol.com> writes:~\\
\emph{> I also don't think *everyone* will switch to Creatures 2. I wouldn't be surprised if a lot of users and even webmasters stick with the original game. In fact, many (like me, if possible) may opt to install both. At this point I believe my site, the Norn Underground, will be poised to provide support for both games by the time Creatures 2 is released.}~\\

I also will keep Creatures 1 installed, and may use it as a testing ground, and maybe a Vacation Spot, I will also try and support Creatures 1-2 on my page!~\\

-- Ping... going Creatures tWo~\\
Ping-- going Creatures tWo~\\
ICQ:  6283750~\\
*******Pingz Nornz*************~\\
\texttt{http://www.crosswinds.net/birmingham/\textasciitilde norndude1/Index.HTM}~\\
Home of Ping's Things~\\
**********************************~\\

 
		
	
		
\textbf{rajamaki --- \emph{\texttt{15 feb 1998, 10:00}}}~\\

Lummox JR <Lummo...@aol.com> wrote:~\\
<snipped>~\\
\emph{> I also don't think *everyone* will switch to Creatures 2. I wouldn't be surprised if a lot of users and even webmasters stick with the original game. In fact, many (like me, if possible) may opt to install both. At this point I believe my site, the Norn Underground, will be poised to provide support for both games by the time Creatures 2 is released.}~\\

\emph{> The Norn Underground}~\\
\emph{> \texttt{http://www.dreamscape.com/lummoxjr/creatures}}~\\
\emph{> Lummox JR}~\\

I'm with Lummox on this one. Even if I won't get Creatures 2 for my Mac, I'll still keep MacNorns up and continue posting new exsiting transplanted Albias for everyone :-) I'll try to evaluate different PC emulators to provide alternative options for Mac users on how they *can* play Creatures 2.~\\

Though when I first saw Toby's interview I was completely depressed and felt like breaking a few innocent crystal animals...  I still want Cyberlife to update that Mac version to 1.02 like they said 4 months ago and I want an object injector. So far Cyberlife has ignored my mails (hint, hint, hint)~\\

/Sandy~\\
----------~\\
MacNorns -- \texttt{htpp://www.crosswinds.net/brussels/\textasciitilde rajamaki}~\\
ICQ \# 7029961~\\

-- -- -- -- -- == Posted via Deja News, The Leader in Internet Discussion == -- -- -- -- --~\\ 
\texttt{http://www.dejanews.com/}   Now offering spam-free web-based newsreading~\\

 
		
	
		
\textbf{Sparrow --- \emph{\texttt{15 feb 1998, 10:00}}}~\\

\emph{> In fact, many (like me, if possible) may opt to install both.}~\\

I opt to install both! Unfortunately, I only have a mere 30 MB left, starting at a command prompt, Windows unknown to RAM, the "exit" command completely irrelevent. When I was using my old 40 MB hard drive (this is TOTAL space), I always wondered what I'd do with 586 MB. Answer: I installed Windows 95...~\\

Good heavens! I can see myself as an old man, telling my grandchildren to stop complaining about lack of disk space...~\\

"Back when I was your age, we had to use those puny three gigabyte hard drives uphill both ways in snow six feet deep..."~\\ 

	
	  	Messages 26 -- 50 sur 112 -- Tout r{\'e}duire  --  Traduire tous les contenus en Fran\c{c}ais 	< Plus anciens  Plus r{\'e}cents >
	
		
\textbf{David ``I Don't Like SPAM'' Wood --- \emph{\texttt{16 feb 1998, 10:00}}}~\\

Ping3506 wrote:~\\
\emph{> "give me Creatures on a Vic 20 or I'll stage a protest"}~\\

Laugh as hard as you possibly can, Ping... everything becomes obsolete sooner or later. And when Creatures 3 won't run on Windows 95 (a klunky mix of 16- and 32-bit code if ever there was one), I hope you can maintain that <SARCASM>*adorable</SARCASM> sense of humor of yours.~\\

--David~\\
Aspiring Carnivore... (mmmmmmm. leg. Maybe mesquite on the other one.)~\\

 
		
	
		
\textbf{Ping3506 --- \emph{\texttt{16 feb 1998, 10:00}}}~\\

In article <34E849EA.7...@sickofSPAM.erols.com>, "David ``I Don't Like SPAM'' Wood" <pyxis...@sickofSPAM.erols.com> writes:~\\
\emph{> Laugh as hard as you possibly can, Ping... everything becomes obsolete sooner or later. And when Creatures 3 won't run on Windows 95 (a klunky mix of 16- and 32-bit code if ever there was one), I hope you can maintain that <SARCASM>*adorable</SARCASM> sense of humor of yours.}~\\

LOL!!~\\

Ping~\\
ICQ:  6283750~\\
*******Pingz Nornz*************~\\
\texttt{http://www.crosswinds.net/birmingham/\textasciitilde norndude1/Index.HTM}~\\
**********************************~\\

 
		
	
		
\textbf{Sparrow --- \emph{\texttt{16 feb 1998, 10:00}}}~\\

\emph{> a klunky mix of 16- and 32-bit code if ever there was one}~\\

Hey! I'm a proud user of that klunky mix of 16 and 32 bit code!~\\

I prefer DOS better. You know there's a guy out there who says he sees no future for DOS-based games? For some reason he reminds me of an antichrist. That wasn't to hint around at an insult or anything, it really is the first thing I thought of :)~\\
		
	
		
\textbf{David ``I Don't Like SPAM'' Wood --- \emph{\texttt{17 feb 1998, 10:00}}}~\\

Sparrow wrote:~\\
\emph{> > a klunky mix of 16- and 32-bit code if ever there was one}~\\
\emph{> Hey! I'm a proud user of that klunky mix of 16 and 32 bit code!}~\\

Which is why you're mentioned prominently in the nightly prayers of some people. (Mind you, I'm not saying which side they're praying FOR...)~\\

\emph{> I prefer DOS better.}~\\

Yeah, I remember the old days of computing too. I was using a TRS-80 model 1 at the time, and I had to write my own dialer for it.~\\

I also remember (in the same time frame) using a shell account on a DEC mainframe running VMS 3.7, and the accompanying rush of power I got when I made available a utility for removing control-Gs from peoples' incoming mail files. (Mailbombs were a problem at the time.)~\\

And then I remember (also the same time frame) the first time I used a Mac. It was a demo, sitting in the computer department's office. It didn't have much software, it was just black-and-white (there were some fancy VT-220s that did full color text downstairs in the main lab), but it did have a paint program which gave me control down to the pixel. I didn't have much use for it at the time, since there was nothing I could put that picture on...~\\

...but it only just NOW occurred to me that I was doing something on that little proprietary box that no other machine in the building, nor probably the entire campus at the time, could match.~\\

This is all tangential to the topic (we had a topic?)...~\\

\emph{> You know there's a guy out there who says he sees no future for DOS-based games?}~\\

...that there were people back then who were saying "No, the Macintosh can't possibly last, who wants to use a box with silly little pictures on it?" at the same time there were other people saying "Apple is onto something here; that whole windows and icons and mice (oh my!) thing could catch on."~\\

Then when Microsoft tried to introduce Windows (versions 1 and 2 were butt-ugly), the "box with silly little pictures on it" camp fragmented into "Microsoft is making a mistake here; if nobody wants to mess with Apple's silly little pictures, who will want to mess with theirs?" and "Microsoft's onto something here; that whole windows and icons and mice (oh my!) thing could catch on." (Note the verbatim copying of the Camp 2 statement above and make of it what you will.)~\\

The "Apple is onto something here" people didn't fragment; they just looked at that second Windows camp as if they had spiny lobsters crawling all over their faces. And their computers.~\\

Camp 1a chose to stay with DOS. Camp 1b chose to go with Windows. Camp 2 stuck with Apple. You'd better start keeping score now; it gets complicated real soon.~\\

Camp 1b fragmented into fans of Windows 3.11 (1b1) and Windows 95 (really Windows 4) (1b2). Camp 2 is still scratching their heads.~\\

Now, Microsoft is planning to introduce Windows 97. Err, 98. Or have they gone to 99 yet? No, they're still on 98. And that's rolling forward quickly. They might even make this one. >:) That's right, it's NT5 which is in all probability delayed to 99.~\\

Anyway, camp 1b2 will probably be split into camps 1b2a (Windows 95 is better) and camp 1b2b (Windows 97+x is better).~\\

And oh, that's right, some people may have moved over to NT for their personal systems and servers, so we need a camp 1c in here.~\\

So we have camps 1a (DOS), 1b1 (Windows 3.1), 1b2a (Windows 95), 1b2b (Windows int(98.5+rnd(0))), 1c (Windows NT), ALL of whom disagree with each other about the better computing paradigm.~\\

But they all agree that Apple has no future.~\\

Meanwhile, Apple Computer Inc. is *still* bigger than Microsoft.~\\

And Camp 2 is still scratching their heads. Not just at Windows itself, but at its user base.~\\

\emph{> For some reason he reminds me of an antichrist. That wasn't to hint around at an insult or anything, it really is the first thing I thought of :)}~\\

And what I thought of, well...~\\

 
		
	
		
\textbf{Cyn Xphile --- \emph{\texttt{18 feb 1998, 10:00}}}~\\

Dear Toby and Cyberlife,~\\
Sorry to hear that you're leaving the Mac market.  It is the ultimate revenge/satisfaction that Mac's are used by so many to make IBM graphics.  And newspapers, and so on and so forth.  The people in the arts keep Apple alive. You'd like to lose us?  Cut us lose?  Couldn't realize there are people like me that will not update to System 8 for at least another year or more and that your company couldn't assure bug fixes for System 7.5?  There must be a reason you embrace your office Macs.~\\

I'd like to put another case in Mac technology dominance for game creation-- Myst.  Hypercard shell and quickdraw.  Look at SimCity, SimTower.  Maxis have a lack of programmers?  Their programs crash? Other's have made the argument for adapting other popular games better than I.  I am at best a knowledgable tinkerer.  All I can imagine while reading the discussion is if your strapped for cash sell the idea to a compentent Mac company.  Oh, but you want them inhouse to safeguard your innovative code.~\\

I'd hate to say that the internet was your downfall.  I trusted the Mindscape label (because I believe they produced a wonderful game for the old 512K Macs that kept me stumped for weeks, but I don't remember the name of it) and feel that if a non-internet user buys your game and treats it gently, will work around the crashes and enjoy it.  A internet user will encounter a crash and leap to the newsgroups and discover that software support is not there at any company on the box, and this person will encounter wonderful posters that will solve the problem.  This makes all companies involved look odd.  Bad internet press is hard to overcome.~\\ 

The idea that my 3 year old computer looks like a Commodore to some makes me laugh.  Being on the cutting edge of computers costs too much.  Depreciation of a computer's 'value' is as bad as the car market!  AOL gets flooded with bug complaints each time they update, but they do it.  Give them a call about why they keep doing it.  Ask them for percentages on Apple VS. IBM machines in the states.  Do they feel they can't lose Apple users?  It takes awhile, but they deliver.  Apple users have always have to wait for conversions.  The past years could be the lost 'golden age' for simultanous releases of software.  We must brace ourselves to wait again while sofware makers count request letters.~\\
Cyn C.~\\
Chicago~\\
\texttt{http://members.aol.com/cyn6x9els}~\\
seeking HTML jobs~\\
Cynthia Clavey~\\
\texttt{http://members.aol.com/cyn6x9els/}~\\

 
		
	
		
\textbf{Sonic28940 --- \emph{\texttt{18 feb 1998, 10:00}}}~\\

Speaking of SimCity 2000 and Myst... I have a mac owning friend who tried both on my P.C and immediately went out and bought them for macintosh. He also came over and saw "The Even More Incredible Machine" and bought it. Later, when "The Incredible Machine 2" came out, he went out and bought it for mac again. He came over and saw "Lode Runner: The Legend Returns" and ran out and bought it for macintosh.~\\

Cyberlife, I have no problem with you not publishing it for mac. I could care less (a lot less). I just wanted to point out a personal example of a macintosh gamer buying games which were released for both P.C and macintosh.~\\

There aren't that many macintosh games to start with, so if a company were to release 2 or 3 great games for Mac (like Sierra did for P.C back in 1990) they could almost corner the entire macintosh gaming market... All 5\% of it! <g>~\\

SimCity 2000 has sold... How many copies? Probably a little over 1,000,000. I know that Myst has sold 1 million coppies. The Incredible Machine? With its appealing box and the fact that it was released by the then-reputable "Sierra", I'd guess around 500,000... Enough to make 5 more additions to the "incredible" series, anyway. How about Lode Runner? Sadly, I don't think that ever became that popular.~\\

-Sonic~\\

===============================================~\\
Visit Piecemeal's Adequate Creatures Page!~\\
\texttt{http://www.geocities.com/SunsetStrip/Venue/1492/creatures.html}~\\
===============================================~\\

 
		
	
		
\textbf{BEAR --- \emph{\texttt{18 feb 1998, 10:00}}}~\\

Cyn Xphile wrote in message <19980218062201.BAA24...@ladder03.news.aol.com>...~\\
\emph{> Dear Toby and Cyberlife,}~\\
\emph{> Sorry to hear that you're leaving the Mac market.  It is the ultimate revenge/satisfaction that Mac's are used by so many to make IBM graphics.}~\\
\emph{> And newspapers, and so on and so forth.  The people in the arts keep Apple}~\\

alive.~\\
<snippity snip snip>~\\

Enter -- Bear's 2 Cents...
     You said a mouthfull right there!  Apple has become a nich market, speciallizing in "artful areas."  Even though there are many people out there who also use these computers for home use, the Apple is currently a work machine...~\\
   For better or worse, you will notice that Cyberlife also neglected to create version of Creatures 2 for people who run Silcone Graphics, Alpha, or Sun Systems machines -- all of which would require a special version to have the program run properly.  (yes I do know people who use some of these machines)  What about those people???~\\
    Simple, THERE IS NOT ENOUGH MARKET TO JUSTIFY THE PROGRAMMING OF A VERSION FOR THEM, just like Apple users.~\\
     Sorry, its just a fact of economics and life in general....~\\

not tryin to flame -- just pointing to the obvious...~\\

Enjoy.~\\
Bear~\\
     : )>~\\

 
		
	
		
\textbf{Fred Haineux --- \emph{\texttt{18 feb 1998, 10:00}}}~\\

Sparrow <terreoBOI...@geocities.com> wrote:~\\
\emph{> > a klunky mix of 16- and 32-bit code if ever there was one}~\\
\emph{> Hey! I'm a proud user of that klunky mix of 16 and 32 bit code!}~\\

Well, we won't tell anyone if you don't.~\\

\emph{> I prefer DOS better. You know there's a guy out there who says he sees no future for DOS-based games? For some reason he reminds me of an antichrist. That wasn't to hint around at an insult or anything, it really is the first thing I thought of :)}~\\

What Toby and the rest have apparently missed is that 100\% of the Intel/Windows PCs run DOS games, and only 25\% of them run Windows 95 games.~\\

Of course, they spent their MAC profits porting Creatures to Windows NT, which has LESS THAN 1\% MARKETSHARE.~\\

bc~\\

 
		
	
		
\textbf{Fred Haineux --- \emph{\texttt{18 feb 1998, 10:00}}}~\\

dw...@I.skipjack.don't.bluecrab.like.org.SPAM! wrote:~\\
\emph{> Purple Mountain Norns were made available to the Mac. So technically, they did provide an upgrade. Of sorts. A little one, anyway.}~\\

\emph{> Gotta be factual about these things...}~\\

Yep. 'Strue. They did release something that took me about 4 minutes to create from the PC disks.~\\

Then they released "Mac versions" of their manuals and readmes. These, in addition to being the exact same things in the printed manuals, were lacking any of the pictures. As you know, those pictures allow you to tell the difference between good and bad plants. Well, nice try eh?~\\

But wait. First they put these documents in a format that nobody's ever heard of. They told Mac users that they could decode them with Stuffit. Untrue. Stuffit can undo about a hundred different file types, except this one.~\\

Eventually, they put these documents into Stuffit's favorite formats. Somehow, the documents they put up are not "plain text" or "RTF" formats as they are labelled. They are Windows Word format.~\\

I used an automatic Windows format converter program, called Mac Link Plus, to get and decode the original Windows documents. It took me about 1 minute.~\\

So in the course of TWO "upgrades" for Macs, they have done work that I can do in FIVE MINUTES.~\\

And they took their Mac profits and ported to Windows NT.~\\

bc~\\

 
		
	
		
\textbf{Indigo --- \emph{\texttt{18 feb 1998, 10:00}}}~\\

Well said, Bear. =)~\\

You people also forget (in comparing CyberLife to Maxis) that Maxis is a rather large company that has been in business for quite awhile and have many, many sim games out on the market. Toby has said it before -- it isn't easy to find an in-house Mac programmer. CyberLife is much smaller than Maxis and thus should not be compared to them. If you Mac-ers really are interested in having a Mac version of Creatures, go ahead and learn the programming skills you need to and go apply for a job at CyberLife. =)~\\
That's the best advice I can give you besides actually buying a PC.~\\

\emph{> For better or worse, you will notice that Cyberlife also neglected to create version of Creatures 2 for people who run Silcone Graphics, Alpha, or Sun Systems machines -- all of which would require a special version to have the program run properly.  (yes I do know people who use some of these machines)  What about those people???}~\\

LOL -- mainly because those types of computers are specifically for designing things and NOT playing games on them. Oh, sure, there are games made for them, but when you look at the whole scheme of things, you use certain computers for certain professions. Gamers, and I say this very loosely, are more likely to have PC's running Windows or DOS rather than SGI's, Alpha's, or Suns running their respective OS's.~\\

\emph{> Simple, THERE IS NOT ENOUGH MARKET TO JUSTIFY THE PROGRAMMING OF A VERSION FOR THEM, just like Apple users.}~\\
\emph{> Sorry, its just a fact of economics and life in general....}~\\

I agree. Besides, no one's telling you to stop playing Creatures. Toby said that there'd still be support for Creatures 1 -- don't worry. And if you're REALLY desperate for Creatures 2 (and have a few extra thousand bucks) go out and by a PC. I suppose it'll eventually pay of considering the amount of games you'll now be able to play. =)~\\

Indigo -- who doesn't HATE Macs, only hates using them. =)~\\

-- ~\\
"Breed, not bread!! Eat wheat, not norns!! Love 'em not loave 'em!!"~\\
The Rights for Norns Society -- We keep your norns from turning into loaves and other products. We are also proud to be allies of the NORN.~\\
Join Today! ;)~\\
\texttt{http://www.sgoinc.com/Indigo/creatures}~\\
Indigo's UIN: \#1530587~\\
Remove BIBBLE from email address to send mail.~\\

----BEGIN\_C-ADD\_CODE\_BLOCK---~\\
Version Number: 1.2~\\
APA/ST d+ s: a---> a?,23/04/80 f- t+++:++ C+~\\
E4++,5++,7+,9++ Q+ W4++ P+ Gd-:Daggerfall/s-:SimCity,~\\
SimTower/o-:Star Wars Monopoly,Streets of SimCity,Nascar 2~\\
-----END\_C-ADD\_CODE\_BLOCK-----~\\

 
		
	
		
\textbf{Fred Haineux --- \emph{\texttt{18 feb 1998, 10:00}}}~\\

sl...@netins.net (slink) wrote:~\\
\emph{> The first Mac I saw was a sealed box with a black-and-white screen, a button-challenged mouse and no way to even turn it on without inserting a diskette with someone else's preprogrammed agenda.  It violated Apple's own history of open architecture and power to the user.  It was a royal disappointment.}~\\

Hope I don't upset you, Slink, but your claims are not backed up by the facts:~\\

1) Four-button mice have been available for the Mac for almost ten years. Apple doesn't sell them, Kensington does. Kensington also sells two button mice, trackballs, and one-button mice. Many of these mice are mechanically identical to PC mice. Because Mac and PC have different serial buses, they have different electrical chips inside, but Mac multi-button mice are at least as useful, configurable, and hackable as PC mice. Heck, I can configure my mouse button to hit "save" and "print" if I wanna.~\\

==========~\\
"Button challenged" is a poor insult to apply, since PCs didn't "come with" mice at ALL for quite some time after Macs did, and there still is no "standard" mouse for Windows PCs. Heaven forbid that Microsoft introduce a new version of their mouse, and build in special support for it in Windows 98/9, like they did with keyboards for Win95!~\\
==========~\\

2) Macs, from Day One, came with 8 typefaces, bold, italic, and a variety of sizes. 6 or 7 of the faces had letters with different widths, a typographic innovation of the Middle Ages.~\\

==========~\\
"Black and White" is a poor insult, because until Windows came out, PCs came with screens full of ugly looking, mono-spaced letters. There was almost NO way to make the type bigger or smaller, or change typefaces, or use proportional type. Even such 'nicenesses' as bold and underline were OPTIONAL. Indeed, original PCs were GREEN AND BLACK. By comparison, black and white was GORGEOUS.~\\
==========~\\

3) Original Macs didn't boot without an operating system floppy in the slot. Neither did original PCs. Macs booted off of hard disk as early as 1986, roughly the same time as PCs, however, Macs at that time used SCSI, a kind of hard disk bus that allowed 6 drives to be connected, not two, and ALL SCSI hard drives worked with ALL Macs, unlike PCs which had a wide variety of hard disk cards, most of which didn't play nicely with each other.~\\

==========~\\
Your criticism is poorly placed, because Macs have up until very recently had superior hard-disk support. PCs have finally become equal to Macs, but they cannot exceed Macs.~\\
==========~\\

4) Original Macs didn't have much expansion, whereas original PCs did. You are absolutely correct.~\\

The Mac II, however, which premiered in 1987, had 6 slots, two floppy drives, and the ability to add up to 6 hard disk drives without adding cards.~\\

It also allowed the user to connect and use up to 6 color monitors at once. Windows could cross screen boundaries, and even go from a black-and-white screen onto a color screen. Each half of the window would be drawn correctly.~\\

No other computer had ever done that before. Not even \$100,000 Sun/SGI/HP/DEC workstations.~\\

These days, you can make a "desktop" of up to 7.5 million color pixels at once. (You CAN theoretically have a bigger desktop than that, even!)~\\

Last time I checked, Windows doesn't support multiple graphics monitors.~\\
It's been 12 years since Mac has.~\\

Even Creatures on a Mac can run in a window spanning several monitors.~\\
I've run it spanning 3 SVGA monitors at once, personally.~\\

==========~\\
Calling Macs a "sealed box" was a fair assessment, for the first 18 months of the product history. For the other 150 months, however, it's not accurate, and it's a poor insult because Macs still have expansion capabilities that Windows PCs don't.~\\
==========~\\

4) Although the original Macintosh didn't have a low-level command system that made hackers happy, it did boot directly into a friendly, powerful, GRAPHICAL user environment that made non-hackers happy. That was the IDEA behind Mac. I can understand that it made some hackers unhappy, but they got what they wanted soon after, as I explain below.~\\

You are absolutely correct that PCs did often come with a programming language -- the oh-so-powerful BASIC -- which Macintosh did not. Now the funny thing is that the reason Mac didn't come with BASIC is because Microsoft demanded it. Honest injun! MacBASIC was an Apple product set to intro shortly after the Mac did, but Bill Gates himself demanded that Apple pull the plug, allowing Microsoft to sell MS-BASIC to users instead. Faced with losing Word and MultiPlan (predecessor to Excel), Apple gave in. (Surprisingly enough, MS-BASIC for Mac in 1984 was just about as good as Visual BASIC for Windows, which didn't come out until, what, 1994? Mac first, even from Microsoft.)~\\

You are also absolutely correct that the original Macintosh came without detailed programming documentation. The Apple II, II+, and IIe came with the famous "Red Book" containing software source code, actual schematics of the computer, and hardware project diagrams that let any hacker with a soldering iron build add-ons and doo-dads for their Apple II. (I built a slide-projector controller, so that people learning foreign languages could see actual photographs of objects for their language drills. )~\\

It wasn't until 1985, almost a half a year after intro, that Inside Macintosh was made available to whoever wanted it, "at cost" of the xeroxed pages.~\\

(The cost was an outrageous \$100 -- something that I got into a big argument with Jerry Pournelle about. He contended that Apple was overcharging -- I called Gnomon Copy in Harvard Square to find out how much it would actually cost to xerox the damn 500 page double-sided monster -- about \$100. This was especially entertaining because Pournelle is a famous, professional, science fiction WRITER, so he knew darn well what xeroxing gigantic manuscripts cost. Of course, he never admitted he was wrong, and the Grand Macintosh Tradition of loudly bemoaning how "Apple Is Ripping Me Off" was born.)~\\

Within two years, a cheap "phonebook" edition of over 1100 pages of documentation was made available for about \$25, at a LOSS from Apple.~\\

And these days, anyone with some patience can download almost all of Apple's Mac programming documentation FOR FREE (<\texttt{http://devworld.apple.com/}>). And there has never been ANY fee or signature requirement to shipping a Mac product. None at all.~\\

==========~\\
Maybe your original criticism of Macintosh being hacker-unfriendly was accurate, for a short period of time, but I don't think it was valid as of 1986. It certainly isn't valid today.~\\

Many "big name" software companies started out as "garage" operations of hackers, on the Mac. This includes ADOBE, MACROMEDIA, and NETSCAPE. These companies wouldn't EXIST if Apple hadn't been "hacker-friendly." All three, by the way, are bigger in marketshare than Microsoft.~\\

Your criticism is particularly poorly placed, considering Microsoft's business tactics. MS has tried to push each of these companies out of business, and used its considerable advantage of "inside information" about the Windows operating system (something the US Government has deemed illegal) to try. Even when cheating, they do not succeed.~\\

Indeed, one reason that there isn't a "good" Creatures Macintosh version is because Microsoft cheated its MFC developers out of Mac support. They promised, they lied, just as they had before, and will again.~\\
==========~\\

The bottom line is that Apple won the war, but lost a whole lot of blood.~\\

I say Apple "won" because you'll have a hard time buying any new computer without "cutesy" icons, pictures, and graphics. (This is something Bill Gates as recently as 1994 was saying "was totally unnecessary" and that he "hated." "DOS is all anyone ever needs." Of course, in 1995, he was saying that "DOS is dead." As usual, he was lying both times.)~\\

On the other hand, through a combination of guts, treachery, lies, and outright theft, Microsoft ended up with the bulk of the money. Apple has been "bled" of money, and Microsoft has moved on to using the proceeds of bleeding Apple to bleed Netscape.~\\

This, however, does not mean that all Apple Macs are obsolete. Au contraire. Every PowerPC chip EVER has provided superior performance while being cheaper, smaller, cooler, and less power-consuming than the equivalent Pentium-or-equivalent. Mac hardware is cool, and will continue to be at least as cool as Intel stuff. Period.~\\

Mac OS still provides features that PC users simply can't get, at any price. And although there are MORE games available for the PC, there are still GOOD games in each game category (except perhaps the "Creatures" category) for Mac.~\\

People have been predicting that Apple would go out of business for something like 18 years. They have been wrong every single time so far. Even with Bill Gates's billions of dollars, and many companies and schools irrationally deciding "No MACS, no matter what!!!", Apple still hasn't gone out of business.~\\

People who are annoyed by Apple now should get psychological help, because Apple will be annoying them for a LONG TIME TO COME.~\\

If you are foolish enough to compare Macs to Amigas, Commodore Pets, TRS-80 computers, or even Apple II's, bear in mind that Apple sells more Macs in a month than all those models of computers sold in their entire history.~\\

Apple sells more computers in a month than all the non-PC computers (workstations, mainframes) sold in the world, combined, sell in a YEAR.~\\

There is no way Apple is going out of business. Assuming, of course, that they EVER get their management act in order. They've got cool hardware, cool software, and significant quantities of people buying and using Macs.~\\

Now, to be fair:~\\
1) I am not claiming that Slink is "Anti-Mac." I know she's neutral about Macs, because I asked her. I am just correcting a few misconceptions that she has, and that are fairly common among the DOS/Windows/PC crowd.~\\
2) I do not think ...~\\

 
		
	
		
\textbf{Cyn Xphile --- \emph{\texttt{19 feb 1998, 10:00}}}~\\

Dear Toby and Cyberlife,~\\
Sorry to hear that you're leaving the Mac market.  It is the ultimate revenge/satisfaction that Mac's are used by so many to make IBM graphics.  And newspapers, and so on and so forth.  The people in the arts keep Apple alive. You'd like to lose us?  Cut us lose?  Couldn't realize there are people like me that will not update to System 8 for at least another year or more and that your company couldn't assure bug fixes for System 7.5?  There must be a reason you embrace your office Macs.~\\

I'd like to put another case in Mac technology dominance for game creation -- Myst.  Hypercard shell and quickdraw.  Look at SimCity, SimTower.  Maxis have a lack of programmers?  Their programs crash? Other's have made the argument for adapting other popular games better than I.  I am at best a knowledgable tinkerer.  All I can imagine while reading the discussion is if your strapped for cash sell the idea to a compentent Mac company.  Oh, but you want them inhouse to safeguard your innovative code.~\\

I'd hate to say that the internet was your downfall.  I trusted the Mindscape label (because I believe they produced a wonderful game for the old 512K Macs that kept me stumped for weeks, but I don't remember the name of it) and feel that if a non-internet user buys your game and treats it gently, will work around the crashes and enjoy it.  A internet user will encounter a crash and leap to the newsgroups and discover that software support is not there at any company on the box, and this person will encounter wonderful posters that will solve the problem.  This makes all companies involved look odd.  Bad internet press is hard to overcome.~\\

The idea that my 3 year old computer looks like a Commodore to some makes me laugh.  Being on the cutting edge of computers costs too much.  Depreciation of a computer's 'value' is as bad as the car market!  AOL gets flooded with bug complaints each time they update, but they do it.  Give them a call about why they keep doing it.  Ask them for percentages on Apple VS. IBM machines in the states.  Do they feel they can't lose Apple users?  It takes awhile, but they deliver.  Apple users have always have to wait for conversions.  The past years could be the lost 'golden age' for simultanous releases of software.  We must brace ourselves to wait again while sofware makers count request letters.~\\
Cyn C.~\\
Chicago~\\
\texttt{http://members.aol.com/cyn6x9els}~\\
seeking HTML jobs~\\
Cynthia Clavey~\\
\texttt{http://members.aol.com/cyn6x9els/}~\\

 
		
	
		
\textbf{Scott Schilz --- \emph{\texttt{19 feb 1998, 10:00}}}~\\

Indigo wrote:~\\
\emph{> Well said, Bear. =)}~\\

I disagree. =)~\\

\emph{> You people also forget (in comparing CyberLife to Maxis) that Maxis is a rather large company that has been in business for quite awhile and have many, many sim games out on the market.}~\\

YOU people also forget that a good example of a company that became successful THANKS to the Mac is Bungie. They started out making very good, popular games for the Mac like Pathways Into Darkness and Marathon. Since then, they've had some measure of success in the PC market. If a company such as Bungie can start with only Mac customers and then become a big success (relatively speaking -- they're no Microsoft), then why doesn't Cyberlife bother with a Mac version, since they are almost GUARANTEED to have a big hit on their hands with the PC version?~\\

\emph{> Toby has said it before -- it isn't easy to find an in-house Mac programmer.} ~\\

'Isn't easy' is not the same as 'impossible'. I just kind of get the feeling that they really didn't want to bother with it. ~\\

\emph{> CyberLife is much smaller than Maxis and thus should not be compared to them. If you Mac-ers really are interested in having a Mac version of Creatures, go ahead and learn the programming skills you need to and go apply for a job at CyberLife. =) } ~\\

Don't mind if I do. Just give me a couple of years... ~\\

\emph{> That's the best advice I can give you besides actually buying a PC.} ~\\

<sob> ~\\

\emph{> > For better or worse, you will notice that Cyberlife also neglected to create version of Creatures 2 for people who run Silcone Graphics, Alpha, or Sun Systems machines -- all of which would require a special version to have the program run properly.  (yes I do know people who use some of these machines)  What about those people???} ~\\

\emph{> LOL -- mainly because those types of computers are specifically for designing things and NOT playing games on them. Oh, sure, there are games made for them, but when you look at the whole scheme of things, you use certain computers for certain professions. Gamers, and I say this very loosely, are more likely to have PC's running Windows or DOS rather than SGI's, Alpha's, or Suns running their respective OS's.} ~\\

Funny, whenever I go to Computer City to see the new games, I always end up drooling over the cool stuff in the Sun section. ;-) ~\\

\emph{> > Simple, THERE IS NOT ENOUGH MARKET TO JUSTIFY THE PROGRAMMING OF A VERSION FOR THEM, just like Apple users.} ~\\
\emph{> > Sorry, its just a fact of economics and life in general....} ~\\

\emph{> I agree. Besides, no one's telling you to stop playing Creatures. Toby said that there'd still be support for Creatures 1 -- don't worry.} ~\\

Did he? Please tell me where. Considering the amount of 'support' we've gotten so far, excuse me if I don't breathe a sigh of relief. ~\\

And if you're REALLY desperate for Creatures 2 (and have a few extra thousand bucks) go out and by a PC. I suppose it'll eventually pay of considering the amount of games you'll now be able to play. =) ~\\

OK, I will! (scratches lottery tickets furiously) ;-) ~\\

\emph{> Indigo -- who doesn't HATE Macs, only hates using them. =)} ~\\

I'm sorry... was it a Classic (AKA the boxy, b\&w ones)? Used to have one of those, what a piece of *\#\$\%\^\@!!! ~\\

Scott ~\\

\_\_\_\_\_\_\_\_\_\_\_\_\_\_\_\_\_\_\_\_\_\_\_\_\_\_\_\_\_\_\_\_\_\_\_\_\_\_\_\_\_\_\_\_\_\_\_\_\_
"Oh, look at me! I'm making people happy! I'm the magical man from Happyland in a gumdrop house on Lollipop Lane! ~\\
Oh, by the way, I was being sarcastic." ~\\
-- Homer Simpson ~\\
\_\_\_\_\_\_\_\_\_\_\_\_\_\_\_\_\_\_\_\_\_\_\_\_\_\_\_\_\_\_\_\_\_\_\_\_\_\_\_\_\_\_\_\_\_\_\_\_\_ ~\\

 
		
	
		
\textbf{Scott Schilz --- \emph{\texttt{19 feb 1998, 10:00}}}~\\

Fred Haineux wrote:
\emph{> yeah, I know, PING and company will never read anything this long. Thanks to those that do.} ~\\

Well I did, and I loved every word of it! Very cool and informative. ~\\
Thank you! ~\\
Scott ~\\

\_\_\_\_\_\_\_\_\_\_\_\_\_\_\_\_\_\_\_\_\_\_\_\_\_\_\_\_\_\_\_\_\_\_\_\_\_\_\_\_\_\_\_\_\_\_\_\_\_\_\_\_\_\_\_\_\_\_\_\_\_\_\_\_\_\_\_\_\_\_\_\_ ~\\

"Aw, being a clown sucks! You get kicked by kids, bit by dogs, and admired by the elderly. Who am I clowning? I have no business being a clown. I'm leaving the clowning business to all the other clowns in the clowning business." ~\\
-- Homer Simpson ~\\

 
		
	
		
\textbf{slink --- \emph{\texttt{19 feb 1998, 10:00}}}~\\

On Wed, 18 Feb 1998 23:44:57 -0700, b...@wetware.com (Fred Haineux) wrote: ~\\

\emph{> sl...@netins.net (slink) wrote:} ~\\
\emph{> > The first Mac I saw was a sealed box with a black-and-white screen, a button-challenged mouse and no way to even turn it on without inserting a diskette with someone else's preprogrammed agenda.  It violated Apple's own history of open architecture and power to the user.  It was a royal disappointment.} ~\\

\emph{> Hope I don't upset you, Slink, but your claims are not backed up by the facts:}

I don't mind discussion, but the description I gave did indeed describe the first Mac that I saw.  We'll go thorugh it point-by-ponit as per your agenda.  :) ~\\

\emph{> 1) Four-button mice have been available for the Mac for almost ten years. Apple doesn't sell them, Kensington does. Kensington also sells two button mice, trackballs, and one-button mice. Many of these mice are mechanically identical to PC mice. Because Mac and PC have different serial buses, they have different electrical chips inside, but Mac multi-button mice are at least as useful, configurable, and hackable as PC mice. Heck, I can configure my mouse button to hit "save" and "print" if I wanna.} ~\\

That's fine.  1998 -- 10 = 1988.  The first Mac that I saw was in 1984 and had a one-button mouse. ~\\

\emph{> "Button challenged" is a poor insult to apply, since PCs didn't "come with" mice at ALL for quite some time after Macs did, and there still is no "standard" mouse for Windows PCs. Heaven forbid that Microsoft introduce a new version of their mouse, and build in special support for it in Windows 98/9, like they did with keyboards for Win95!} ~\\

This is off the main topic of the post, but Microsoft's 2-button mouse is pretty much "standard" for PCs.  The leading alternative for some time was Logitech's 3-button mouse but not very many programs could come up with a use for the third button.  I'm pretty sure I've never seen a 1-button PC mouse.  And although a mouse did not "come with" PCs in 1984 it was available, and it had 2 buttons.  Personally I had no real use for a mouse (even though I owned one per PC) until 1993. ~\\

\emph{> 2) Macs, from Day One, came with 8 typefaces, bold, italic, and a variety of sizes. 6 or 7 of the faces had letters with different widths, a typographic innovation of the Middle Ages.} ~\\

So?

\emph{> "Black and White" is a poor insult, because until Windows came out, PCs came with screens full of ugly looking, mono-spaced letters. There was almost NO way to make the type bigger or smaller, or change typefaces, or use proportional type. Even such 'nicenesses' as bold and underline were OPTIONAL. Indeed, original PCs were GREEN AND BLACK. By comparison, black and white was GORGEOUS.} ~\\

What has type font and sizes got to do with color?   ~\\

Yes, the original PCs were greenscreen.  The original computers were teletype.  Once again, so?  Would teletype output have been acceptable as long as it used proportional fonts of differing typefaces and point sizes? ~\\

I guess it's personal taste whether black-and-white is "gorgeous" compared to greenscreen.  However, compared to the colors available for Atari, Commodore and previous Apple products the black-and-white display of the Mac was a bitter disappointment.  Even the RGB (which soon became EGA) of the PC was better than that. ~\\

\emph{> 3) Original Macs didn't boot without an operating system floppy in the slot. Neither did original PCs. Macs booted off of hard disk as early as 1986, roughly the same time as PCs, however, Macs at that time used SCSI, a kind of hard disk bus that allowed 6 drives to be connected, not two, and ALL SCSI hard drives worked with ALL Macs, unlike PCs which had a wide variety of hard disk cards, most of which didn't play nicely with each other.} ~\\

Not true.  The original PCs booted to BASIC if there was no floppy, same as Atari, Commodore and the earlier Apples. ~\\

\emph{> Your criticism is poorly placed, because Macs have up until very recently had superior hard-disk support. PCs have finally become equal to Macs, but they cannot exceed Macs.} ~\\

<chuckle>  They can if Apple is dead and gone, which is where it appears to be headed.  However, I was *not* touting PC superiority, nor even talking about the Macs of today in my post. ~\\

\emph{> 4) Original Macs didn't have much expansion, whereas original PCs did. You are absolutely correct.}~\\

Actually, I was thinking of the ancestry of the Mac, from Apple, as well as the PCs.~\\

\emph{> The Mac II, however, which premiered in 1987, had 6 slots, two floppy drives, and the ability to add up to 6 hard disk drives without adding cards.}~\\

"Original".  Read my post carefully.

\emph{> It also allowed the user to connect and use up to 6 color monitors atonce. Windows could cross screen boundaries, and even go from ablack-and-white screen onto a color screen. Each half of the window wouldbe drawn correctly.}~\\

\emph{> No other computer had ever done that before. Not even \$100,000 Sun/SGI/HP/DEC workstations.}~\\

\emph{> These days, you can make a "desktop" of up to 7.5 million color pixels at once. (You CAN theoretically have a bigger desktop than that, even!)}~\\

\emph{> Last time I checked, Windows doesn't support multiple graphics monitors. It's been 12 years since Mac has.}~\\

So?  I can't watch two screens.  For what would I use two graphics monitors when I can have multiple windows?~\\

\emph{> Even Creatures on a Mac can run in a window spanning several monitors. 've run it spanning 3 SVGA monitors at once, personally.}~\\

Odd thought.  Albia in surround-vision with sixteen 17" monitors in a circle.  It'd be hell getting out to go to the bathroom, though.~\\

\emph{> Calling Macs a "sealed box" was a fair assessment, for the first 18 months of the product history. For the other 150 months, however, it's not accurate, and it's a poor insult because Macs still have expansion capabilities that Windows PCs don't.}~\\

Read my post.  "Original".  And you keep saying "insult".  I posted *my* reaction (royally disappointed) to a Mac of 1984, which was when I was making the decision of which hardware to buy for the best chances of getting what I wanted in the future.  Apple IIe -> 1984 Mac was the wrong trendline for me.  You keep saying that the description of 1984 Macs is an insult to the Macs of today.  So it would be if I was applying to the Macs of today.  I was not, and I do not.~\\

\emph{> 4) Although the original Macintosh didn't have a low-level command system that made hackers happy, it did boot directly into a friendly, powerful, GRAPHICAL user environment that made non-hackers happy. That was the IDEA behind Mac. I can understand that it made some hackers unhappy, but they got what they wanted soon after, as I explain below.}~\\

Yes, Macs did for the Apple community what AOL did for the online community and WIN95 did for the PC community.  Think about that.~\\

\emph{> You are absolutely correct that PCs did often come with a programming language -- the oh-so-powerful BASIC -- which Macintosh did not. Now the funny thing is that the reason Mac didn't come with BASIC is because Microsoft demanded it. Honest injun! MacBASIC was an Apple product set to intro shortly after the Mac did, but Bill Gates himself demanded that Apple pull the plug, allowing Microsoft to sell MS-BASIC to users instead. Faced with losing Word and MultiPlan (predecessor to Excel), Apple gave in. (Surprisingly enough, MS-BASIC for Mac in 1984 was just about as good as Visual BASIC for Windows, which didn't come out until, what, 1994? Mac first, even from Microsoft.)}~\\

Actually, I never even mentioned PCs in my post.  *You* are the one who turned a simple statement of fact about the *first* Mac I ever saw into an excuse to rant about PCs and the superiority of the current Mac product line over same.~\\


If you want to name drop, I've also talked with Jerry Pournelle.  I ate lunch with him in Norfolk, Va. at a NARDAC.  I'm pleased to say that Macintoch computers never once entered our conversation.~\\

\emph{> Within two years, a cheap "phonebook" edition of over 1100 pages of documentation was made available for about \$25, at a LOSS from Apple.}~\\
\emph{> And these days, anyone with some patience can download almost all of Apple's Mac programming documentation FOR FREE (<\texttt{http://devworld.apple.com/}>). And there has never been ANY fee or signature requirement to shipping a Mac product. None at all.}~\\
\emph{> Maybe your original criticism of Macintosh being hacker-unfriendly was accurate, for a short period of time, but I don't think it was valid as of 1986. It certainly isn't valid today. }~\\

It's not valid today to criticize the original Mac?  Why, did it get better in ...~\\

 
		
	
		
\textbf{David ``I Don't Like SPAM'' Wood --- \emph{\texttt{19 feb 1998, 10:00}}}~\\

Fred Haineux wrote:~\\
\emph{> Of course, they spent their MAC profits porting Creatures to Windows NT, which has LESS THAN 1\% MARKETSHARE.}~\\

They spent surprisingly little on the Windows NT version, and that could be a greater source of mirth for us down the road.~\\

Have you ever used Visual C++, Fred? Pretty much all they do is run out a Windows NT version using the MFCs, much the same way they ran out the Mac version using the MFCs.~\\

And we all know how the Mac version came out...~\\

Some people have suggested that in a perverse, roundabout way, Cyberlife is doing us a great favor by NOT releasing a Mac version. The scary thing is that I'm beginning to see a certain logic in it myself...~\\

And you know something? There could be great opportunity in it, too...~\\

Muahahahahahah!~\\

--David~\\

 
		
	
		
\textbf{David ``I Don't Like SPAM'' Wood --- \emph{\texttt{19 feb 1998, 10:00}}}~\\

Scott Schilz wrote:~\\
\emph{> Indigo wrote:}~\\
\emph{> > Well said, Bear. =)}~\\
\emph{> I disagree. =)}~\\

I'll also agree in that disagreement...~\\

\emph{> If a company such as Bungie can start with only Mac customers and then become a big success (relatively speaking -- they're no Microsoft), then why doesn't Cyberlife bother with a Mac version, since they are almost GUARANTEED to have a big hit on their hands with the PC version?}~\\

Because they don't have a big hit on their hands with the PC version yet? Sure, a lot of people are vociferous on this newsgroup and can't get enough of it, but everyone else isn't necessarily as enthused. Can you say "skewed sample?" I think I saw the white shelf on the box of the computer store before getting it, and I didn't give it a second thought at the time. If I hadn't gotten a word-of-mouth recommendation from someone else, I might still not have gotten it.~\\

On the bright side of that scenario, I wouldn't be so emotionally involved over the (already settled) debate regarding the Mac version of the software, and I wouldn't be endlessly frustrated with what's on the mind of that brain-dead creature beating its head against a wall. But enough about Ping...~\\

\emph{> > Toby has said it before -- it isn't easy to find an in-house Mac programmer.}~\\
\emph{> 'Isn't easy' is not the same as 'impossible'. I just kind of get the feeling that they really didn't want to bother with it.}~\\

I would cite the following reasons for their not following through on this:~\\

1. To code a proper version of Mac Creatures, we've already established, would require a dedicated development platform for the Macintosh. Since they already use Macs in the office (Toby said so himself in an attempt to placate us), their primary expenditure would be for the development platform.~\\

(However, I recommended Metrowerks CodeWarrior to them, and I got it myself for about \$300. Either money is *extremely* tight at Cyberlife, or they just didn't bother for some other reason.)~\\

2. They would need to hire dedicated Mac developers to work in-house. They can be hard to find, whereas PC programmers are easy to come by.~\\

(That could be either because nobody knows how to program the Mac, or the PC programming field is glutted, maybe both. In either case, a Mac programmer could probably make steeper salary demands, as per the law of supply and demand. Toby never said whether they couldn't find a Mac programmer, a *competent* Mac programmer, or a Mac programmer *in budget*.)~\\

3. And then they would need to convert what they have of their code base from MFC orientation to MSL orientation, assuming they went with Metrowerks on the Mac.~\\

(This one *would* get tricky. I've seen the project files on both systems, and have to say that Metrowerks' project management is much cleaner. The Mac guy would have to be at least somewhat versant in MFC to remove all the gingerbread MFC dumps in there.)~\\

(I should also note that MSL also compile to Windows 95 \& NT, though I admit I've never had to find out how well. It could be that MSL-Windows is as bad a train wreck as MFC-MacOS, though the fact that Metrowerks actually MAINTAINS their tools ought to account for something.)~\\

\emph{> CyberLife is much smaller than Maxis and thus should not be compared to them. If you Mac-ers really are interested in having a Mac version of Creatures, go ahead and learn the programming skills you need to and go apply for a job at CyberLife. =)}~\\
\emph{> Don't mind if I do. Just give me a couple of years...}~\\

Just make sure your passport is updated, you don't mind changing nationality, and you take along several changes of underwear. Remember, for the forseeable future, they'll have you working on MFC!!~\\

\emph{> > That's the best advice I can give you besides actually buying a PC.}~\\
\emph{> <sob>}~\\

Now now, Scott... let's not resort to name-calling! >:)~\\

\emph{> > >   For better or worse, you will notice that Cyberlife also neglected to create version of Creatures 2 for people who run Silcone Graphics}~\\

Silicone Graphics... isn't that the machine they use to touch up pictures of nude models?

\emph{> > LOL -- mainly because those types of computers are specifically for designing things and NOT playing games on them. Oh, sure, there are games made for them, but when you look at the whole scheme of things, you use certain computers for certain professions. Gamers, and I say this very loosely, are more likely to have PC's running Windows or DOS rather than SGI's, Alpha's, or Suns running their respective OS's.}~\\
\emph{> Funny, whenever I go to Computer City to see the new games, I always end up drooling over the cool stuff in the Sun section. ;-)}~\\

Even funnier thing... Computer City, which is a division of the Tandy Companies (someone had mentioned the TRS-80 a while back, I recall...) closed down their local branch. Now the only big local computer stores are a Best Buy and a CompUSA.~\\

\emph{> > I agree. Besides, no one's telling you to stop playing Creatures. Toby said that there'd still be support for Creatures 1 -- don't worry.}~\\
\emph{> Did he? Please tell me where. Considering the amount of 'support' we've gotten so far, excuse me if I don't breathe a sigh of relief.}~\\

He mentioned it in his big alt.games.creatures posting of a little while back. However, I also duplicated his statement about putting most of the company's resources into supporting the main version and put them next to each other, and the "support for Creatures 1" statement paled in comparison. Scott, FWIW I think your skepticism is justified.~\\

\emph{> And if you're REALLY desperate for Creatures 2 (and have a few extra thousand bucks) go out and by a PC. I suppose it'll eventually pay of considering the amount of games you'll now be able to play. =)}~\\
\emph{> OK, I will! (scratches lottery tickets furiously) ;-)}~\\

So, who here would actually pay \$2000 to play Creatures 2? \$1750? \$1500? Hands, people? I'm not seeing many just yet... \$1250? \$1000? Oh, come on! The thing is unique! No other program like it yet! \$900? \$800? Gee, you people must not be as dedicated as you say...~\\

--David~\\
Aspiring Carnivore... looking to take down a few cattle...~\\

 
		
	
		
\textbf{Shy Puppy --- \emph{\texttt{19 feb 1998, 10:00}}}~\\

In article <34EC7B78.7...@sickofSPAM.erols.com>,~\\
\emph{> So, who here would actually pay \$2000 to play Creatures 2? \$1750? \$1500? Hands, people? I'm not seeing many just yet... \$1250? \$1000? Oh, come on! The thing is unique! No other program like it yet! \$900? \$800? Gee, you people must not be as dedicated as you say...}~\\

Do I hear \$700???! Sold.   Actually, if Creatures does not run significantly faster under Virtual PC 2 than the original VPC, I'm saving up for a PC card for my Mac -- JUST FOR CREATURES.  And I don't have the money.... I'll live in central park...  but I'll be playing Creatures 2!!!!!~\\

I get C2 or die!~\\

Now is that dedication or what?~\\
-SP~\\

 
		
	
		
\textbf{Indigo --- \emph{\texttt{19 feb 1998, 10:00}}}~\\

Scott Schilz wrote in message <34EBFAB4.4...@earthlink.net>...~\\
\emph{> If a company such as Bungie can start with only Mac customers and then become a big success (relatively speaking -- they're no Microsoft), then why doesn't Cyberlife bother with a Mac version, since they are almost GUARANTEED to have a big hit on their hands with the PC version?}~\\

Mainly because of what I already said (Inability to find an in-house Mac programmer). But don't forget how many other companies make games (let alone other software) that are strictly PC.~\\

\emph{> 'Isn't easy' is not the same as 'impossible'. I just kind of get the feeling that they really didn't want to bother with it.}~\\

Perhaps. Be grateful, though, that you got Creatures 1 at least.~\\

\emph{> >  Indigo -- who doesn't HATE Macs, only hates using them. =)}~\\
\emph{> I'm sorry... was it a Classic (AKA the boxy, b\&w ones)? Used to have one of those, what a piece of \*\#\$\%\^\@!!!}~\\

I have used those at school in the past (and they were a piece of \$\#!*), however, our school has since upgraded to...I dunno...PPC's or something that has color and a real monitor (but still that \%@\&! one button mouse!!!).~\\

Indigo -- just addicted to PC's, I guess.~\\

-- ~\\
"Breed, not bread!! Eat wheat, not norns!! Love 'em not loave 'em!!"~\\
The Rights for Norns Society -- We keep your norns from turning into loaves and other products. We are also proud to be allies of the NORN.~\\
Join Today! ;)~\\
\texttt{http://www.sgoinc.com/Indigo/creatures}~\\
Indigo's UIN: \#1530587~\\
Remove BIBBLE from email address to send mail.~\\

----BEGIN\_C-ADD\_CODE\_BLOCK---~\\
Version Number: 1.2~\\
APA/ST d+ s: a---> a?,23/04/80 f- t+++:++ C+~\\
E4++,5++,7+,9++ Q+ W4++ P+ Gd-:Daggerfall/s-:SimCity,~\\
SimTower/o-:Star Wars Monopoly,Streets of SimCity,Nascar 2~\\
-----END\_C-ADD\_CODE\_BLOCK-----~\\

 
		
	
		
\textbf{bllb --- \emph{\texttt{19 feb 1998, 10:00}}}~\\

In article <bc-1802982130250...@cappella.apple.com>,  b...@wetware.com (Fred Haineux) wrote:~\\

\emph{> And they took their Mac profits and ported to Windows NT.}~\\

\emph{> bc}~\\

Um...what profits? From what I have heard, the very reason they aren't going to do a Mac version of Creatures II is because they didn't make a profit on the Mac version of this one.~\\

Flame~\\
get yes life!~\\

-- -- -- -- -- == Posted via Deja News, The Leader in Internet Discussion == -- -- -- -- -- ~\\
\texttt{http://www.dejanews.com/}   Now offering spam-free web-based newsreading~\\

 
		
	
		
\textbf{Toby Simpson --- \emph{\texttt{19 feb 1998, 10:00}}}~\\

Fred Haineux wrote in message ...~\\
\emph{> What Toby and the rest have apparently missed is that 100\% of the Intel/Windows PCs run DOS games, and only 25\% of them run Windows 95 games.}~\\

I'm pleased that you're such an expert on the different market segments that we are aiming for, and have such a deep knowledge of the kinds of machines that Creatures users have.~\\

I can sweet talk everyone if you'd rather, but I've gone to a great deal of effort to explain to people why we have done what we've done, and the reasoning behind it. I'm also attempting to be honest about the situation and explain the facts from our point of view, and despite your theories on the subject, I can assure you we've spent a great deal of time looking very closely at the markets we are aiming for. Likewise, if we'd have got it monumentally wrong, we'd have not sold the units we have.~\\

\emph{> Of course, they spent their MAC profits porting Creatures to Windows NT, which has LESS THAN 1\% MARKETSHARE.}~\\

... and required a 50th of the effort to port. Making a Win32 product run under NT was not difficult, and to maintain the "works under Windows 95" label all Win95 software these days has to run under NT. DirectX prevented this initially, but DirectX 3 addressed this -- allowing a few simple straight forward modifications to ensure Creatures would run under NT also. This is NOT the case with the Macintosh version. If it was that straight forward, we'd have been able to provide Mac updates also. I can re-explain the "why's" for you again if you wish by e-mail, or you can read my previous posts on the subject.~\\

The decision to not produce Creatures 2 on the Macintosh was not a decision that was taken lightly, despite the views of many Mac users in this newsgroup. There are a number of *serious technical issues* that prevent a port being practical at this stage. Should those issues change, then the situation may change also -- however, in the short term, this is very unlikely.~\\

Toby~\\
-- ~\\
Toby Simpson~\\
Executive Producer/Manager -- Creatures Products~\\
CyberLife Technology Ltd.~\\
\texttt{www.creatures.co.uk}~\\

 
		
	
		
\textbf{slink --- \emph{\texttt{20 feb 1998, 10:00}}}~\\

On Thu, 19 Feb 1998 09:37:33 -0500, "David ``I Don't Like SPAM'' Wood"

Hate to disappoint you, David, but the earliest version of C2 that I laid eyes on worked best on one of my NT systems.  <grin>~\\

Sandra -> \texttt{http://www.netins.net/showcase/slink/}~\\
GEEK CODE Version 3.12: GS> AT !d(++)@ s:+ a+ C+++(\$) !U(C/H\$) P(+)@~\\
L E? W++ N++ o? K? w(++)@ O !M V(+) PS+() PE(++)@ Y+ PGP? t++@ 5? X?~\\
R+ tv-- b++(++++)@ DI++++ D G e++++ h+(++)(\$) r+++ x+++~\\

 
		
	
		
\textbf{Toby Simpson --- \emph{\texttt{20 feb 1998, 10:00}}}~\\

Fred Haineux wrote in message (amongst other things) ...~\\
\emph{> If you are foolish enough to compare Macs to Amigas, Commodore Pets, TRS-80 computers, or even Apple II's, bear in mind that Apple sells more Macs in a month than all those models of computers sold in their entire history.}~\\

Could I please trouble you for the figures on this? As someone who knows what both the Amiga (all flavours) and the PET sold in their entire history, I'm very curious as to the source of information you have that demonstrates that Apple sell more Macs in a month than these sold in their entire history.~\\

Best Regards,~\\
Toby~\\
-- ~\\
Toby Simpson~\\
Executive Producer/Manager -- Creatures Products~\\
CyberLife Technology Ltd.~\\
\emph{http://www.cyberlife.co.uk}~\\

 
		
	
		
\texttt{Sujet remplac{\'e} par "C1 Marketing, WAS:Re: not available for MAC?" par Mia Soderquist}~\\
		
	
		
\textbf{Mia Soderquist --- \emph{\texttt{20 feb 1998, 10:00}}}~\\

David "I Don't Like SPAM" Wood wrote:~\\
\emph{> Because they don't have a big hit on their hands with the PC version yet? Sure, a lot of people are vociferous on this newsgroup and can't get enough of it, but everyone else isn't necessarily as enthused. Can you say "skewed sample?" I think I saw the white shelf on the box of the computer store before getting it, and I didn't give it a second thought at the time. If I hadn't gotten a word-of-mouth recommendation from someone else, I might still not have gotten it.}~\\

I don't think that Creatures was marketed very well (in the US, anyway).  I most likely would have gotten it anyway, since I had read a good review, and I am a simulation junkie. Heck, I probably would have gotten it if it had gotten a bad review... BUT...~\\
I know several people who have gotten it recently at my recommendation.~\\
Many had never heard of it before. Of those who did, most thought it was another "Dogz" type program, not understanding the complexity of it...~\\
Everyone I know (personally) who has gotten it has found it addictive.~\\

At least one person went to a software shop to buy it, and the sales person didn't know what she was talking about and tried to sell her Dogz instead.~\\

(FWIW, I am not implying that there is anything wrong with Dogz. It's just different from Creatures. I have Dogz also. I can't resist a computer animal. Heck, with the zoo I have in my house, it becomes plain even to the casual observer that I can't resist any animal at all.)~\\
-- ~\\
Mia Soderquist, tuoz...@bellatlantic.net~\\
\emph{http://www.geocities.com/Athens/Olympus/7330/} (Eclectica)~\\
\emph{http://www.ppbbs.com/creatures/} (The Creatures Playpen)~\\
Mia push e-mail. Get yes Mia.~\\

 
		
	
		
\texttt{Sujet remplac{\'e} par "not available for MAC?" par Toby Simpson}~\\
		
	
		
\textbf{Toby Simpson --- \emph{\texttt{20 feb 1998, 10:00}}}~\\

David "I Don't Like SPAM" Wood wrote in message <34EC7B78.7...@sickofSPAM.erols.com>...~\\
\emph{>1. To code a proper version of Mac Creatures, we've already established, would require a dedicated development platform for the Macintosh.}~\\

... and a *substantial* re-code, as I have explained before. A dedicated development platform won't brush the other issues under the carpet.~\\

\emph{> Since they already use Macs in the office (Toby said so himself in an attempt to placate us), their primary expenditure would be for the development platform. (However, I recommended Metrowerks CodeWarrior to them, and I got it myself for about \$300. Either money is *extremely* tight at Cyberlife, or they just didn't bother for some other reason.)}

I'm really not sure if you feel you are achieving anything here, but I can assure you that you are not. Coming across as bitter and illogical about the situation isn't going to give you a Macintosh version of Creatures 2. ~\\
Likewise, continuing with the general feeling/impression that CyberLife made this decision for all the wrong reasons, over a few glasses of wine, without thinking it through is short sighted and will continue to undermine your arguments. My point about the machines was that one uses the right tools for the job. We use Macintoshes because they are the right tools for certain jobs. If you think I'm trying to placate everyone, humour everyone, etc., then you've monumentally misunderstood and failed to take on board much or all of what I have said.~\\

So, I'll say this once more, and then I'll stop taking part in this conversation (which is a loss for you, as you'll be talking to yourself in future). A compiler is not going to solve our problems. A 747 is little use to a 5 year old that cannot fly a plane. If you really, genuinely believe that buying the right compiler is going to magically solve our other problems (which I have generously taken the time to explain to you, even though it is none of your business) then I am shocked. I have taken on board your recommendations, but that doesn't solve the past. It also doesn't miraculously solve the serious technical issues -- believe me, I wish it did. We made our development decisions for Creatures in 1995, and as I've said before, in the absense of a time machine, there is nothing whatsoever I can do about that.~\\

But even if we had both the staff and the development software, I'm afraid that unless the port was exceptionally straight forward, the shear economics of the situation prevent a viable port. I've yet to see one hard fact or figure that demonstrates otherwise, and I've spent a lot of time working this problem through over the past 6 months.~\\

\emph{> He mentioned it in his big alt.games.creatures posting of a little while back. However, I also duplicated his statement about putting most of the company's resources into supporting the main version and put them next to each other, and the "support for Creatures 1" statement paled in comparison. Scott, FWIW I think your skepticism is justified.}~\\

You're twisting every little word I say into something that you can interpret differently, with an ability that most trashy newspapers would be proud of. All you're doing is alienating people who want to help, and try to help. It is also insulting to us, and for someone who has appeared to base his arguments on common sense in the past I must confess to a little surprise at the substantial lack of it in your current postings. The only comfort I have is that other than the few individuals on this newsgroup who appear to be temporarily detached from reality, the majority of Macintosh owners appear to understand our position (even though it is disappointing), agree that it is a regrettable situation, and jointly hope along with us that things will be better in the future.~\\

I am not answerable to you. A business is not a democracy. However, I believe in being open, and I have done my best to be honest and explain our reasoning -- even to the extent of discussing the individual show stoppers.~\\
Most software companies that have left the Macintosh behind have not even gone that far. On reflection, maybe I should have issued the statement and left it at that. In future, I'll bear this in mind.~\\

Toby~\\
-- ~\\
Toby Simpson~\\
Executive Producer/Manager -- Creatures Products~\\
CyberLife Technology Ltd.~\\
\emph{http://www.cyberlife.co.uk}~\\

	
	  	Messages 51 -- 75 sur 112 -- Tout r{\'e}duire  --  Traduire tous les contenus en Fran\c{c}ais 	< Plus anciens  Plus r{\'e}cents >
	
		
\textbf{bllb --- \emph{\texttt{20 feb 1998, 10:00}}}~\\

In article <34ed7ec7.1073...@news.netins.net>, sl...@netins.net (slink) wrote:
\emph{> Hate to disappoint you, David, but the earliest version of C2 that I laid eyes on worked best on one of my NT systems.  <grin>}~\\
\emph{> Sandra -> http://www.netins.net/showcase/slink/}~\\
\emph{> GEEK CODE Version 3.12: GS> AT !d(++)@ s:+ a+ C+++(\$) !U(C/H\$) P(+)@}~\\
\emph{> L E? W++ N++ o? K? w(++)@ O !M V(+) PS+() PE(++)@ Y+ PGP? t++@ 5? X?}~\\
\emph{> R+ tv-- b++(++++)@ DI++++ D G e++++ h+(++)(\$) r+++ x+++}~\\

What??? Hey! She's seen a version of C2...GET HERRRRRR!!!~\\

*hordes of norn dolls rush out from all sides and jump on Slink while Flame runs for the central computer*~\\

Flame~\\
get yes life!~\\

-- -- -- -- -- == Posted via Deja News, The Leader in Internet Discussion == -- -- -- -- --~\\ 
\texttt{http://www.dejanews.com/}   Now offering spam-free web-based newsreading~\\

 
		
	
		
\texttt{Sujet remplac{\'e} par "not available for MAC?" par jwbo...@mailcity.com}~\\
		
	
		
\textbf{jwbooth --- \emph{\texttt{20 feb 1998, 10:00}}}~\\

In article <6chpoh\$64...@plutonium.compulink.co.uk>, "Toby Simpson" <t...@lobster.cix.co.uk> wrote:~\\
\emph{> The decision to not produce Creatures 2 on the Macintosh was not a decision that was taken lightly, despite the views of many Mac users in this newsgroup. There are a number of *serious technical issues* that prevent a port being practical at this stage. Should those issues change, then the situation may change also -- however, in the short term, this is very unlikely.}

I work a lot on developing systems that have to be portable between different machines and operating systems. In my case it is usually Win32 and some flavour of Unix. I can definitely understand where you are coming from when you talk about 'serious technical issues'.~\\

In a lot of cases it is possible to seperate out GUI, OS services, etc to enable ease of porting but you do tend to gravitate towards the least common denominator of the platforms you are using. In the case of commercial software, especially games, this can be impractical.~\\

Take for example Win32. Using the DirectX API makes the graphics side of things much easier. Even more fundamental is the issue of using COM on the Windows platform. By using COM you can facilitate the kind of inter-application communications that Creatures uses for its applets. You can package parts of your application up into discrete packages that can be used by other programs. The advantages of doing this are large but it directly affects your ability to port to other platforms. Is COM/DCOM available for the Macintosh? I know it's available on Unix but not for any price tag that would make it practible for a commercial game.~\\

I can also understand where Fred and others are coming from. I was a huge OS/2 fan a number of years ago and it was frustrating that no software was available for it. But the commercial realities have to come first. Why would Cyberlife want to invest tens of thousands of dollars in developing a Mac version that will probably not ship enough versions to cover the cost?~\\
Perhaps it is best that they wait until the porting tools mature enough to bring the cost of the port down and the quality of the end product up.~\\

Anyway, just my 2 cents.~\\

Regards,~\\
John.~\\

-- -- -- -- -- == Posted via Deja News, The Leader in Internet Discussion == -- -- -- -- --~\\ 
\texttt{http://www.dejanews.com/}   Now offering spam-free web-based newsreading~\\

 
		
	
		
\texttt{Sujet remplac{\'e} par "C1 Marketing, WAS:Re: not available for MAC?" par jwbo...@mailcity.com}~\\
		
	
		
\textbf{jwbooth --- \emph{\texttt{20 feb 1998, 10:00}}}~\\

In article <34ED66B8.1...@bellatlantic.net>, tuoz...@bellatlantic.net wrote:~\\
\emph{> I don't think that Creatures was marketed very well (in the US, anyway).  I most likely would have gotten it anyway, since I had read a good review, and I am a simulation junkie. Heck, I probably would have gotten it if it had gotten a bad review... BUT...}~\\

I know what you mean. I purchased it because I'm a simulation fan as well. I think it would appeal to a whole other market if they marketed the ability to play with genetics and add new objects, etc. When I first bought the product these features weren't available. It wasn't until a few months later on the Web when someone created COE and a few COB's started to dribble through. Then my interest in the game got a lot higher!~\\

James.~\\

-- -- -- -- -- == Posted via Deja News, The Leader in Internet Discussion == -- -- -- -- --~\\ 
\texttt{http://www.dejanews.com/}   Now offering spam-free web-based newsreading~\\

 
		
	
		
\texttt{Sujet remplac{\'e} par "Secret Agent Man (oops, Woman) (Was:Re: not available for MAC?)" par slink}~\\
		
	
		
\textbf{slink --- \emph{\texttt{20 feb 1998, 10:00}}}~\\

Can't get me -- I'm floating on the normalspace/transwarp boundary.~\\
Thanks be to Daljit of the NORN for my practice at that.  <grin>~\\

Of course I've seen a version of it.  If I'm working on C2 genetics I pretty much need a version of C2 to run since C1 won't support C2 genetics.  :)  While I was at CyberLife I even saw some of the Sacred Source Code.  Oddly enough it looked like any other Visual C++ source code.  No gilded letters or cherubs in the margins.  <big grin>~\\

Sandra -> http://www.netins.net/showcase/slink/~\\
GEEK CODE Version 3.12: GS> AT !d(++)@ s:+ a+ C+++(\$) !U(C/H\$) P(+)@~\\
L E? W++ N++ o? K? w(++)@ O !M V(+) PS+() PE(++)@ Y+ PGP? t++@ 5? X?~\\
R+ tv-- b++(++++)@ DI++++ D G e++++ h+(++)(\$) r+++ x+++~\\

 
		
	
		
\textbf{Rocketgrl --- \emph{\texttt{20 feb 1998, 10:00}}}~\\

::breaking down and sobbing::  why couldn't i be litereate in any languages other than english?!?!?  why?!?  why?!?~\\

 
		
	
		
\textbf{Daljit of NORN --- \emph{\texttt{20 feb 1998, 10:00}}}~\\

\emph{> Can't get me -- I'm floating on the normalspace/transwarp boundary. Thanks be to Daljit of the NORN for my practice at that.  <grin>}~\\

::takes a bow::  No problem.  You wouldn't mind sending a DNA sample so we could prevent others from doing this in the future, would you?  ;)~\\

-- ~\\
Daljit of NORN~\\
"We sacrifice Navens/Naven Boys!"~\\
ICQ UIN\#:  3206846~\\
Homepage:  \texttt{http://members.aol.com/djsred1/index.html}~\\

Daljit:  The NORN are members of the NDWAL, the SPCG, and the RfNS.~\\
Indigo:  ::brandishes Picket sign::  Get yes rights for Norns!~\\
Flame:  Sigfile shmigfile!  Err, get yes Norn Doll!~\\
Karma:  Grendels are your friends too!~\\
Nornan:  ::big nornish grin::~\\

NORN Members:  CindyPsych, Karma, Alyssa, Trissy, Flame~\\
Military/Strategic Advisor:  JDreddfull~\\
Head of Covert Operations:  DAL the self proclaimed demigod~\\
NORN Maintenance Director:  Shon~\\
Head of General Weirdness (CPO):  Spoon~\\

 
		
	
		
\texttt{Sujet remplac{\'e} par "C1 Marketing, WAS:Re: not available for MAC?" par Katie Owen}~\\
		
	
		
\textbf{Katie Owen --- \emph{\texttt{20 feb 1998, 10:00}}}~\\

I saw it on the shelf months before I bought it.  I love sim games, but I have been burned too many times by ones that just aren't complex enough to hold my attention -- or are too monotonous.~\\

I didn't buy it until after I read a review that made it sound like it might keep me entertained for a little while...Yep, 6 months and counting!  For me that is incredibly long.~\\

-Katie~\\
\texttt{http://www.geocities.com/TimesSquare/Arcade/5968}~\\

 
		
	
		
\texttt{Sujet remplac{\'e} par "not available for MAC?" par RudeDog}~\\
		
	
		
\textbf{RudeDog --- \emph{\texttt{21 feb 1998, 10:00}}}~\\

Toby Simpson wrote in message ...~\\
\emph{> Fred Haineux wrote in message (amongst other things) ...}~\\
\emph{> > If you are foolish enough to compare Macs to Amigas, Commodore Pets, TRS-80 computers, or even Apple II's, bear in mind that Apple sells more Macs in a month than all those models of computers sold in their entire history.}~\\
\emph{> Could I please trouble you for the figures on this? As someone who knows what both the Amiga (all flavours) and the PET sold in their entire history, I'm very curious as to the source of information you have that demonstrates that Apple sell more Macs in a month than these sold in their entire history.}~\\

If your talking about commodore sales figures, you should also include sale of the c-64.~\\

RudeDog~\\
-- ~\\
:doing a total overhaul of his web site:~\\

 
		
	
		
\texttt{Sujet remplac{\'e} par "Secret Agent Man (oops, Woman) (Was:Re: not available for MAC?)" par Rocketgrl}~\\
		
	
		
\textbf{Rocketgrl --- \emph{\texttt{21 feb 1998, 10:00}}}~\\

<head explodes in a ball of logic><hands grope around on floor, looking for missing head, and attach it to neck>~\\

Sorry about that <G>~\\

Rocketgrl~\\

 
		
	
		
\texttt{Sujet remplac{\'e} par "C1 Marketing, WAS:Re: not available for MAC?" par Tirjasdyn}~\\
		
	
		
\textbf{Tirjasdyn --- \emph{\texttt{21 feb 1998, 10:00}}}~\\

:)  Because hp outsources don't let you do anything but look at the hp webpage when your not at the phone, I managed to find and article on hp's web site that was treating creatures and dogz and catz and may others as a kindof hoax.  The page provided links and I went to all those pages,  I bought creatures a month later :)~\\

Jasdyn~\\




 
		
	
		
\texttt{Sujet remplac{\'e} par "not available for MAC?" par Lis <<Potato>> Morris}~\\
		
	
		
\textbf{Lis 'Potato' Morris --- \emph{\texttt{22 feb 1998, 10:00}}}~\\

Personally I think it's good to see a game developer interact with the players in this way...~\\
I think it is a crying shame that the game cannot be released for the mac...but I'm no programmer, and your argument seems fair to me(don't bother flaming me- i know you love your macs :-)) Macs are better at graphics etc than pc's, but the game market for them is attrocious.~\\
Don't go, Toby Simpson! The replies from the creatures developers is something I'd like to see in more software companies... the number of arguments I've seen that are based, subconciously, on the fact that the game makers never, ever speak to the gamers are too numerous to mention.~\\

Lis Morris,~\\
Potato Lady Extraordinaire!~\\
Get your hippy norns here!~\\
\texttt{http://ourworld.compuserve.com/homepages/morris\_family/hippy.htm}~\\

 
		
	
		
\texttt{Sujet remplac{\'e} par "Secret Agent Man (oops, Woman) (Was:Re: not available for MAC?)" par DEMO101}~\\
		
	
		
\textbf{DEMO101 --- \emph{\texttt{22 feb 1998, 10:00}}}~\\

\emph{> > Can't get me -- I'm floating on the normalspace/transwarp boundary.}~\\
\emph{> > Thanks be to Daljit of the NORN for my practice at that.  <grin>}~\\

>\emph{::takes a bow::  No problem.  You wouldn't mind sending a DNA sample so we could prevent others from doing this in the future, would you?  ;)}~\\

now that she knows how to make them she could just trap them in a transwarp field, which btw I can make too using the engine of the GrendelPrise, seeing how that's the entire BASIS of Trekkie ship movement...........~\\

ok everyone, get out your headphones 'cause I'm going for the description of the explanation of warp movement~\\

ok, the dilithium chamber is the engine basically, and is the only thing that is known to be able to harness the warp, a subspace (also known as transwarp) flux or whatever you wanna call it is created with the merging of matter and anti-matter, a truely strong power generator indeed, and goes that goes through to subspace, and moves it, subspace and space are linked together, move subspace, you move space..........but because all of space moves it only seems like the ship with the engine moved the stronger engines can move subspace faster and also safer this also explains why higher warp is more dangerous, you make it faster on an the same amount of anti-matter and it stresses the barrier~\\

don't question me on this, my buddy told me about it, he is a total trekkie,he programs on a bunch of Trek MUSHs, and he even speaks fluent Klingon with the aid of a dictionary....~\\

he seriously needs a girlfriend <G>~\\
                               -Das, The logical Athiest gamer   UIN\#6920984~\\
\texttt{http://members.aol.com/DEMO101/Albia.html}~\\
Albia Express~\\
tell me if I can use your norns/grendels/cobs/utilities~\\
founder of DaNs. members: Das, Ping~\\

 
		
	
		
\textbf{Daljit of NORN --- \emph{\texttt{22 feb 1998, 10:00}}}~\\

\emph{> don't question me on this, my buddy told me about it, he is a total trekkie, he programs on a bunch of Trek MUSHs, and he even speaks fluent Klingon with the aid of a dictionary....}~\\

Your friend is right about everything but one minor detail which could really screw someone over.  The explanation he gave you applies for a warp field, not transwarp.  Transwarp is what the BORG use in Star Trek, not the normal races.  The normal races haven't figured out how it works, cause theoretically, you'd need infinite energy to use transwarp.  The BORG have it somehow.  Of course they didn't develop it, they assimilated it from some Delta Quadrant species.~\\

Fluent Klingon???  Wow, that is lifeless.  ;)  What's a MUSH, BTW?  Or don't I wanna know?  ;)~\\

-- ~\\
Daljit of NORN~\\
"We sacrifice Navens/Naven Boys!"~\\
ICQ UIN\#:  3206846~\\
Homepage:  \texttt{http://members.aol.com/djsred1/index.html}~\\

Daljit:  The NORN are members of the NDWAL, the SPCG, and the RfNS.~\\
Indigo:  ::brandishes Picket sign::  Get yes rights for Norns!~\\
Flame:  Sigfile shmigfile!  Err, get yes Norn Doll!~\\
Karma:  Grendels are your friends too!~\\
Nornan:  ::big nornish grin::~\\

NORN Members:  CindyPsych, Karma, Alyssa, Trissy, Flame~\\
Military/Strategic Advisor:  JDreddfull~\\
Head of Covert Operations:  DAL the self proclaimed demigod~\\
NORN Maintenance Director:  Shon~\\
Head of General Weirdness (CPO):  Spoon~\\

 
		
	
		
\texttt{Sujet remplac{\'e} par "not available for MAC?" par Daniel Silverstone}~\\
		
	
		
\textbf{Daniel Silverstone --- \emph{\texttt{22 feb 1998, 10:00}}}~\\

In article <34ed7ec7.1073...@news.netins.net> sl...@netins.net "slink" writes:~\\
\emph{> Hate to disappoint you, David, but the earliest version of C2 that I laid eyes on worked best on one of my NT systems.  <grin>}~\\

Will it work on W95?~\\

-- ~\\
Daniel Silverstone (Kinnison (at) farstar.demon.co.uk)~\\
        !! Proud to be abstaining from breading norns <grin> !!~\\
 ICQ UIN: 5356528~\\
Creatures Site ---> http://www.farstar.demon.co.uk/creatures/~\\

----BEGIN\_C-ADD\_CODE\_BLOCK----
Version Number: 1.2a AST/CS d+() s:+>: a---,09/04/1980
y\textasciitilde  t:+ C+ A S- gQ++/W1,2---,3+,4++,5-,6+
rC/c/B--/N-/E1++,2++,3,4++,5++,6++,7++,8++,9++/O---/o+:,
W+ P++>++\$ Gd:,/s--:,/p*:/o*:
-----END\_C-ADD\_CODE\_BLOCK-----~\\

Get CrEd32 -- Why? -- Because I say so, okay?~\\

Nornan : Nornan Sad!~\\
Hand   : *Skritch*~\\
Nornan : Nornan Angry!~\\
Hand   : *Skritch*~\\
Nornan : Nornan Hot! Nornan Cold! Nornan Hurt!~\\
Hand   : *Load CrEd32*~\\
Nornan : Nornan Happy ::grins::~\\

Get CrEd32 -- Why? -- Because it is endorsed by Nornan ::grin::~\\

 
		
	
		
\textbf{slink --- \emph{\texttt{22 feb 1998, 10:00}}}~\\

On Sun, 22 Feb 98 19:13:58 GMT, Daniel Silverstone~\\

<NewsT...@farstar-nospam.demon.co.uk> wrote:~\\
\emph{> In article <34ed7ec7.1073...@news.netins.net> sl...@netins.net "slink" writes:}~\\

\emph{> >  Hate to disappoint you, David, but the earliest version of C2 that I laid eyes on worked best on one of my NT systems.  <grin>}~\\

\emph{> Will it work on W95?}~\\

Oh, yes, I've since seen it working fine on WIN95.~\\

Sandra -> \texttt{http://www.netins.net/showcase/slink/}~\\
GEEK CODE Version 3.12: GS> AT !d(++)@ s:+ a+ C+++(\$) !U(C/H\$) P(+)@~\\
L E? W++ N++ o? K? w(++)@ O !M V(+) PS+() PE(++)@ Y+ PGP? t++@ 5? X?~\\
R+ tv-- b++(++++)@ DI++++ D G e++++ h+(++)(\$) r+++ x+++~\\

 
		
	
		
\textbf{David ``I Don't Like SPAM'' Wood --- \emph{\texttt{23 feb 1998, 10:00}}}~\\

jwbo...@mailcity.com wrote:~\\

\emph{> In article <6chpoh\$64...@plutonium.compulink.co.uk>, "Toby Simpson" <t...@lobster.cix.co.uk> wrote:}~\\
\emph{> > The decision to not produce Creatures 2 on the Macintosh was not a decision that was taken lightly, despite the views of many Mac users in this newsgroup. There are a number of *serious technical issues* that prevent a port being practical at this stage. Should those issues change, then the situation may change also -- however, in the short term, this is very unlikely.}~\\

\emph{> I work a lot on developing systems that have to be portable between different machines and operating systems. In my case it is usually Win32 and some flavour of Unix. I can definitely understand where you are coming from when you talk about 'serious technical issues'.}~\\

What flavor of Unix? For that matter, how many flavors of Unix are there? 2? 3? 5? 12? You have 'serious technical issues' porting from Unix to *Unix*, fer cryin' out loud!~\\

\emph{> In a lot of cases it is possible to seperate out GUI, OS services, etc to enable ease of porting but you do tend to gravitate towards the least common denominator of the platforms you are using. In the case of commercial software, especially games, this can be impractical.}~\\

"Impractical?"~\\

It's impractical to maintain your application's main algorithms apart from OS services and low-level interface routines for ease of maintenace? So that if a new version of an OS comes out, you only have to edit half your code instead of all of it?~\\

It's impractical to keep them separated for ease of portability? (It would seem that way if you only think one operating system is viable, but don't ask what all the other platforms' users say about you behind your back.)~\\

It's impractical to keep them small and modular for ease of storage and reuse? So that if you decide to create a new program based on certain functions of the original program, you don't have to tear down the old program for the bits of code you need on the new one?~\\

I'm not going to argue with you on the philosophy of program design or software quality assurance, but failing to take advantage of the properties of a computer language when designing and writing software in that language -- well, let's just say investing heavily in software written that way would be impractical.~\\

\emph{> Take for example Win32. Using the DirectX API makes the graphics side of things much easier. Even more fundamental is the issue of using COM on the Windows platform. By using COM you can facilitate the kind of inter- application communications that Creatures uses for its applets. You can package parts of your application up into discrete packages that can be used by other programs. The advantages of doing this are large but it directly affects your ability to port to other platforms. Is COM/DCOM available for the Macintosh? I know it's available on Unix but not for any price tag that would make it practible for a commercial game.}~\\

Does this mean that Unix applications either never talk to each other (for lack of a channel like COM/DCOM), or that developing any Unix applications that talk to each other cost a packet?~\\

There are almost certainly other means of interprocess communication in UNIX which, while not being COM/DCOM or working like them, perform a similar function.~\\

Likewise, there are interprocess communication channels on the Mac. I'm not sure what they are because I'm not a programming guru on the Mac yet, but I'm sure they're there. (FWIW, Creatures Mac used OLE as well, and worked as well as could be expected under the circumstances.)

And whatever operating-system-specific IPC you're using to move data between the main program and the applets would be covered in the low-level routines and maintained separately for each platform ...if you wanted to go to the time and expense to make it good, anyway.

\emph{> I can also understand where Fred and others are coming from. I was a huge OS/2 fan a number of years ago and it was frustrating that no software was available for it.}

OS/2 is probably still a better OS than Windows. More stable, probably easier to use, and plus it's not Windows. >:)

So to do some research, I called up the OS/2 page on IBM's website. It looks like IBM is pushing OS/2 Warp 4 primarily to "the enterprise, small and medium businesses, and connected users." 'Connected users' in this case, I think, refers to those users connected to the network in the enterprise.

This is where the comparison between MacOS and OS/2 falls flat: given OS/2's lackluster performance against Windows, IBM has relegated it primarily to the server side, and few people want games that can be played on a server. Meanwhile, the MacOS is still targeted at all users, home and business alike.

And as for "no software," what you really mean is "no software of the sort you were looking for." Not true on the Mac.

\emph{> But the commercial realities have to come first.}

I just have to ask, 'cos I bet the answer will be amusing: if the commercial realities must always came first, then why would anyone but the laziest slacker with too much free time on his hands buy games. And with the constant need for quality business and management software, why would anyone *write* computer games?

\emph{> Why would Cyberlife want to invest tens of thousands of dollars in developing a Mac version that will probably not ship enough versions to cover the cost?}

Well, they wouldn't...

...but would a proper Mac version really sell that badly?

See, one of the things that bothered me about the decision not to produce for the Mac is that Toby cited poor sales figures. There are two things which would account for the Mac version selling poorly.~\\

1. Bad Placement. Computer stores, which sell primarily PC software, often fail to attract Mac shoppers, unless said shoppers know the secret of hybrid disks, are looking for specific things, and like picking fights with store staff. Most Mac users, however, shop mail-order or online for better pricing, availability, and delivery. That white box may stand out on the shelf, but as a single line on a listing of entertainment software, it stinks. Some better advertising might not have been a bad idea. And if they let Mindscape handle the campaign, well, do the words "blind leading the blind" ring a bell?~\\

2. Bad Press. While it was a crying shame that MacOS 8 came out, and broke Creatures Mac which was designed using barely supported tools intended for the previous version, the death-blow came when various sources reviewed Creatures for the Mac and called it buggy and unstable. Some of them got it to run, and those called it clever, but they still said there were serious issues with it. And while it wasn't Cyberlife's intention to create an eyesore, that's what the press saw it as, and that's how they reported it. Sandy has some of that press on MacNorns just so you can see how utterly damning it really was.~\\

3. Bad Reporting. Some people, like myself, had Creatures recommended to them by word of mouth. It was new at the time, so I ran out to the computer store and bought it. According to EVERY reporting mechanism at the Egghead I went to, I bought that program for a PC, which was a dirty lie. Some people bought their copies of Creatures retail, thus crediting the PC side with sales.~\\

Result: Some people saw the single-line listing in the likes of Cyberian Outpost or a paragraph in MacWarehouse, wondered about Creatures, searched for reviews, and passed it up. Some, like myself, heard about it, were curious, and foolishly bought it so the PC side looked more profitable (Never Again!) I suppose a few could have heard the good recommendations over the bad ink and bought it mail-order, and that's why the Mac side's sales figures weren't zero.~\\

We can't be perfectly sure that's what happened to Creatures 1, but we CAN be sure that won't happen on Creatures 2. Let's hope the misrung Mac sales didn't bolster the PC's sales figures *too much*, shall we? >:)~\\

\emph{> Perhaps it is best that they wait until the porting tools mature enough to bring the cost of the port down and the quality of the end product up.}~\\

I'm seeing that myself now, too. And this is good; I'm now getting some much-needed practice with Myth.~\\

\emph{> Anyway, just my 2 cents.}~\\

Thanks. Here's your change.~\\

 
		
	
		
\textbf{Amy Hughes --- \emph{\texttt{23 feb 1998, 10:00}}}~\\

David "I Don't Like SPAM" Wood wrote:~\\
\emph{> 1. Bad Placement. Computer stores, which sell primarily PC software, often fail to attract Mac shoppers, unless said shoppers know the secret of hybrid disks, are looking for specific things, and like picking fights with store staff.}~\\

I have to agree with this one.  The store I bought it at had it on the PC shelf.  Maybe they only had one copy, in which case, it has a better chance of moving there.  However, why don't they have a Mac/PC shelf for dual disks?  Hell, just a simple list of dual titles that can be used to guide you through the PC shelf would help as well.  I'm willing to do a little looking when I want to try something new.~\\

I'll note though, that when I sent in the registration card, I made a note that I was a Mac user, even though the card didn't ask about my platform.~\\

amy!~\\

 
		
	
		
\textbf{Fred Haineux --- \emph{\texttt{23 feb 1998, 10:00}}}~\\

First, let me make it clear that I am in no way intending to flame Toby Simpson or Cyberlife, or anyone else for that matter. I don't think Cyberlife is "EEEEVIL." I don't think PCs are "EEEEEEVIL." Nope. Not gonna go there, unless someone starts making fun of Macs.~\\

Second, unlike a few of my Compatriots of the Flag of the Six Colors, I accept that Cyberlife is not going to make Creatures 2 Mac. That's their decision, and although it greatly saddens me, I accept it. I do not wish to argue about it.~\\

So here is my theory, and it is mine, and it does NOT have anything to do with dinosaurs being skinny at one end, getting thick-ish in the middle, and then getting skinny again at the other end:~\\

1) Cyberlife thinks they have practically no Mac users, or at least, that's what they seem to say in this newsgroup. I believe that Cyberlife is incorrect in their assessment of Mac share breakdown, because of the following facts:~\\

1a) Cyberlife is not effectively tracking their market breakdown (ie. Mac versus DOS versus Win3.1 versus Win95 versus WinNT versus "Other), because they don't even ask. All Creatures boxes support all platforms, and all boxes contain the same registration card, which does not ask platform.~\\

1b) Industry tracking of market breakdown is notoriously bad. Here is a recent article that supports this claim:~\\

      <\texttt{http://www.gamecenter.com/News/Item/0,3,1512,00.html}>~\\

(This article makes it clear that the so-called "Best Selling Mac Games" list is totally incorrect, because industry reportage files all "cross-platform" games under "PC." Thus, in the Christmas 97 season, the official list of top-selling games was "Myst, Dark Seed, and Duke Nukem 3D Atomic." According to this list, no Mac owners purchased Riven, Myth, Close Combat 2, and Fallout! In actuality, those games were the top 4 Mac games, and the three "official best-selling" games were not even in the Top 10!)~\\

1c) Creatures was a "Top 10" selling program at ComputerWare/MacSource, a regional Mac-only store chain here in the US. (Congratulations!) Need I point out, it's highly unlikely that PC users bought Creatures at ComputerWare. If you need, I can ask Cware to contact you with sales figures.~\\

2) Cyberlife has claimed "marketshare" as a major reason for their Mac policies. Well, I do not believe that that's the major reason. I think the major reaosn Cyberlife does not have a more positive Mac policy is that Cyberlife does not have sufficient Mac expertise. They claim to have been looking for a Mac person for some time now, and unable to find any.~\\

(Perhaps, since they have decided not to make a Mac version of Creatures 2, they might consider hiring a contract programmer instead of a full-time in-house programmer, for the express purpose of creating a "fix" version of Creatures 1 Mac. I do have several references of people "I would trust with my life," and I have repeatedly offered to connect Cyberlife to them, but I haven't received any replies to those messages.)~\\

I would be glad to discuss these issues with Toby or anyone else. I would be glad to help in any way I can to fix Creatures 1 Mac.~\\

This includes:~\\
1) Mac programming;~\\
2) Serious Mac testing (I have real credentials as a full-time-employed tester, I am not kidding);~\\
3) Connection to other Mac programmers and testers who will be glad to help improve Creatures 1 Mac at below their usual salaries, just because they happen to like the game~\\
4) Contacts at Apple who will help Cyberlife programmers improve Creatures 1 Mac, free of charge, free of contractual obligation, and under full NDAs.~\\

So, what say you, Toby?~\\

bc~\\

 
		
	
		
\texttt{Sujet remplac{\'e} par "Object-Oriented Design discussion" par Fred Haineux}~\\
		
	
		
\textbf{Fred Haineux --- \emph{\texttt{24 feb 1998, 10:00}}}~\\

Some people wrote:~\\
\emph{> >  In a lot of cases it is possible to seperate out GUI, OS services, etc to enable ease of porting but you do tend to gravitate towards the least common denominator of the platforms you are using. In the case of commercial software, especially games, this can be impractical.}~\\

>"Impractical?"~\\

Whoah! Hold up a second, there!~\\

Now, let's not get into religious wars about programming style.~\\

The simple fact is that Creatures 1 was written with an object-oriented programming system called MFC, from Microsoft. Translating the entire codebase to a different OOPS (yes, that's the real acronym!) would be HUGE work.~\\

Now, some people think MFC is good, and some people think it's yucky. But that's beside the point. The point is, MFC is what they used, and they cannot switch at this time, without throwing almost every line of code out and starting over.~\\

Since MFC has up and decided that they don't have to support Macintosh any more, this leaves Cyberlife high and dry.~\\

(btw, the first person, by using the word "impractical," meant that games should not "gravitate towards the lowest common denominator" -- a very reasonable sentiment, considering that the games market wants 3D realtime games, not text-based games.)~\\

 
		
	
		
\textbf{jwbooth --- \emph{\texttt{24 feb 1998, 10:00}}}~\\

In article <bc-2402981545350...@cappella.apple.com>, b...@wetware.com (Fred Haineux) wrote:~\\
\emph{> Since MFC has up and decided that they don't have to support Macintosh any more, this leaves Cyberlife high and dry.}~\\
\emph{> (btw, the first person, by using the word "impractical," meant that games should not "gravitate towards the lowest common denominator" -- a very reasonable sentiment, considering that the games market wants 3D realtime games, not text-based games.)}~\\

Yes, that is the gist of what I meant, Rereading my message and Davids reply I may have come across a bit strong. It's unfortunate that MFC support for the Mac is not good enough to do a port of creatures. Perhaps it will be in the near future. I imagine the design decision to integrate tightly to MFC must be what is the biggest obstacle. If you look at software like Doom or Quake which has been ported to a number of platforms (Windows \& Linux that I know of...is it available on the Mac?) it is certainly possible to do it. But there is probably a large code base that needs to be ported at the moment.~\\

I find the recent comments on the possible distortion of Macintosh software sales quite interesting. It's something I had never considered and certainly bears consideration when weighing up the cost of a port. Particularly in light of your own offers to help out with resource, etc. I hope Cyberlife can do something in this respect. The more users of Creaturse the better!~\\

John.~\\

-- -- -- -- -- == Posted via Deja News, The Leader in Internet Discussion == -- -- -- -- --~\\ 
\texttt{http://www.dejanews.com/}   Now offering spam-free web-based newsreading~\\

 
		
	
		
\texttt{Sujet remplac{\'e} par "not available for MAC?" par Daniel Silverstone}~\\
		
	
		
\textbf{Daniel Silverstone --- \emph{\texttt{25 feb 1998, 10:00}}}~\\

In article <34f09f7c.30043...@news.netins.net> sl...@netins.net "slink" writes:~\\
\emph{> Oh, yes, I've since seen it working fine on WIN95.}~\\

YUM! (BTW, Will W95 ever work?)~\\

-- ~\\
Daniel Silverstone (Kinnison (at) farstar.demon.co.uk)~\\
        !! Proud to be abstaining from breading norns <grin> !!~\\
 ICQ UIN: 5356528~\\
Creatures Site ---> http://www.farstar.demon.co.uk/creatures/~\\

----BEGIN\_C-ADD\_CODE\_BLOCK----~\\
Version Number: 1.2a AST/CS d+() s:+>: a---,09/04/1980~\\
y\textasciitilde  t:+ C+ A S- gQ++/W1,2-- -- --,3+,4++,5-,6+~\\
rC/c/B-- --/N-/E1++,2++,3,4++,5++,6++,7++,8++,9++/O-- -- --/o+:,~\\
W+ P++>++\$ Gd:,/s--:,/p*:/o*:~\\
-----END\_C-ADD\_CODE\_BLOCK-----~\\

Get CrEd32 -- Why? -- Because I say so, okay?~\\

Nornan : Nornan Sad!~\\
Hand   : *Skritch*~\\
Nornan : Nornan Angry!~\\
Hand   : *Skritch*~\\
Nornan : Nornan Hot! Nornan Cold! Nornan Hurt!~\\
Hand   : *Load CrEd32*~\\
Nornan : Nornan Happy ::grins::~\\

Get CrEd32 -- Why? -- Because it is endorsed by Nornan ::grin::~\\

		
\textbf{Fred Haineux --- \emph{\texttt{25 feb 1998, 10:00}}}~\\

"Toby Simpson" <t...@cyberlife.co.uk> wrote:~\\
\emph{> So, I'll say this once more, and then I'll stop taking part in this conversation (which is a loss for you, as you'll be talking to yourself in future). A compiler is not going to solve our problems.}~\\

I hope Toby has read my response to this sub-thread, which I've retitled to "Object-Oriented Design discussion."~\\

(summary: I agree with Toby. Moving Creatures code to another compiler/object-oriented programming system is almost "rewriting from scratch" and not, as some might think, a trivial assignment.)~\\

\emph{> I am not answerable to you. A business is not a democracy. However, I believe in being open, and I have done my best to be honest and explain our reasoning -- even to the extent of discussing the individual show stoppers.}~\\
\emph{> Most software companies that have left the Macintosh behind have not even gone that far. On reflection, maybe I should have issued the statement and left it at that. In future, I'll bear this in mind.}~\\

No, Toby, you ARE answerable to your customers. The old saying is WRONG, the customer is NOT always right, but you, as a businessman, have to please most of your customers, and not make the rest of your customers angrier than a bee-stung Grendel.~\\

On the one hand, most game companies just "port and punt" (write a cheezy version for the other platform, and never ever support it). That's one end of the spectrum, and I commend you guys for not laming out.~\\

On the other hand, you can't make all your customers happy, because as Dilbert says, "What our customers want is great new versions of our products for free." You can't keep spending \$\$\$\$\$ doing new versions without getting money back.~\\

A middle ground here, where Mac Creatures gets an update, and then the edict is issued, "Unless there's something horribly wrong that causes you to have to re-format your hard drive to recover, this is it," might be appropriate.~\\

Yes, some people will still be mad, because they want more free stuff.~\\
They will complain, and you can't do a thing about it.~\\

Almost everyone else will see that they have a workable program with some fine features, (almost) equal to the Windows version, and they'll accept what they've been given. They won't be mad enough to harangue endlessly.~\\

A few of us will actually be pleased, because we are looking forward to the improvements, and we'll gladly tell our friends that you did live up to your promises, and provided a far higher level of service to your customers than most games companies.~\\

Business is not a democracy, but either you'll be a wise king or a lonely one.~\\

 
		
	
		
\textbf{Cyn Xphile --- \emph{\texttt{26 feb 1998, 10:00}}}~\\

And what you saw on that screen came out on the paper just as you saw it!  No funcky colors for italic or bold.  No printing over and over to correct stuff you had no idea would overlap.  You draw a circle, you got a circle. And it shipped with a mouse.  To draw you clicke and pulled.  Mac created at home desktop publishing! ~\\
My father was an Apple dealer in '82 when the Lisa came out.  It was placed in the special area of the showroom and I don't think it wowed the IBM users. They were stuck with their networks of IBM's at their businesses and the price of the Lisa was a bit wild.  I was a girl of 11 and liked art.  That computer drew me to it.  What followed Mac's introduction was IBM's mad scramble to save their image of user-friendlyness.~\\

I find it funny Bill Gates didn't like the icons and things.  I remember hating much of the DOS commands you had to learn to run an IBM without wiping out a harddrive.  Even finding a program was a puzzle. Icon's don't solve the problem either.  I'm running into a problem now at work teaching people at work to open the gateway to the old network terminals on the Windows 3.1 machines.  But those women wouldn't have as easy of a time of it as they do now if it were still a DOS enviroment.  Mac pushed friendliness along.~\\
Long live some sort of Mac!~\\

Cyn C.~\\
Chicago~\\

In article <34ec2af2.803...@news.netins.net>, sl...@netins.net wrote:~\\
\emph{> > sl...@netins.net (slink) wrote:}~\\
\emph{> > > The first Mac I saw was a sealed box with a black-and-white screen, a button-challenged mouse and no way to even turn it on without inserting a diskette with someone else's preprogrammed agenda.  It violated Apple's own history of open architecture and power to the user.  It was a royal disappointment.}~\\

Cynthia Clavey~\\
\texttt{http://members.aol.com/cyn6x9els}/~\\

 
		
	
		
\textbf{Ping3506 --- \emph{\texttt{26 feb 1998, 10:00}}}~\\

In article <bc-2502981712170...@cappella.apple.com>, b...@wetware.com (Fred Haineux) writes:~\\
\emph{> No, Toby, you ARE answerable to your customers.}~\\

LOLOLOLOLOLOLOLOLOLOLOLOLOLOL!!!!! If that were true, I'd have Bill Gates under my thumb by now, fetching me coffee while I wiped his systems clean!!!~\\

\emph{> because as Dilbert says,}~\\

OH NO!!! A frantic MAC user who likes Dilbert!! "Cyberlife run Mac users yes!"~\\

\emph{> > "Unless there's something horribly wrong that causes you to have to re-format your hard drive to recover, this is it,"}~\\

How dare you!!! I only used that one 'cos I didn't want to bother uploading the Multimedia Pack to all those FTP sites... <looks leftt>.... <looks right>.... ugh ohh!!~\\

\emph{> > Yes, some people will still be mad, because they want more free stuff.}~\\
\emph{> > They will complain, and you can't do a thing about it.}~\\

1 Question... why are MAC users obsessed with Free Creatures Updates? the way this thread has been going, it would serve you right if Cyberlife charged for the update (if they provide it, which I doubt a great deal) people at Cyberlife have to live you know! you can't expect them to work 12 hours a day, making free stuff for your MACs, and then go home to a Cardboard box in the middle of a dark alley!! free free free free free, why don't you try learning a little trading skills? money = goods!~\\

Money for Mac update = money towards Cyberlife = more updates for Mac and PC versions~\\

No money for Mac update = Cyberlife workers get nothing but a sore wrist from writing miles of MAC code!~\\

I buy Creatures 2 = (maybe) somewhere down the line, the revenue would go towards some sort of Mac update~\\

Wouldn't the world be a much simpler place if we all had decent PC's? and actually PAID for stuff? instead of Free Free Free!! Cyberlife have already been generous with all the free updates/new Norns/cobs to PC users!! I'll wager that if Cyberife knew that releasing Creatures on MACs, would cause so much trouble, they would have thought twice about it!~\\

Ping -- Gamerous Addictedous.
ICQ:  6283750
*******Ping's Norns*************
\texttt{http://www.crosswinds.net/birmingham/\textasciitilde norndude1/Index.HTM}~\\
**********************************

 
		
	
		
\textbf{cjk --- \emph{\texttt{26 feb 1998, 10:00}}}~\\

ping3...@aol.com (Ping3506) wrote:

\emph{> 1 Question... why are MAC users obsessed with Free Creatures Updates?}

{snip}

\emph{> Cyberlife have already been generous with all the free updates/new Norns/cobs to PC users!!}

See, the answer wasn't that tough afterall, was it?

c...@flash.net.SPAMOFF       http://www.flash.net/\textasciitilde cjk         ICQ 4034900
PGP Key : http://pgp.ai.mit.edu:11371/pks/lookup?op=get\&search=0xDFEB475A


	  	Messages 76 -- 100 sur 112 -- Tout r{\'e}duire  --  Traduire tous les contenus en Fran\c{c}ais 	< Plus anciens  Plus r{\'e}cents >
	
		
\textbf{David ``I Don't Like SPAM'' Wood --- \emph{\texttt{26 feb 1998, 10:00}}}~\\

Ping3506 wrote:~\\
\emph{> In article <bc-2502981712170...@cappella.apple.com>, b...@wetware.com (Fred Haineux) writes:}~\\
\emph{> > No, Toby, you ARE answerable to your customers.}~\\
\emph{> LOLOLOLOLOLOLOLOLOLOLOLOLOLOL!!!!! If that were true, I'd have Bill Gates under my thumb by now, fetching me coffee while I wiped his systems clean!!!}~\\

"Accountability" on the part of a reputable software company means delivering goods to the users' satisfaction in exchange for the price paid. After the customers ponies up the old septiroop, he expects software that works.~\\

Microsoft can be said to be exempt from this because 1) they are so freaking huge that they can afford to ignore the requests of individual users, and 2) given their past performance on so many products, anyone that buys their stuff should already be aware of the chances they are taking. "Caveat emptor," as they say.~\\

That, and Bill Gates is this fairly little fellow, and as many users as he's ticked off, there isn't enough of him to spread under that many thumbs. >:)~\\

\emph{> > because as Dilbert says,}~\\

\emph{> OH NO!!! A frantic MAC user who likes Dilbert!! "Cyberlife run Mac users yes!"}~\\

1. Big corporate policy (and the pointy-haired individuals who enforce it) is often the target of Dilbert.~\\

2. Big corporate policy (and the pointy-haired individuals who enforce it) has a near unshakeable belief that the PC is superior.~\\

Do the math. "Ping push oncoming truck!"~\\

\emph{> > > Yes, some people will still be mad, because they want more free stuff.}~\\
\emph{> > > They will complain, and you can't do a thing about it.}~\\

\emph{> 1 Question... why are MAC users obsessed with Free Creatures Updates? the way this thread has been going, it would serve you right if Cyberlife charged for the update (if they provide it, which I doubt a great deal) people at Cyberlife have to live you know! you can't expect them to work 12 hours a day, making free stuff for your MACs, and then go home to a Cardboard box in the middle of a dark alley!! free free free free free, why don't you try learning a little trading skills? money = goods!}~\\

Uhhhhhhh, yeah. Four words, pal: "Ping get life kit?" Remember months back, when the Life Kit was released in the US and you tried to hyp-mo-tyze everyone and his dog Rex into sending it to you? Now you think Mac users are whiny for wanting PARITY? Sheeeeee-IT! Talk about the cauldron calling the kettle black!~\\

(And no, that's not the misquote you think it is; the "pot" was simply not big enough to make the analogy work the way it had to.)~\\

\emph{> Money for Mac update = money towards Cyberlife = more updates for Mac and PC versions}~\\

Translation for the Ping-impaired: "Money for Mac update = money towards Cyberlife = more updates for PC versions."~\\

("Ping-impaired..." that sounds almost redundant, doesn't it?)~\\

\emph{> I buy Creatures 2 = (maybe) somewhere down the line, the revenue would go towards some sort of Mac update}~\\

...but not if you can help it.~\\

\emph{> Wouldn't the world be a much simpler place if we all had decent PC's?}~\\

I agree completely. It's a pity our definitions of what the decent personal computer is diverge so completely.~\\

Ping prefers the machine that requires CONFIG.SYS, AUTOEXEC.BAT, WIN.INI, SYSTEM.INI, *.DLL files, and IRQ settings over the machine that doesn't need them ...in the name of simplicity. What's wrong with THIS picture, folks??~\\

\emph{> and actually PAID for stuff? instead of Free Free Free!! Cyberlife have already been generous with all the free updates/new Norns/cobs to PC users!!}~\\

I'll defer to CJK's answer in this case.~\\

But I will add this: "Hear that, Toby? It's okay to make 'em pay for Creatures 2 updates!" >:)~\\

\emph{> I'll wager  that if Cyberife knew that releasing Creatures on MACs, would cause so much trouble, they would have thought twice about it!}~\\

Cyberlife would have done a LOT of things differently if they knew what sort of raging horizontal poopy-storm their current course would have taken them through. But cutting off the Mac would probably not be one of them.~\\

I bet your parents would have done a few things differently too...~\\

--David~\\
Aspiring Carnivore~\\

 
		
	
		
\textbf{Daljit of NORN --- \emph{\texttt{26 feb 1998, 10:00}}}~\\

\emph{> I bet your parents would have done a few things differently too...}~\\

Woah, ok, someone needs a timeout.  I've stayed out of this till now.  But this is getting ridiculous.  It's not a good thing that Macs aren't getting C2.  I don't think it's fair.  But that's just my opinion, and Cyberlife has their reasons.  It's settled.  Macs won't have C2.  Get over it.  Arguing about how much cooler and better Macs are than PCs and how much more awesome you are for having one isn't gonna get you C2.  Telling Cyberlife how they should make a Mac version isn't getting you C2 either.  And saying something like the above statement, to anyone, especially someone who has ABSOLUTELY NOTHING to do with Cyberlife and C2 isn't gonna get you what you want either.  Flame me for this post if you wish, the NORN adaptive shielding are long used to taking heat.~\\

-- -- ~\\
Daljit of NORN~\\
"We sacrifice Navens/Naven Boys!"~\\
ICQ UIN\#:  3206846~\\
Homepage:  \texttt{http://members.aol.com/djsred1/index.html}~\\

Daljit:  The NORN are members of the NDWAL, the SPCG, and the RfNS.~\\
Indigo:  ::brandishes Picket sign::  Get yes rights for Norns!~\\
Flame:  Sigfile shmigfile!  Err, get yes Norn Doll!~\\
Karma:  Grendels are your friends too!~\\
Nornan:  ::big nornish grin::~\\
Grndl:  ::big grendel grin::~\\

NORN Members:  Karma, Alyssa, Trissy, Flame~\\
Military/Strategic Advisor:  JDreddfull~\\
Head of Covert Operations:  DAL the self proclaimed demigod~\\
NORN Maintenance Director:  Shon~\\
Head of General Weirdness:  Spoon~\\
Official Counselor of the GroupMind:  CindyPsych~\\

 
		
	
		
\textbf{Toby Simpson --- \emph{\texttt{26 feb 1998, 10:00}}}~\\

Fred Haineux wrote in message ...~\\
>(summary: I agree with Toby. Moving Creatures code to another compiler/object-oriented programming system is almost "rewriting from scratch" and not, as some might think, a trivial assignment.)~\\

Indeed.~\\

\emph{> No, Toby, you ARE answerable to your customers. The old saying is WRONG, the customer is NOT always right, but you, as a businessman, have to please most of your customers, and not make the rest of your customers angrier than a bee-stung Grendel.}~\\

You're absolutely right -- so the chances are I failed to make my point particularly clearly. The intention was to say that I do not have to explain why we make internal corporate decisions.~\\

Our customers are our most valuable asset -- if we didn't have any, we would not have a business, and it is by listening to them that we have been in a position to do so much with Creatures 2. It is always regrettable if we have to lose customers, particularly those who enjoy our products would dearly like to have more. And, as I have said, the decision to not port Creatures to the Mac was not taken lightly and I am frustrated that we have not been able to do more.~\\

Whilst I know it does not magically make everything right, we are going to be releasing an Object Injector for the Macintosh in the coming few weeks.~\\
This was programmed as a separate application and does not use MFC. Of course, it will not inject any COBs that utilise Creatures 1.0.1 or above macro commands, it will inject many of the most popular ones. Some code has been put in to attempt to spot "bad" COBs that will not run well in advance of injecting them to prevent crashing Macintosh Creatures.~\\

I'll post information about the actual release date on the web site and in this newsgroup.~\\

We regret that we have not been able to support Macintosh Creatures, and particularly with the "heated conversations" over the last week or two I think we've all gotten a little "worn". I appreciate that you fully understand the reasons why we are not porting C2 to the Mac. I also ask you to appreciate that it is this same reasoning that has caused us the many problems that continue to stall further add-ons to the existing Macintosh code-base. We've spent a great deal of time and money researching methods of trying to fix these issues (and a couple of you have been in receipt of e-mail from me a few months back saying that we intended to make those fixes), and it is frustrating for us too that it is not been possible.~\\
Upsetting any customer is a bad thing to do, particularly ones that believe in the work that we have done on Creatures so much. I know the object injector doesn't fix everything, but I hope that it demonstrates that there was and is intent on our part to do as much as we can given the situation we have.~\\

I would just like to emphasise that whilst there is not going to be a Macintosh version of Creatures 2, we are continuing to look at ways of better supporting Mac C1. Likewise, the door to further updates is not closed, as I hope the Object Injector will demonstrate.~\\

Regards,~\\
Toby~\\
-- ~\\
Toby Simpson~\\
Executive Producer/Manger -- Creatures Products~\\
CyberLife Technology Ltd.~\\
\texttt{www.creatures.co.uk}~\\

 
		
	
		
\textbf{Sparrow --- \emph{\texttt{26 feb 1998, 10:00}}}~\\

When I really get into programming, I will never cross-platform. You know why? Do you? Well?~\\

It's because I don't know how...~\\

 
		
	
		
\textbf{rajamaki --- \emph{\texttt{27 feb 1998, 10:00}}}~\\

Yeah, Mr "Body by Fisher, Brains by Mattel" Ping just doesn't get it. I've put him in my kill file so I don't have to read his malinfomred postings and get totally upset. That's all I gotta say.~\\

/Sandy~\\
--------------~\\
MacNorns -- \texttt{http://www.crosswinds.net/brussels/\textasciitilde rajamaki}~\\
ICQ\# 7029961~\\

-- -- -- -- -- == Posted via Deja News, The Leader in Internet Discussion == -- -- -- -- -- ~\\
\texttt{http://www.dejanews.com/}   Now offering spam-free web-based newsreading~\\

 
		
	
		
\textbf{DEMO101 --- \emph{\texttt{27 feb 1998, 10:00}}}~\\

\emph{> "Ping push oncoming truck!"}~\\
\emph{> Translation for the Ping-impaired: "Money for Mac update = money towards Cyberlife = more updates for PC versions."}~\\
\emph{> ("Ping-impaired..." that sounds almost redundant, doesn't it?)}~\\
\emph{> ...but not if you can help it. }~\\
\emph{> Ping prefers the machine that requires CONFIG.SYS, AUTOEXEC.BAT, WIN.INI, SYSTEM.INI, *.DLL files, and IRQ settings over the machine that doesn't need them ...in the name of simplicity. What's wrong with THIS picture, folks??}~\\
\emph{> I bet your parents would have done a few things differently too...}~\\

EGADS!!!  what exactly did Ping do to YOU to make you THIS angry?  so he doesn't like Macs?  did he deserve all THAT??~\\

and I don't really like MACs either......gonna flame me too?  mainly I dislike them for 2 reasons.....~\\
1.  Apple makes them....<G>~\\
2.  Games are much easier to find on PC, and I have much more experience with PCs, and there are many more brands to choose from, and....ohhh....the list goes on (ok, maybe that was more than 2 <G>)~\\

note :  this is NOT flaming MAC users....I am merely stating my preference and using facts (and a little of the ole Das charm, hee hee) to state my case.......~\\

-Das, The logical Athiest gamer   UIN\#6920984~\\
\texttt{http://members.aol.com/DEMO101/Albia.html}~\\
Albia Express~\\
tell me if I can use your norns/grendels/cobs/utilities~\\
founder of DaNs. members: Das, Ping~\\


		
\textbf{Mannkind --- \emph{\texttt{27 feb 1998, 10:00}}}~\\

Yeh, thats right~\\
Go ahead and Flame me i don't give a rats A\$\$ cause MACS suck they are pieces of crap that don't have the right to be built....has anyone seen that commerical that shoes a snial with a PII processer on its back, then it says "most people think the PII is the fastests chip availble and then its shows some piece of crap MAC chip and says its faster""~\\
Yeh right MACS(one button exspicaly) are the crappist computers avaible infact they shouldn't even be called computers....~\\

Thats my two cents and I don't care if I'm flamed~\\

Mannkind~\\
-Creator of 23 norn world and HEAD norns and RELEASED the shotgun.cob~\\
-Got more sig space yes!~\\

***************************************************~\\
Creatures Page: http://www.crosswinds.net/seattle/\textasciitilde mannkind/creatures.htm~\\
Grand Theft Auto Page: http://www.crosswinds.net/seattle/\textasciitilde mannkind/gta.htm~\\
(Now supporting frames!)~\\
***************************************************~\\

 
		
	
		
\textbf{RudeDog --- \emph{\texttt{27 feb 1998, 10:00}}}~\\

Toby Simpson wrote in message <6d4td0\$fr...@plutonium.compulink.co.uk>...~\\
\emph{> Whilst I know it does not magically make everything right, we are going to be releasing an Object Injector for the Macintosh in the coming few weeks. This was programmed as a separate application and does not use MFC. Of course, it will not inject any COBs that utilise Creatures 1.0.1 or above macro commands, it will inject many of the most popular ones. Some code has been put in to attempt to spot "bad" COBs that will not run well in advance of injecting them to prevent crashing Macintosh Creatures.}~\\

\emph{> I'll post information about the actual release date on the web site and in this newsgroup.}~\\

Thank you Toby, (and the rest of CyberLife of course) for making an Object Injector for the Mac users.~\\

Maybe some of the COB writers out there, could make MAC versions of their stuff that use just 1.0 macro comands.~\\

RudeDog~\\
-- ~\\
Send in those recipe for the Creatures cookbook!  ;->~\\

Nornivore: likes creatures so much, he has to eat them too.~\\

 
		
	
		
\textbf{Stuart Taylor --- \emph{\texttt{27 feb 1998, 10:00}}}~\\

Flibble Snoff~\\

 
		
	
		
\textbf{David ``I Don't Like SPAM'' Wood --- \emph{\texttt{27 feb 1998, 10:00}}}~\\

DEMO101 wrote:~\\
\[Every one of my original ventings snipped\]~\\
\emph{> EGADS!!!  what exactly did Ping do to YOU to make you THIS angry?}~\\

Angry? That wasn't angry. That was *playful* backbiting. Maybe certain parts were unnecessarily ferocious, but I said them not in anger, but mere callousness, sarcasm, spite, and a desire to expose Ping's biases.~\\

(The last was what got me in trouble with Toby a little while back; it's not that I was twisting his words; let's just say that I was "reading between the lines a little too aggressively.")~\\

There's a safety valve that kicks in when I get *angry:* I write at such length that somewhere near the end, I calm down, see exactly what kind of a butthead I sound like, and usually just cancel the message without sending it.~\\

\emph{> so he doesn't like Macs?  did he deserve all THAT??}~\\

"All that?" My biggest stinger in the set was around four lines. The rule (unwritten until now) in "playful backbiting" is to make each insult or derision as short as possible while still being effective. I like to brush up every so often as a writing exercise.~\\

And judging by your and Daljit's reaction, I've Still Got It.~\\

\emph{> and I don't really like MACs either......gonna flame me too?}~\\

Naw, the drive in was too nice. I'm still just feeling playful. >:)~\\

\emph{> mainly I dislike them for 2 reasons.....}~\\
\emph{> 1.  Apple makes them....<G>}~\\

I'll admit, Apple has inspired less than total confidence in me too. And there are a lot of Mac users that have had their faith in Apple waver. Things like the autoigniting PB5300, certain high-end Macs that weren't 100\% compatible with the Macintosh standard, etc.~\\

But then Apple turns out things like QuickTime and QuickDraw 3D, which are cross-platform implementations of Apple technologies. The PowerPC chip is being used in places beside Apple's boxes, and if it weren't for the ever-increasing speeds of those, you would never see significantly faster chips from Intel.~\\

In short, most Mac users have a love-hate relationship with Apple. But none of them want to see it go away completely.~\\

(Forgive the following passage; I had to break it out into separate points to comment on each one individually. The text of each point is kept as it was typed.)~\\

[2a. Games are much easier to find on PC]~\\

\texttt{http://www.warehouse.com/MacWarehouse/Software/Home\_Computing/Enterta...}~\\
\texttt{http://www.warehouse.com/MacWarehouse/Software/Home\_Computing/Games/}~\\
\texttt{http://www.macconnection.com/}~\\
\texttt{http://www.cc-inc.com/cfm/frames/macmall/showcase/classes.cfm?categor...}~\\
\texttt{http://www.outpost.com/gameoutpost/}~\\

I haven't noticed any trouble finding Mac games. In fact, like I mentioned earlier, I have entirely TOO MANY games on my system. There are more PC games, though, because the supply of decent, unemployed Mac programmers is short.~\\

[2b. I have much more experience with PCs]~\\

As a wise man once said, "You made your bed... now lie in it."~\\

[2c. There are many more brands to choose from]~\\

The most reputable ones are Compaq (although people still complain about them from time to time)~\\

...and Dell (which people still complain about from time to time).~\\

There's IBM of course (but they also make the PowerPC chips for the Apple/IBM/Motorola alliance; what do you suppose they know that you don't?).~\\

Then there are companies like Packard Bell which have a glowing reputation ...from being nuked by dissatisfied customers so often.~\\

Sony has been making PCs for some time, but they don't have much of a reputation at it; they've done soooooooo much better with Walkmans and Playstations anyway.~\\

And then there are more small Pacific Rim companies than you can shake a stick at, companies you've never heard of which produce computers barely compatible with the PC "standard" and will fall apart the moment you look at them funny.~\\

"More" != "Better."~\\

And remember that I pointed out that even Apple has had its problems. Show me a brand of computer that no user complains about, and I'll show you a computer nobody is using.~\\

\emph{> note :  this is NOT flaming MAC users....I am merely stating my preference and using facts (and a little of the ole Das charm, hee hee) to state my case.......}~\\

Likewise, I'm not necessarily flaming PC users. I'm offering facts in answer to your facts and using certain of my charms as well. My charms just tend to be ...toothier.~\\

--David~\\
Aspiring Carnivore~\\

 
		
	
		
\textbf{David ``I Don't Like SPAM'' Wood --- \emph{\texttt{27 feb 1998, 10:00}}}~\\

Daljit of NORN wrote:~\\

\emph{> > I bet your parents would have done a few things differently too...}~\\

\emph{> Woah, ok, someone needs a timeout.  I've stayed out of this till now.  But this is getting ridiculous.}~\\

Awwwww, And I thought we were all just having fun.~\\

This is how so many misunderstandings spring up on the Internet: everyone has slightly different boiling points. So when (A) says something slightly desparaging about (B) in fun, (B) may take it as slightly desparaging, or he might take it as a declaration of war.~\\

EXERCISE: Consider that problem of communications for a moment, and then define a "troll" and a "flamer" in that context.~\\

(I'm serious about this! Call it a UseNet mental exercise. Change the name of the thread if you have to, but please respond to the above exercise!)~\\

\emph{> It's not a good thing that Macs aren't getting C2.  I don't think it's fair.}~\\

Agreed.~\\

\emph{> But that's just my opinion, and Cyberlife has their reasons.}~\\

And Toby posted them. Granted, I perhaps read a little more into them than was there, but they are understood.~\\

\emph{> It's settled.  Macs won't have C2.}~\\

Um-hum. This is what I've come to understand.~\\

\emph{> Get over it.}~\\

Get over what?~\\

\emph{> Arguing about how much cooler and better Macs are than PCs and how much more awesome you are for having one isn't gonna get you C2.}~\\

Granted, but it will get under the skin of anyone who says the Mac isn't worthy to act as Bill Gates' footrest. This I do simply for irritation value.~\\

\emph{> Telling Cyberlife how they should make a Mac version isn't getting you C2 either.}~\\

Oh, I know better than to hope for that too. But in that case, I figured if there was a chance it'd help...~\\

\emph{> And saying something like the above statement, to anyone, especially someone who has ABSOLUTELY NOTHING to do with Cyberlife and C2 isn't gonna get you what you want either.}~\\

Are you so sure?~\\

I don't expect to see C2 Mac now or ever. In fact, I'll be surprised to see *any* non-C1-Mac releases from Cyberlife at this point. Now ask yourself what it is I *do* want at this point.~\\

\emph{> Flame me for this post if you wish, the NORN adaptive shielding are long used to taking heat.}~\\

How do they handle factual arguments? Or acid? I tend to use both...~\\

[19 lines of signature snipped]~\\

"David push more sig space over Ping's dead body!"~\\

--David~\\
Aspiring Carnivore~\\

 
		
	
		
\textbf{David ``I Don't Like SPAM'' Wood --- \emph{\texttt{27 feb 1998, 10:00}}}~\\

Mannkind wrote:~\\
\emph{> Yeh, thats right Go ahead and Flame me}~\\

No. You didn't say "please."~\\

\emph{> i don't give a rats A\$\$}~\\

Multiple responses came to mind immediately. Please choose from the following:~\\
"Ahhhh, HORDING them, I see..."~\\
"What's wrong, 'z' key broke?"~\\
"Back half and stiff tail of a rat: Mannkind's definition of an 'all-day sucker.'~\\

\emph{> cause MACS suck}~\\

Can you be a little more specific here?~\\

I mean, I could say "you suck," but that doesn't have the same impact as, say, "you have no idea how to use punctuation marks," or "you are an ignoramus with the charisma of a small soap dish," or even "you lack faith and are poorly trained."~\\

\emph{> they are pieces of crap that don't have the right to be built.}~\\

(Oooh, a punctuation mark! There goes THAT streak...)~\\

Sadly, you are mistaken: In late 1997, the practice of manufacturing Macintoshes was legalized when a new Bill was passed... it was Gates'check for \$150 million to Apple.~\\

\emph{> ...has anyone seen that commerical that shoes a snial with a PII processer on its back, then it says "most people think the PII is the fastests chip availble and then its shows some piece of crap MAC chip and says its faster"}~\\

Actually, no, it shows an entire G3 system at the tail-end of that commercial. Another great image I think they could use for that message is a calendar showing the month of January, over which comes a wave of gooey dark molasses, and riding atop that, well... why spoil the surprise?~\\

(Although the idea of shoeing (i.e. stomping on) the snail and PII does seem an attractive image too...)~\\

\emph{> Yeh right}~\\

And the thing they dared base that statement on? Research. The webpage:~\\

\texttt{http://www.apple.com/hotnews/features/bytemark.html}~\\

gives an overview of the outcome of the research, as well as links to information in much greater detail. And while there are ways that "testing professionals" can jigger test results (an article called "Making a P166 Outperform a P2/300" or something like that), I've glanced over the parameters of the Bytemark analysis, and there appears to be no such jiggering there.~\\

And if the claim that a G3/266 is faster than a PII/333 still sounds outrageous, answer this: which is stronger, 266 horses, or 333 dogs?~\\

\emph{> MACS(one button exspicaly) are the crappist computers avaible}~\\

............................especially.........crappiest..........available~\\

At least I can get a spelling checker for it, pal...~\\

If you really need a multiple button mouse, they're available. They're just not *necessary*.~\\

\emph{> infact they shouldn't even be called computers....}~\\

That I could agree with; in fact, the whole trend of the industry has been away from "computers as computers" to "computers as information appliances:"~\\

"Desktop boxes so simple anyone can use them."~\\

(Given the tone of the rest of the message toward the original poster, readers are encouraged to read whatever meaning they feel is appropriate for the above statement. Because odds are, they're RIGHT. >:)~\\

\emph{> Thats my two cents and I don't care if I'm flamed}~\\

I've seen your writing style. Those aren't pennies, Mannkind, they're slugs. And not the molluscy kind either, although in light of the earlier discussion about snails...~\\

\emph{> Mannkind}~\\

(who appears to be neither, but does enjoy professional wrestling, which is only one of the two)~\\

--David~\\
Aspiring Carnivore~\\

 
		
	
		
\textbf{Ping3506 --- \emph{\texttt{27 feb 1998, 10:00}}}~\\

In article <19980227034901.WAA14...@ladder02.news.aol.com>, demo...@aol.com (DEMO101) writes:~\\
\emph{> EGADS!!!  what exactly did Ping do to YOU to make you THIS angry?  so he doesn't like Macs?  did he deserve all THAT??}~\\

::Takes out "I hate Macintosh, don't we all" animated .gif and slaps it on his Main Page::~\\

Actually... I don't think it's fair that MAC users aren't getting Creatures 2, but I still don't think they should keep harassing everyone under the sun about it!~\\

Ping -- Gamerous Addictedous.~\\
ICQ:  6283750~\\
*******Ping's Norns*************~\\
\texttt{http://www.crosswinds.net/birmingham/\textasciitilde norndude1/Index.HTM}~\\
**********************************~\\

 
		
	
		
\textbf{Fred Haineux --- \emph{\texttt{27 feb 1998, 10:00}}}~\\

"Toby Simpson" <t...@lobster.cix.co.uk> wrote:~\\
\emph{> Whilst I know it does not magically make everything right, we are going to be releasing an Object Injector for the Macintosh in the coming few weeks.}~\\

Toby and I have had some off-line discussions about this, but I do want to check in saying THANK YOU and I CAN'T WAIT and happy sentiments like that.~\\

 
		
	
		
\textbf{Fred Haineux --- \emph{\texttt{27 feb 1998, 10:00}}}~\\

ping3...@aol.com (Ping3506) wrote:~\\
>::Takes out "I hate Macintosh, don't we all" animated .gif and slaps it on his Main Page::~\\

I hope you will. Save us all a lot of trouble.~\\

\emph{> Actually... I don't think it's fair that MAC users aren't getting Creatures 2, but I still don't think they should keep harassing everyone under the sun about it!}~\\

Nobody's "harassing" anyone except you.~\\

Mac users said, "We'd like Creatures 2." Toby said "No." Most of us said, "OK."~\\

And then Ping said, "OH MY GOD, WE'RE SURROUNDED BY ANGRY MAC USERS WHO WON'T STOP HARASSING US."~\\

Why are you so damn afraid of the Mac? Because you wish you'd bought one? Sheesh.~\\



\texttt{Sujet remplac{\'e} par "Object-Oriented Design discussion (MAC)" par Fred Haineux}~\\
	
		
\textbf{Fred Haineux --- \emph{\texttt{27 feb 1998, 10:00}}}~\\

In article <6d0650\$6a...@nnrp1.dejanews.com>, jwbo...@mailcity.com wrote:~\\
\emph{> It's unfortunate that MFC support for the Mac is not good enough to do a port of creatures. Perhaps it will be in the near future.}~\\

I hope so, but so far Microsoft seems to be moving in the opposite direction. I wish I had more info and could say otherwise.~\\

 
		
	
		
\texttt{Sujet remplac{\'e} par "not available for MAC?" par Fred Haineux}~\\

\textbf{Fred Haineux --- \emph{\texttt{27 feb 1998, 10:00}}}~\\

I think Us Mac Users should just totally ignore Ping's rantings.~\\

He's so hung up on his PC Superiority Complex that he simply must try to prevent Mac users from having calm, rational discussions with Cyberlife and others.~\\

!!!!!!!!!!!!!!!!!!!!!!!!!!!!!!!!!!!!!!!!!!!!!!!!!!!!!!!!!!!!!!!!!!!!~\\
!!!!!!!!!THIS HURTS THE ENTIRE CREATURES COMMUNITY, PING.!!!!!!!!!!!~\\
!!!!!!!!!!!!!!!!!!!!!!!!!!!!!!!!!!!!!!!!!!!!!!!!!!!!!!!!!!!!!!!!!!!!~\\
!!!!!!!!!!!!!!!!!!!!     It hurts YOU.     !!!!!!!!!!!!!!!!!!!!!!!!!~\\
!!!!!!!!!!!!!!!!!!!!!!!!!!!!!!!!!!!!!!!!!!!!!!!!!!!!!!!!!!!!!!!!!!!!~\\

Ping, there are plenty of games and programs that are PC-only. Go be smug about them. Creatures is PC *and* Mac, and us Mac users are just as entitled to support, discussion, and the right not to be flamed just because we bought Not-Your-Personal-Favorite-Computer.~\\

Whether YOU feel that Mac users should be important is irrelevant. You are irrelevant. Cyberlife feels that Mac users are valuable customers that should be supported. YOU don't get to vote on this.~\\

 
		
	
		
\textbf{Daljit of NORN --- \emph{\texttt{27 feb 1998, 10:00}}}~\\

\emph{> Awwwww, And I thought we were all just having fun.}~\\

::pulls out shotgun::  I wanna have fun too!  <G>~\\

\emph{> This is how so many misunderstandings spring up on the Internet: everyone has slightly different boiling points. So when (A) says something slightly disparaging about (B) in fun, (B) may take it as slightly disparaging, or he might take it as a declaration of war.}~\\

\emph{> EXERCISE: Consider that problem of communications for a moment, and then define a "troll" and a "flamer" in that context.}~\\

OK, so you wanna be known as a troll, not a flamer?  ;)  But I see what you're saying.~\\

\emph{> > Get over it.}~\\

\emph{> Get over what?}~\\

Before I read this message, and your reply to Das, it seemed that you were being all pouty about not having a Mac version of C2.  I see now that you are just a person who likes to argue.  <G>  And you have actually accepted the fact that Macs probably won't have C2.~\\

\emph{> Granted, but it will get under the skin of anyone who says the Mac isn't worthy to act as Bill Gates' footrest. This I do simply for irritation value.}~\\

I'm a PC user, but I've messed with Macs somewhat, and my blind messing around with their systems, plus seeing things like QuickTime VR ::drool:: (ask Anthony of NORN more about this) make me believe that Bill Gates isn't worthy to act as a Mac's mousepad.  I'd get a Mac myself, but there are a few problems.  All my software instantly becomes garbage, I'm addicted to 2 button mice (I'd be willing to attend 2 button mouse-a-holics anon to fix this though), and there's just not a lot of software for Macs.  I have to go with what works.  (Yes, I see the opening to say something like "PCs don't work.... etc.")~\\

\emph{> > And saying something like the above statement, to anyone, especially someone who has ABSOLUTELY NOTHING to do with Cyberlife and C2 isn't gonna get you what you want either.}~\\

\emph{> Are you so sure?}~\\

If you can prove to me that saying something like the comment that prompted my initial reply to someone with no affiliation with Cyberlife whatsoever will get you C2, I'll be willing to insult people's right to live too, if it would get me some cool games.  <G>~\\

\emph{> >  Flame me for this post if you wish, the NORN adaptive shielding are long used to taking heat.}~\\

\emph{> How do they handle factual arguments? Or acid? I tend to use both...}~\\

Acid can be deflected by the shields.  Facts?  Well, at the risk of sounding like a broken record....~\\

Facts are irrelevant.  ;)~\\

\emph{>[19 lines of signature snipped]}~\\

Wow, is my sig really that long?  Not like I'm gonna shorten it or anything, I'm gonna take full advantage of my new unlimited sig space.  :)  BTW, you might wanna consider :) as an alternative to >:).  It just seems to convey more of that "devil-like" look to me.~\\

\emph{>"David push more sig space over Ping's dead body!"}~\\

Sorry for all this extra weight on your body Ping.~\\

-- -- ~\\
Daljit of NORN~\\
"We sacrifice Navens/Naven Boys!"~\\
ICQ UIN\#:  3206846~\\
Homepage:  \texttt{http://members.aol.com/djsred1/index.html}~\\

Daljit:  The NORN are members of the NDWAL, the SPCG, and the RfNS.~\\
Indigo:  ::brandishes Picket sign::  Get yes rights for Norns!~\\
Flame:  Sigfile shmigfile!  Err, get yes Norn Doll!~\\
Karma:  Grendels are your friends too!~\\
Nornan:  ::big nornish grin::~\\
Grndl:  ::big grendel grin::~\\

NORN Members:  Karma, Alyssa, Trissy, Flame~\\
Military/Strategic Advisor:  JDreddfull~\\
Head of Covert Operations:  DAL the self proclaimed demigod~\\
NORN Maintenance Director:  Shon~\\
Head of General Weirdness:  Spoon~\\
Official Counselor of the GroupMind:  CindyPsych~\\

 
		
	
		
\textbf{Ping3506 --- \emph{\texttt{28 feb 1998, 10:00}}}~\\

In article <34F6D62B.3...@sickofSPAM.erols.com>, "David ``I Don't Like SPAM'' Wood" <pyxis...@sickofSPAM.erols.com> writes:~\\
\emph{> expose Ping's biases.}~\\

Awww! I wanna' do that!!~\\

Ping -- Gamerous Addictedous.~\\
ICQ:  6283750~\\
*******Ping's Norns*************~\\
\texttt{http://www.crosswinds.net/birmingham/\textasciitilde norndude1/Index.HTM}~\\
**********************************~\\

 
		
	
		
\textbf{Ping3506 --- \emph{\texttt{28 feb 1998, 10:00}}}~\\

In article <bc-2702981251300...@cappella.apple.com>, b...@wetware.com (Fred Haineux) writes:~\\
\emph{> Nobody's "harassing" anyone except you.}~\\

Yeah right!!~\\

\emph{> Mac users said, "We'd like Creatures 2." Toby said "No." Most of us said, "OK."}~\\

Keywords here: MOST (eg. not all.  eg. not me)~\\

\emph{> And then Ping said, "OH MY GOD, WE'RE SURROUNDED BY ANGRY MAC USERS WHO WON'T STOP HARASSING US."}~\\

Now that's totally wrong! I never said anything of the sort.~\\

\emph{> Why are you so damn afraid of the Mac? Because you wish you'd bought one?}~\\

LOL!! Yes... I'm actually afraid that the 1 button mice, will suddenly overthrow the 3 button empire!~\\

Ping -- Gamerous Addictedous.~\\
ICQ:  6283750~\\
*******Ping's Norns*************~\\
\texttt{http://www.crosswinds.net/birmingham/\textasciitilde norndude1/Index.HTM}~\\
**********************************~\\

 
		
	
		
\textbf{Ping3506 --- \emph{\texttt{28 feb 1998, 10:00}}}~\\

What the hell did I do? I made a joke about the MAC users revolting! and I get this? I might as well keep my mouth shut, and only talk when spoken to! you always overeact on these things.... Fred!~\\

Ahhhh nuts!~\\

I was *JOKING* obviously MAC users cannot take a joke!!~\\

Ping -- Gamerous Addictedous.~\\
ICQ:  6283750~\\
*******Ping's Norns*************~\\
\texttt{http://www.crosswinds.net/birmingham/\textasciitilde norndude1/Index.HTM}~\\
**********************************~\\


		
\textbf{Ping3506 --- \emph{\texttt{28 feb 1998, 10:00}}}~\\

In article <6d7adh\$...@camel20.mindspring.com>, "Daljit of NORN"~\\

<N...@Collective.com> writes:~\\
\emph{> Sorry for all this extra weight on your body Ping.}~\\

::Daljit looks down and Suddenly Ping is wearing a huge metal battlesuit::~\\

HA HA HA!!~\\

::gets up::~\\

Ping -- Gamerous Addictedous.~\\
ICQ:  6283750~\\
*******Ping's Norns*************~\\
\texttt{http://www.crosswinds.net/birmingham/\textasciitilde norndude1/Index.HTM}~\\
**********************************~\\



\texttt{Sujet remplac{\'e} par "Object-Oriented Design discussion (MAC)" par jwbo...@mailcity.com}~\\
		
	
		
\textbf{jwbooth --- \emph{\texttt{28 feb 1998, 10:00}}}~\\

In article <bc-2702981259360...@cappella.apple.com>, b...@wetware.com (Fred Haineux) wrote:~\\
\emph{> I hope so, but so far Microsoft seems to be moving in the opposite direction. I wish I had more info and could say otherwise.}~\\

I've discovered that myself. For various reasons I've been trying to get the Visual C++ add in that generates code for the Mac. It's incredibly hard to get hold of. I can get the add on for Visual C++ Version 4 but I can't find out if a version 5 exists anywhere.~\\

Looks like I'll be sticking to Metroworks for that side of things.~\\

-- -- -- -- -- == Posted via Deja News, The Leader in Internet Discussion == -- -- -- -- -- ~\\
\texttt{http://www.dejanews.com/}   Now offering spam-free web-based newsreading~\\

 
		
	
		
\texttt{Sujet remplac{\'e} par "not available for MAC?" par Sparrow}~\\
		
	
		
\textbf{Sparrow --- \emph{\texttt{1 mar 1998, 10:00}}}~\\

I'm going to kill two [other] birds with one stone by replying to the reply:~\\

Ping3506 wrote:~\\
\emph{> In article <bc-2702981301480...@cappella.apple.com>, b...@wetware.com (Fred Haineux) writes:}~\\

\emph{> > I think Us Mac Users should just totally ignore Ping's rantings.}~\\

Practice what you preach.. enough said...~\\

\emph{> > He's so hung up on his PC Superiority Complex that he simply must try to prevent Mac users from having calm, rational discussions with Cyberlife and others.}~\\

Y'know, I doubt he or any other PC user is REALLY trying to hurt Mac users.~\\

\emph{> >!!!!!!!!!!!!!!!!!!!!!!!!!!!!!!!!!!!!!!!!!!!!!!!!!!!!!!!!!!!!!!!!!!!!}~\\
\emph{> >!!!!!!!!!THIS HURTS THE ENTIRE CREATURES COMMUNITY, PING.!!!!!!!!!!!}~\\
\emph{> >!!!!!!!!!!!!!!!!!!!!!!!!!!!!!!!!!!!!!!!!!!!!!!!!!!!!!!!!!!!!!!!!!!!!}~\\
\emph{> >!!!!!!!!!!!!!!!!!!!!     It hurts YOU.     !!!!!!!!!!!!!!!!!!!!!!!!!}~\\
\emph{> >!!!!!!!!!!!!!!!!!!!!!!!!!!!!!!!!!!!!!!!!!!!!!!!!!!!!!!!!!!!!!!!!!!!!}~\\

Please, Fred (if I may call you that), have an open mind and a sense of humor. A lack of a sense of humor on the part of Mac users (or any computer user for that matter) only hurts the Mac (or the other respective computer) community by creating the image that Mac users (or the users of the-- well, you get the idea) all have no sense of humor. I'm not saying stand by and let yourself be impaled by flamy PC users for using a Mac. I'm asking you to not reply so explosively to harmless Mac jokes, and TRY NOT to see every little comment about Macs or their users as a personal insult. Regardless of what you think, they aren't all meant to be insulting. Take my comment that started our argument (the one between you and I): The whole purpose of my post was to comment about how nice it would be to have total compatibility. Maybe I worded it a bit badly, but it still gave the general desired impression. At least I think so. Regardless, you saw it as a bomb thrown in your direction and the highest insult to Macs and their users. So, you started a flamewar.~\\

\emph{> > Ping, there are plenty of games and programs that are PC-only. Go be smug about them. Creatures is PC *and* Mac, and us Mac users are just as entitled to support, discussion, and the right not to be flamed just because we bought Not-Your-Personal-Favorite-Computer.}~\\

\emph{> > Whether YOU feel that Mac users should be important is irrelevant. You are irrelevant. Cyberlife feels that Mac users are valuable customers that should be supported. YOU don't get to vote on this.}~\\

Please tell me your not saying that PC users are irrelevent. That would be quite a contradiction on your part.~\\

\emph{> What the hell did I do? I made a joke about the MAC users revolting! and I get this? I might as well keep my mouth shut, and only talk when spoken to! you always overeact on these things.... Fred!}~\\

Sorry, but I must agree with Ping here. You do over-react.~\\

		
	
		
\textbf{Sparrow --- \emph{\texttt{1 mar 1998, 10:00}}}~\\

\emph{> > Sorry for all this extra weight on your body Ping.}~\\

\emph{>::Daljit looks down and Suddenly Ping is wearing a huge metal battlesuit::}~\\

Hmm.. talk about extra weight... 

	
	  	Messages 101 -- 112 sur 112 -- Tout r{\'e}duire  --  Traduire tous les contenus en Fran\c{c}ais 	< Plus anciens 
	
		
\textbf{David ``I Don't Like SPAM'' Wood --- \emph{\texttt{2 mar 1998, 10:00}}}~\\

Fred Haineux wrote:
\emph{> ping3...@aol.com (Ping3506) wrote:}~\\
\emph{> > Actually... I don't think it's fair that MAC users aren't getting Creatures 2, but I still don't think they should keep harassing everyone under the sun about it!}~\\
\emph{> Nobody's "harassing" anyone except you.}~\\

Well, let's be honest here. I've been using some harassing techniques too. Not at all PC users, just at Ping, really.

--David
Aspiring Carnivore

 
		
	
		
\textbf{David ``I Don't Like SPAM'' Wood --- \emph{\texttt{2 mar 1998, 10:00}}}~\\

Daljit of NORN wrote:

\emph{> > Awwwww, And I thought we were all just having fun.}~\\

\emph{> ::pulls out shotgun::  I wanna have fun too!  <G> This is how so many misunderstandings spring up on the Internet: everyone has slightly different boiling points. So when (A) says something slightly disparaging about (B) in fun, (B) may take it as slightly disparaging, or he might take it as a declaration of war.}~\\

\emph{> > EXERCISE: Consider that problem of communications for a moment, and then define a "troll" and a "flamer" in that context.}~\\

\emph{> OK, so you wanna be known as a troll, not a flamer?  ;)  But I see what you're saying. }~\\

Exactly! And from Ping's postings lately, I'd consider him something of a troll as well. And somehow, I don't see being a troll's troll as that bad a thing. Maybe it's because I'm becoming a psychotic sociopath...

\emph{> I'd get a Mac myself, but there are a few problems.}~\\

Nothing that can't be addressed...

\emph{> All my software instantly becomes garbage, }~\\

SoftWindows isn't a perfect solution, but it does provide a way of using much of your existing software base.

\emph{> I'm addicted to 2 button mice (I'd be willing to attend 2 button mouse-a-holics anon to fix this though), }~\\

How about getting a multi-button Mac mouse? Third parties make 2-button, 3-button, and 4-button mice, in traditional, track-ball, and glidepoint configurations. I've never needed one myself, but they can ease the transition...~\\

\emph{> and there's just not a lot of software for Macs.}~\\

14,000 titles isn't a lot? How big is YOUR hard drive?! A goodly number of PC programs are available for Mac. Some are available JUST for Mac. If you need to do it on a PC, odds are there is either a Mac version, or there's an analog on the Mac side which fills the same role.~\\

\emph{> I have to go with what works. }~\\

No, I'm not going to say the PC doesn't work. But the Mac works just as well in many cases, and better in several. BTW, prices on the G3 desktop models have been dropping, and QuickTime 3.0 is in its final candidate release.~\\

\emph{> > >  And saying something like the above statement, to anyone, especially someone who has ABSOLUTELY NOTHING to do with Cyberlife and C2 isn't gonna get you what you want either.}~\\

\emph{> > Are you so sure?}~\\

\emph{> If you can prove to me that saying something like the comment that prompted my initial reply to someone with no affiliation with Cyberlife whatsoever will get you C2, I'll be willing to insult people's right to live too, if it would get me some cool games.  <G>}~\\

No, what I meant was, are you so sure that I said what I did in order to get C2? From what you've already read, I had a different agenda for these postings, and in fact, I already GOT what I wanted.~\\

--David~\\
Aspiring (but still feeling rather peckish) Carnivore~\\

 
		
	
		
\textbf{Ping3506 --- \emph{\texttt{2 mar 1998, 10:00}}}~\\

In article <34FAAD02.2...@sickofSPAM.erols.com>, "David ``I Don't Like SPAM'' Wood" <pyxis...@sickofSPAM.erols.com> writes:

\emph{> Exactly! And from Ping's postings lately, I'd consider him something of a troll as well.}~\\

Why thank you, I'm a trainee actually, my Horns and tail are in the Mail...~\\

\emph{> Maybe it's because I'm becoming a psychotic sociopath...}~\\

Welcome to the club!

Ping -- Gamerous Addictedous.~\\
ICQ:  6283750~\\
*******Ping's Norns*************~\\
\texttt{http://www.crosswinds.net/birmingham/\textasciitilde norndude1/Index.HTM}~\\
**********************************~\\

 
		
	
		
\textbf{Daljit of NORN --- \emph{\texttt{2 mar 1998, 10:00}}}~\\

\emph{> Exactly! And from Ping's postings lately, I'd consider him something of a troll as well. And somehow, I don't see being a troll's troll as that bad a thing. Maybe it's because I'm becoming a psychotic sociopath...}~\\

LOL!~\\

\emph{> SoftWindows isn't a perfect solution, but it does provide a way of using much of your existing software base.}~\\

I take it this is a Windows Emulator type thing?~\\

\emph{> How about getting a multi-button Mac mouse? Third parties make 2-button, 3-button, and 4-button mice, in traditional, track-ball, and glidepoint configurations. I've never needed one myself, but they can ease the transition...}~\\

On a 2 button Mac mouse, would right clicking get me a context menu, where ever there was supposed to be one?~\\

\emph{>14,000 titles isn't a lot? How big is YOUR hard drive?! A goodly number of PC programs are available for Mac. Some are available JUST for Mac. If you need to do it on a PC, odds are there is either a Mac version, or there's an analog on the Mac side which fills the same role.}~\\

Haven't you read the origins of the NORN on my site (plug! plug! See Dan, I can do it too.  <G>)?  It says that I have the TerraDrive\texttrademark  that backs up data onto 1000 terrabyte cartridges at a rate of 1 terrabyte per sec.~\\ 
That's a hell of a lot of room.  Plus there's my 40,000 terrabyte hard drive.  :)  Is there a Mac version of MS Pub, Outlook Express, Photodeluxe, Kai's Power Goo, Cheyenne AntiVirus, NBA Live 95, Thunderbyte AntiVirus, and Word Pro?  Cause with those, I'll be set for life.  <G>  Oh, and can you hear .ra/.ram/.rm and .wav files on a Mac w/o any odd converters that require hours to figure out and don't work half the time (spoken from experience with certain sound file converters on PCs that shall go nameless)?  If the answers to the above are yes, then this June, rather than upgrading my current PC, I might decide to get a Mac instead.~\\

\emph{> No, what I meant was, are you so sure that I said what I did in order to get C2? From what you've already read, I had a different agenda for these postings, and in fact, I already GOT what I wanted.}

Ahhh, that's different.  I see what you're saying. It had seemed to me that you were lashing out at everyone cause you were in a pouty mood about not getting C2.  ;)~\\

-- ~\\
Daljit of NORN~\\
"We sacrifice Navens/Naven Boys!"~\\
ICQ UIN \#:  3206846~\\
Homepage:  \texttt{http://members.aol.com/djsred1/index.html}~\\

Daljit:  The NORN are members of the NDWAL, the SPCG, and the RfNS.~\\
Indigo:  ::brandishes Picket sign::  Get yes rights for Norns!~\\
Flame:  Sigfile shmigfile!  Err, get yes Norn Doll!~\\
Karma:  Grendels are your friends too!~\\
Nornan:  ::big nornish grin::~\\
Grndl:  ::big grendel grin::~\\

Military/Strategic Advisor:  Anthony~\\
Head of Covert Operations:  DAL~\\
NORN Maintenance Director:  Shon~\\
Head of General Weirdness:  Spoon~\\
Doctor of NORN Psychology:  CindyPsych~\\
Maintainer of the Karmic Shield:  Karma~\\
General Inspector Of Homepages and Insanity Quotients:  Flame~\\
Village Idiot:  Trissy~\\
Director of Dangerously Bored Drones:  Alyssa~\\

 
		
	
		
\textbf{Sparrow --- \emph{\texttt{2 mar 1998, 10:00}}}~\\

\emph{> > and there's just not a lot of software for Macs.}~\\

\emph{> 14,000 titles isn't a lot?}~\\

He means GOOD titles :)~\\

(for all you mac-user-flamers out there, this is a joke)~\\

 
		
	
		
\textbf{rajamaki --- \emph{\texttt{2 mar 1998, 10:00}}}~\\

Daljit of NORN <N...@Collective.com> wrote:
<snip>~\\
\emph{> > SoftWindows isn't a perfect solution, but it does provide a way of using much of your existing software base.}~\\

\emph{> I take it this is a Windows Emulator type thing?}~\\

yep, Then there's also Virtual PC and RealPC~\\

\emph{> > How about getting a multi-button Mac mouse? Third parties make 2-button, 3-button, and 4-button mice, in traditional, track-ball, and glidepoint configurations. I've never needed one myself, but they can ease the transition...}~\\

\emph{> On a 2 button Mac mouse, would right clicking get me a context menu, where ever there was supposed to be one?}~\\

You can program the right click to be whatever you want. If you want it to be the contextual menu, you program it for control-click which is the mac equivalent of a right click. But you can also program it to be auto-fire on your favorite shoot 'em game or to make you toast ;-)~\\

I have no clue what MS Pub is, sounds like a web editor. I don't know if it exists for mac, but there is Claris Homapage which is so easy even an idiot can use it. For email I recommend Claris Emailer, don't know about Outlook. May I suggest you point the NORN browser towards \texttt{www.micro\$soft.com} :-) Photodeluxe? That's Adobe right? Yes it exists amd Kai's Power Goo does as well.~\\

Again I don't know Cheyenne anti-virus, but there's Symantec Anti-Virus for Mac; Virex and Disinfectant which is totally free.~\\

As for NBA '95, I doubt it...  :-( But that just one silly game and you can always play real basketball :-) And Word Pro? You mean like Word '98? Well that just came out. Then there's ClarisWorks which I prefer and loads of others.~\\

As for sound files, well Simple Sound comes with the OS and can covert a whole bunch of different sounds. I know you can listen to wavs with it. .ra/.ram/ those are real audio right? Well you can download the real audio player from the same place you get the pc versiona nd it proabably works just the same.~\\

And if you're still in doubt, head over to:~\\
\texttt{http://macspeedzone.com/4.0/WindowsvsMac.html}~\\
\texttt{http://fampm201.tu-graz.ac.at/karl/timings30.html}~\\

for some interwesting speed tests...~\\

::Takes Daljit's hand and leads him to the friendly Mac retailer:: Shall we?~\\

\emph{> > No, what I meant was, are you so sure that I said what I did in order to get C2? From what you've already read, I had a different agenda for these postings, and in fact, I already GOT what I wanted. }~\\

\emph{> Ahhh, that's different.  I see what you're saying. It had seemed to me that you were lashing out at everyone cause you were in a pouty mood about not getting C2.  ;)}~\\

I know I was in a worse than pouty mood when I heard the announcement. But now I'm over that and happy that I'm getting my object injector :-D~\\

/Sandy of NORN -- Goddess of MacNORNs~\\
--- --- --- --- --- --- --- --- --- ---~\\
MacNORNs -- \texttt{http://www.crosswinds.net/brussels/\textasciitilde rajamaki}~\\
ICQ\# 7029961~\\

-- -- -- -- -- == Posted via Deja News, The Leader in Internet Discussion == -- -- -- -- -- ~\\
\texttt{http://www.dejanews.com/}   Now offering spam-free web-based newsreading~\\

 
		
	
		
\textbf{rajamaki --- \emph{\texttt{2 mar 1998, 10:00}}}~\\

Sparrow <terreoBOI...@geocities.com> wrote:~\\
\emph{> > > and there's just not a lot of software for Macs.}~\\

\emph{> > 14,000 titles isn't a lot?}~\\

\emph{> He means GOOD titles :) (for all you mac-user-flamers out there, this is a joke)}~\\

There are only GOOD titles on the Mac :-) (not a joke or a flame)~\\

/Sandy of NORN -- Goddess of MacNORNs~\\
--- --- --- --- --- --- --- --- --- ---~\\
MacNORNs -- \texttt{http://www.crosswinds.net/brussels/\textasciitilde rajamaki}~\\
ICQ\# 7029961~\\

-- -- -- -- -- == Posted via Deja News, The Leader in Internet Discussion == -- -- -- -- -- ~\\
\texttt{http://www.dejanews.com/}   Now offering spam-free web-based newsreading~\\

 
		
	
		
\textbf{Daljit of NORN --- \emph{\texttt{3 mar 1998, 10:00}}}~\\

\emph{> >  > SoftWindows isn't a perfect solution, but it does provide a way of using much of your existing software base.}~\\

\emph{> >  I take it this is a Windows Emulator type thing?}~\\

\emph{> yep, Then there's also Virtual PC and RealPC}~\\

Oooh, maybe I can use this to run those windows programs that don't have Mac versions that I like.~\\

\emph{> You can program the right click to be whatever you want. If you want it to be the contextual menu, you program it for control-click which is the Mac equivalent of a right click.}~\\

Woo hoo!~\\

\emph{> But you can also program it to be auto-fire on your favorite shoot 'em game or to make you toast ;-)}~\\

Hmmm . . . ;)~\\

\emph{> I have no clue what MS Pub is, sounds like a web editor.}~\\

MS Publisher 97.  It's a program that creates cards, posters, and the like. It also has a thing that lets you create origami stuff and paper airplanes. Those are cool.  :)  But I use it mainly for web editing.  It's insanely easy to use.~\\

\emph{> I don't know if it exists for Mac, but there is Claris Homapage which is so easy even an idiot can use it.}~\\

Are you implying something here!?  ;)~\\

\emph{> For e-mail I recommend Claris Emailer, don't know about Outlook.}~\\
\emph{> May I suggest you point the NORN browser towards \texttt{www.micro\$soft.com} :-)}~\\

ROFL!~\\

\emph{> Photodeluxe? That's Adobe right? Yes it exists amd Kai's Power Goo does as well. }~\\

Woo hoo!  I love those programs.  Especially Power Goo, you can have so much fun with that.  ;)~\\

\emph{> Again I don't know Cheyenne anti-virus, but there's Symantec Anti-Virus for Mac; Virex and Disinfectant which is totally free.}~\\

As long as they work well and can detect a lot of unknown stuff, then I'm happy.~\\

\emph{> As for NBA '95, I doubt it...}~\\

DOH!!!!~\\

\emph{> But that just one silly game}~\\

::gasp::~\\

\emph{> and you can always play real basketball :-)}~\\

Well yeah, and I'm good at real basketball, but . . . but . . . the weather isn't always nice enough for me to go outside and where else can I take out my frustration on particular NBA players by running them over with my little character.  ;)~\\

\emph{> And Word Pro? You mean like Word '98? }~\\

!!!!!!!  Do not EVER compare the filthy Word 98 which is a pathetic excuse for software with the almighty Lotus WordPro 97.  Ha, the 16 bit version of Word Pro 96/97 (Ami Pro 3.1) was found to be more stable, easier to use, had a better interface, the list goes on than MS Word 97.  98 isn't much better. I am thoroughly disgusted with that line of MS products, even more than normal.~\\

\emph{> Well that just came out. Then there's ClarisWorks which I prefer and loads of others. }~\\

Nothing compares to the god-like Word Pro 97.  ;)  Especially now since it works with IBM Via Voice.  "I talk, it types."~\\

\emph{> As for sound files, well Simple Sound comes with the OS and can covert a whole bunch of different sounds. I know you can listen to wavs with it. .ra/.ram/ those are real audio right? Well you can download the real audio player from the same place you get the pc versiona and it proabably works just the same. }~\\

Ok, cool.~\\

\emph{> And if you're still in doubt, head over to:}~\\

\emph{>\texttt{http://macspeedzone.com/4.0/WindowsvsMac.html}}~\\
\emph{>\texttt{http://fampm201.tu-graz.ac.at/karl/timings30.html}}~\\

\emph{> for some interwesting speed tests...}~\\

What is this, Mac propoganda?  ;)

\emph{>::Takes Daljit's hand and leads him to the friendly Mac retailer:: Shall}~\\

we?~\\

::takes a hold of Sandy's hand and says excitedly, "Oh I do hope they have one I like!"::~\\

And in closing, I'd just like to ask, "Who's assimilating who here?!"  ;)~\\

-- ~\\
Daljit of NORN~\\
"We sacrifice Navens/Naven Boys!"~\\
ICQ UIN \#:  3206846~\\
Homepage:  \texttt{http://members.aol.com/djsred1/index.html}~\\

Daljit:  The NORN are members of the NDWAL, the SPCG, and the RfNS.~\\
Indigo:  ::brandishes Picket sign::  Get yes rights for Norns!~\\
Flame:  Sigfile shmigfile!  Err, get yes Norn Doll!~\\
Karma:  Grendels are your friends too!~\\
Nornan:  ::big nornish grin::~\\
Grndl:  ::big grendel grin::~\\

Goddess of the MacNORN branch:  Sandy~\\
Military/Strategic Advisor:  Anthony~\\
Head of Covert Operations:  DAL~\\
NORN Maintenance Director:  Shon~\\
Head of General Weirdness:  Spoon~\\
Doctor of NORN Psychology:  CindyPsych~\\
Maintainer of the Karmic Shield:  Karma~\\
General Inspector Of Homepages and Insanity Quotients:  Flame~\\
Village Idiot:  Trissy~\\
Director of Dangerously Bored Drones:  Alyssa~\\

 
		
	
		
\textbf{Fred Haineux --- \emph{\texttt{3 mar 1998, 10:00}}}~\\

ping3...@aol.com (Ping3506) wrote:~\\
\emph{> I was *JOKING* obviously MAC users cannot take a joke!!}~\\

So let's review, shall we:~\\

Ping writes venomous flame about how terrible Mac users are, claims that Mac users are revolting, etc. This, we are told by Ping, is his perfect right. It is, he says, a JOKE.~\\

Meanwhile, when a Mac user responds to his flame, that's HARASSMENT.~\\

Uh hunh.~\\

Look, Ping, maybe you don't intend to annoy Mac users (although I find that really hard to believe, considering what you post).~\\

It doesn't mean you can't joke about Macs. Heck, I do all the time. It's when you repeatedly post extremely unpleasant sentiments -- "Mac users shouldn't expect tech support," for example -- that people stop looking at your jokes as jokes, and instead start seeing them as "yet more abusive flamage."~\\

Get it?~\\

 
		
	
		
\texttt{Sujet remplac{\'e} par "A short answer to a hard question" par Fred Haineux}~\\
		
	
		
\textbf{Fred Haineux --- \emph{\texttt{3 mar 1998, 10:00}}}~\\

I wrote:~\\
\emph{> > > He's so hung up on his PC Superiority Complex that he simply must try to prevent Mac users from having calm, rational discussions with Cyberlife and others.}~\\

Sparrow wrote:~\\
\emph{> Y'know, I doubt he or any other PC user is REALLY trying to hurt Mac users.}~\\

Perhaps, perhaps not.~\\

But the practical result is that a lot of the Mac users are just going to killfile Ping, and therefore never see anything he ever says. Apparently, most of what he says is "mac users are whiny" but I have seen a few intelligent postings come out of him. Too bad. Throw out the baby with the bath water, and all that.~\\

Another brief point: If a given user only has 20 minutes per day to read the Creatures newsgroup, do you think that they are going to see the intelligent discussion in between the miles and miles of Mac-versus-PC flamewars? Or will they just unsubscribe and go away, leaving nothing but flamers?~\\

 
		
	
		
\texttt{Sujet remplac{\'e} par "not available for MAC?" par Sparrow}~\\
		
	
		
\textbf{Sparrow --- \emph{\texttt{3 mar 1998, 10:00}}}~\\

\emph{> I take it this is a Windows Emulator type thing? }~\\

Ya know, there are also Mac emulators out there :)

 
		
	
		
\texttt{Sujet remplac{\'e} par "A short answer to a hard question" par Ping3506}~\\
		
	
		
\textbf{Ping3506 --- \emph{\texttt{4 mar 1998, 10:00}}}~\\

In article <bc-0303981533480...@cappella.apple.com>, b...@wetware.com (Fred Haineux) writes:
\emph{> But the practical result is that a lot of the Mac users are just going to killfile Ping, and therefore never see anything he ever says. }~\\

So what? half the people who post here Block my messages..... it's nothing new!

\emph{> Apparently, > most of what he says is "mac users are whiny" }~\\

Ok... FRED! let me help you get this through your skull... it (bear with me) was (here's the tricky part) a (ouch) JOKE!!! (did you get all that?, or do you want me to type it all in CAPS!!!!! or shall I give it to you in your native tounge? ok.... oo oo oohh ohh ook, ahh oooo ohhh o ooo ahh ook ohhh oooo!)

\emph{> but I have seen a few intelligent postings come out of him. }~\\

News to me! hey, fwd them to me will ya!~\\

\emph{> Too bad. Throw out the baby with the bath water, and all that. }~\\

Huh?~\\

\emph{> Another brief point: If a given user only has 20 minutes per day to read the Creatures newsgroup, do you think that they are going to see the intelligent discussion in between the miles and miles of Mac-versus-PC flamewars? Or will they just unsubscribe and go away, leaving nothing but flamers?}~\\

Heeeeeeey! I don't know who your talking about, by my Newsreader downloads all the posts, and I can read them for however long I want.  But if your talking about Sparrow: shut up! and err..... stop FLAMING us PC users.~\\

Ping -- Gamerous Addictedous.~\\
ICQ:  6283750~\\
*******Ping's Norns*************~\\
\texttt{http://www.crosswinds.net/birmingham/\textasciitilde norndude1/Index.HTM}~\\
********************************** ~\\

\end{multicols*}

\end{document}
