\documentclass[11pt,twoside,a4paper]{article}
% http://www-h.eng.cam.ac.uk/help/tpl/textprocessing/latex_maths+pix/node6.html symboles de math
% http://fr.wikibooks.org/wiki/Programmation_LaTeX Programmation latex (wikibook)
%=========================== En-Tete =================================
%--- Insertion de paquetages (optionnel) ---
\usepackage[french]{babel}   % pour dire que le texte est en fran{\'e}ais
\usepackage{a4}	             % pour la taille   
\usepackage[T1]{fontenc}     % pour les font postscript
\usepackage{epsfig}          % pour gerer les images
%\usepackage{psfig}
\usepackage{amsmath, amsthm} % tres bon mode mathematique
\usepackage{amsfonts,amssymb}% permet la definition des ensembles
\usepackage{float}           % pour le placement des figure
\usepackage{verbatim}

\usepackage{longtable} % pour les tableaux de plusieurs pages

\usepackage[table]{xcolor} % couleur de fond des cellules de tableaux

\usepackage{lscape} % changement orientation page
%\usepackage{frbib} % enlever pour obtenir references en anglais
% --- style de page (pour les en-tete) ---
\pagestyle{empty}

% % % en-tete et pieds de page configurables : fancyhdr.sty

% http://www.trustonme.net/didactels/250.html

% http://ww3.ac-poitiers.fr/math/tex/pratique/entete/entete.htm
% http://www.ctan.org/tex-archive/macros/latex/contrib/fancyhdr/fancyhdr.pdf
% \usepackage{fancyhdr}
% \pagestyle{fancy}
% \renewcommand{\chaptermark}[1]{\markboth{#1}{}}
% \renewcommand{\sectionmark}[1]{\markright{\thesection\ #1}}
% \fancyhf{}
% \fancyhead[LE,RO]{\bfseries\thepage}
% \fancyhead[LO]{\bfseries\rightmark}
% \fancyhead[RE]{\bfseries\leftmark}
% \renewcommand{\headrulewidth}{0.5pt}
% \renewcommand{\footrulewidth}{0pt}
% \addtolength{\headheight}{0.5pt}
% \fancypagestyle{plain}{
% 	\fancyhead{}
% 	\renewcommand{\headrulewidth}{0pt}
% }

\author{Gabriel Chandesris}
\title{Proposition de stage -- <<Conception et d{\'e}veloppement d'un syst{\`e}me de mod{\'e}lisation biologique et d'outils d'analyses associ{\'e}s>>}
\date{26 avril 2009}

\usepackage[paper=a4paper,tmargin=1cm,bmargin=1cm,lmargin=1cm,rmargin=1cm]
	{geometry}

%--- Definitions de nouvelles commandes ---
\newcommand{\N}{\mathbb{N}} % les entiers naturels

%============================= Corps =================================
\begin{document}

\begin{center}
\textbf{<<Conception et d{\'e}veloppement d'un syst{\`e}me~\\ de mod{\'e}lisation biologique et d'outils d'analyses associ{\'e}s>>}
\end{center}

L'objectif est de cr{\'e}er un logiciel informatique de simulation {\`a} base d'agents mod{\'e}lisant un syst{\`e}me biologique (ensemble de cellules, d'animaux, de plantes...) et leurs interactions au  sein d'un environnement.
\begin{itemize}
	\item d{\'e}finition des agents et de leurs propri{\'e}t{\'e}s, ainsi que l'environnement h{\^o}te (cellules en cultures dans des boites de P{\'e}tri, plantes et animaux dans une serre...)  
	\item outils d'analyses du mod{\`e}le (dans sa globalit{\'e}) et des {\'e}l{\'e}ments du mod{\`e}le (milieu, agents, propri{\'e}t{\'e}s...)
\end{itemize}~\\

\textbf{\underline{Outils {\`a} construire autour de la mod{\'e}lisation : }}
\begin{itemize}
	\item[ ] \underline{Interaction avec les agents de la simulation}
	\begin{itemize}
		\item[ ] Au sein de l'environnement, ou avec celui-ci (outils d'extractions,  d'introduction, d'analyses automatique en son sein), Et {\'e}galement <<hors environnement>> :
		\begin{itemize}
			\item[ ] Analyse distincte des agents et de leurs composantes et propri{\'e}t{\'e}s
			\item[ ] Modifications des agents, croisement de leurs propri{\'e}t{\'e}s
		\end{itemize}
	\end{itemize}
	\item[ ] \underline{Outils globaux d'analyse}, impact sur le d{\'e}roulement de la simulation, indications sur des {\'e}l{\'e}ments pertinents. Ces outils sont {\`a} int{\'e}grer au logiciel principal de mod{\'e}lisation.
\end{itemize}~\\

\textbf{\underline{T{\^a}ches pour le d{\'e}veloppeur / stagiaire : }}
\begin{itemize}
	\item[ ] Bibliographie et orientation des choix de conception et d'impl{\'e}mentation
	\item[ ] Langage de d{\'e}veloppement (de pr{\'e}f{\'e}rence objet : JAVA, C++...)
	\item[ ] Choix des propri{\'e}t{\'e}s des {\'e}l{\'e}ments au sein du ou des mod{\`e}les impl{\'e}ment{\'e}s
	\begin{itemize}
		\item[ ] Agents (comportement, g{\'e}n{\'e}tique, interactions)
		\item[ ] Environnement (propri{\'e}t{\'e}s internes, support des agents)
		\item[ ] Outils d'interaction et outils d'analyse globale
		\item[ ] R{\'e}sultats obtenus : exemples
	\end{itemize}
\end{itemize}

\nocite{*}
%toutes references biblio : 6 lettres + 2 chiffres
\bibliography{proposition_stage}
\bibliographystyle{frplain} % plain or frplain
\setcounter{page}{0}
\thispagestyle{empty}
\clearpage
\end{document}